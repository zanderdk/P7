\chapter{Reflection on Development Method}\label{chap:devmethodreflection}
This chapter discusses and reflects on the development method used during this project.

\section{Development Method Description}
This project deals with many uncertain factors and covers a theoretical and practical area which is new for us. Since it is notoriously difficult to plan in the presence of large uncertainties, we anticipated a considerable likelihood of change in major parts of the project. To combat this, we need a way of continuously refining our work, and to be able to quickly accommodate a change. As such we steered away from analysis-heavy approaches such as the waterfall model or OOA\&D\@.

At the other end of the spectrum, we discussed agile methods. While agile methods fit well within the scope of uncertainties and often changing requirements, some group members felt that most popular development frameworks such as Scrum and XP bring too much overhead to the development process, and thought the value of such frameworks was negligible compared to the overhead of ceremonies.

We ended up agreeing on tailoring our own development process with inspiration from well-known methods. This included task management on paper cards as well as a daily ceremony similar in structure to a stand-up meeting from Scrum. We also had a ``Sprint 1'' planning. The development time had no real structure and was self-organized.

In general, we have used a very lightweight development process. As such, we employ a more self-organizing method.
\todo{\url{https://en.wikipedia.org/wiki/Cowboy_coding} seems to fit our process pretty well.}

In the beginning, our process 

However, tasks quickly depleted and new ones were not added as frequently. We have spontaneously added tasks from time to time, however during a large proportion of the time, no tasks were available. Likewise, the frequency of the daily ceremony also faded.

\section{Evaluation}
\todo{Daily ``stand-up'' meetings turned into one hour discussions.}

\section{Reflection}