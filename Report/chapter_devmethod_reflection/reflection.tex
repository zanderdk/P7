\chapter{Reflection on Development}\label{chap:devmethodreflection}
This chapter discusses and reflects on the development method used during this project. \todo{write this introduction}

\section{Development Method Description}
This project deals with many uncertain factors and covers a theoretical and practical area which is new for us. Since it is notoriously difficult to plan in the presence of large uncertainties, we anticipated a considerable likelihood of change in major parts of the project. To combat this, we needed a way of continuously refining our work, and to be able to quickly accommodate a change. As such, we steered away from analysis-heavy approaches such as the waterfall model or OOA\&D\@.

At the other end of the spectrum, we discussed agile methods. While agile methods fit well within the scope of uncertainties and changing requirements, part of the team felt that most popular development frameworks such as Scrum and XP bring too much overhead to the development process. Overhead in this context is various ceremonies and practices such as pair programming. If the entire team is not keen on using such methods, it will eventually become cumbersome to practice it. Therefore, the value of these frameworks would at best be negligible in the long run.

We ended up tailoring our own development process with inspiration from popular development frameworks. We chose a self-organized development method with the majority of the time allocated for work. We also adopted task management on paper cards as well as a daily ceremony similar in structure to a stand-up meeting from Scrum. Furthermore we had an informal sprint 0~\cite{sprint-zero}, however we chose not to have any more sprints, because part of the team felt that sprints constrain development into rigid boxes. Furthermore, no roles were chosen since they would not support any of the chosen artifacts or ceremonies.

\section{Evaluation and Reflection}
Our development process has given us a large fraction of the time to spend on development. This has indeed enabled much work to be done. 

In the beginning, we stuck to the defined process. However, as part of the team was not keen on utilizing our lightweight development method, its value soon began to decrease. Tasks quickly depleted and new ones were not added as frequently. We spontaneously added tasks from time to time, however during a large proportion of the time, not enough tasks were available. This sometimes caused some suboptimal division of labor, because part of the team would be busy with their own thing, while the rest was struggling to come up with new tasks.

Despite this, we did manage to complete a large body of work. Driven by an interest in the project, we compensated for the lack of a proper development method. Some team members thrived in this setting, but this is of course a matter of personal preference.

%Ville vi have nået mere, hvis vi fulgte en udviklingsproces?
%Ville vi have haft en mere ligelig arbejdsbyrde, både på tværs af medlemmerne i gruppen, men også fordelt over hver dag, og over hele projektperioden?
A more well-developed development method would probably have been beneficial during some periods in the project --- especially when we had no clear direction in which to go. We would probably also have had a more balanced division of labor as well as a more sustainable pace across the weeks.

The code written in the project is not overly complicated; A large part of the project requires setting up and using tools as well as handling and transforming data. For this, our development method proved to work reasonably well.

% Practices such as pair programming was not used

% the frequency of the daily ceremony also faded, and instead turned into team-wide discussions, often spanning more than an hour.

% Arbejde hjemmefra og om lørdag/søndagen. Manglende vidensdeling. dette skabte ``specialister''.