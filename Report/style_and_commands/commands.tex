% Project related.
  % Project information.
  \newcommand{\projecttitle}{Project Title}
  \newcommand{\projecttheme}{Internet Technologies}
  \newcommand{\projectperiod}{Fall semester 2016}
  \newcommand{\projectcopies}{6}
  \newcommand{\projectcompletion}{December 21, 2016}

  % Names.
  \newcommand{\groupname}{sw705e16}
  \newcommand{\supervisor}{Nattiya Kanhabua}
  \newcommand{\groupmembersbyfirstname}[0]{%
    Alexander Krog\\
    Kasper Kohsel Terndrup\\
		Lasse Just Petersen\\
		Mathias Vestergaard Rasmussen\\
    Michael Toft Jensen\\
		Simon Vandel Sillesen
  }

% Custom styling commands.
  % Monospaced text.
  \newcommand*\justify{%
    \fontdimen2\font=0.4em% interword space
    \fontdimen3\font=0.2em% interword stretch
    \fontdimen4\font=0.1em% interword shrink
    \fontdimen7\font=0.1em% extra space
    \hyphenchar\font=`\-% allowing hyphenation
  }
  \newcommand{\mono}[1]{\texttt{\justify {#1}}}

  % Code mono
  \newcommand{\code}[1]{\sethlcolor{clcodeshade}\hl{\texttt{\justify{#1}}}} % Yes, this line is identical to the one in \mono, however the soul-package is fragile, and cannot see into another command.
  \soulregister{\justify}{1}

  % Chapter introductions.
  \newcommand{\chapterintro}[1]{#1}

  % CD paths with icon.
  %\newcommand{\cdpath}[1]{%
  %  % Param1: path
  %  \hyperref[app:cd]{\raisebox{-0.28ex}{\includegraphics[height=0.85em]{cd}}\mono{/#1}}%
  %}

  % Margin text.
  \newcommand{\margintext}[1]{\marginline{\textsf{\footnotesize #1}}}

  % Todo.
  \newcommand{\todo}[1]{\fxnote{#1}}

  % Dummy line.
  \newcommand{\dummy}{\todo{\color{black}Lorem ipsum dolor sit amet, consectetur adipiscing elit.}}

% Document structure.
  % Sprints.

  % Document organization.
  \newenvironment{documentorganization}
    {\vspace{.5cm}\noindent\textbf{Report Organization}\quad The remainder of this report is organized in the following fashion:\begin{itemize}}
    {\end{itemize}}

  % Chapter organization.
  \newenvironment{chapterorganization}
    {\vspace{.5cm}\noindent\textbf{Chapter Organization}\quad This chapter is organized in the following fashion:\begin{itemize}}
    {\end{itemize}}

   % Abbreviation.
  \newenvironment{abbreviations}
    {\vspace{.5cm}\noindent\textbf{Chapter Abbreviations}\quad This chapter introduces the following abbreviations:\begin{addmargin}[\leftmargin]{0em}\begin{multicols}{2}\begin{description}[noitemsep, style=sameline]}
    {\end{description}\end{multicols}\end{addmargin}}

    % Dates.
    %\newenvironment{dates}
    %{\vspace{.5cm}\noindent\textbf{Project Dates}\quad \dummy:\begin{addmargin}[\leftmargin]{0em}\begin{multicols}{2}\begin{description}[noitemsep, style=sameline]}
    %{\end{description}\end{multicols}\end{addmargin}}

% Figures.
  % Normal figure.
  \newcommand{\fig}[3]{
    \begin{figure}[tbp]
      \centering
      %\rule{\textwidth}{0.005in}
      \includegraphics[width=0.75\textwidth]{#1}
      \caption[#2]{#3}\label{fig:#1}
      %\rule{\textwidth}{0.005in}
    \end{figure}
  }

  % Scaled figure.
  \newcommand{\figscaled}[4]{
    \begin{figure}[tbp]
      \centering
      \includegraphics[scale=#4]{#1}
      \caption[#2]{#3}\label{fig:#1}
    \end{figure}
  }

  % Figure with custom width.
  \newcommand{\figcustomwidth}[4]{
    \begin{figure}[tbp]
      \centering
      \includegraphics[width=#4]{#1}
      \caption[#2]{#3}\label{fig:#1}
    \end{figure}
  }

% References.
  \newcommand{\appendixref}[1]{\hyperref[#1]{Appendix \ref*{#1}}}
  \newcommand{\chapterref}[1]{\hyperref[#1]{Chapter \ref*{#1}}}
  \newcommand{\sectionref}[1]{\hyperref[#1]{Section \ref*{#1}}}
  \newcommand{\figureref}[1]{\hyperref[#1]{Figure \ref*{#1}}}
  \newcommand{\tableref}[1]{\hyperref[#1]{Table \ref*{#1}}}
  \newcommand{\listingref}[1]{\hyperref[#1]{Listing \ref*{#1}}}
  \newcommand{\onpage}[1]{on \hyperref[#1]{page \pageref*{#1}}}
  \newcommand{\equationref}[1]{\hyperref[#1]{Equation \ref*{#1}}}
  \newcommand{\algorithmref}[1]{\hyperref[#1]{Algorithm \ref*{#1}}}
  %\newcommand{\sprintref}[1]{\hyperref[#1]{Sprint \ref*{#1}}}

% Column types for tabulars.
%\newcolumntype{L}[1]{>{\raggedright\let\newline\\\arraybackslash\hspace{0pt}}m{#1}}
%\newcolumntype{R}[1]{>{\raggedleft\let\newline\\\arraybackslash\hspace{0pt}}m{#1}}

% Fix todo notes with externalize
\let\oldTodo\todo
\renewcommand{\todo}[1]{\tikzexternaldisable{}\oldTodo{#1}\tikzexternalenable{}}

% Fixme stuff.
%\newcommand{\fillin}[1]{\fxnote*[inline]{#1}{Fill in: }}

% Vector command.
%\newcommand{\omatrix}[1]{\ensuremath{\boldsymbol{#1}}}

% A better plus minus sign.
\makeatletter
\newcommand{\gpm}{\mathbin{\mathpalette\@gpm\relax}}
\newcommand{\@gpm}[2]{\ooalign{%
  \raisebox{.1\height}{$#1+$}\cr
  \smash{\raisebox{-.6\height}{$#1-$}}\cr}}
\makeatother

% Nice shaded box for formal text.
\newmdenv[
  nobreak=true,
  suppressfirstparskip,
  topline=false,
  rightline=false,
  bottomline=false,
  linewidth=2.5pt,
  linecolor=clframe,
  backgroundcolor=clshade]{formal}
  