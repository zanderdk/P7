\section{Related Work}\label{sec:related_work}

Wikipedia is the subject of multiple papers and research projects~\cite{wiki-research-newsletter}, and so there is much material to gain inspiration from. But before we can draw inspiration from solutions to related problems, we must consider what our problem is. As with almost all problems, this problem can be perceived from multiple angles, and modeled by just as many. Here we will consider some general approaches.

\subsection{Semantic and Contextual Relations}\label{related_semantic_contextual}

The linking problem can be seen as a problem of deducing semantic or contextual relations. Solutions to this problem usually involves a degree of textual analysis in order to determine whether a piece of text refers to another subject. There are usually two parts to this problem, which are each given a different level of importance, depending on the solution. First we got a syntactical recognition of references to some subject. One approach is to find keywords or shingles using some n-gram technique and matching those to subjects, as seen in~\cite{mihalcea2007wikify}. 

Secondly there is semantic determination, where possible syntactical disambiguation is combatted. A prime example here is to determine whether the syntactical reference to a tree, semantically refers to a data structure or a type of plant. One of many ways of approaching this problem is to train a classifier, as seen in~\cite{milne2008learning}. Here they attempt to determine the semantic reference by training on metrics for commonness, relatedness and context quality. In short, commonness is the probability distribution of references, relatedness is a measure of similarity between the referrer and the possible referenced subject, and context quality is a measure of whether a given term is usually linked. The example given in~\cite{milne2008learning} is the English grammatical article \enquote{The}, which is used often, but rarely links to the subject article.

\subsection{Structural Analysis}\label{related_structural_analysis}

Another way of viewing the linking problem is as missing structural connections in a dataset. With this approach, solutions attempt to analyze a structure in order to gain insight into possible patterns that influences linking. For Wikipedia the most relevant structure to consider is a graph structure with articles as vertices. There are different ways to model edges in the graph. Examples of this include~\cite{hyperlink-structure-using-logs} and~\cite{west2015mining} where the authors create edges from the navigation of Wikipedias visitors, based on the belief that an optimal linking structure can be deduced from user behavior. Of course anything that can be considered a relation between any two articles, can be the basis for a set of edges, as explored in~\cite{lu2011link}.

Regardless of the approach a metric for a good link must be considered. We have already introduced Wikipedia guidelines on linking~\cite{wiki-editor-guidelines}, which is Wikipedias own view of a good set of metrics. However, even though they clearly hold authority on the matter, there exists alternatives worth considering. In~\cite{hyperlink-structure-using-logs} and~\cite{west2015mining} they consider good links to be ones that are in use, and as such they rank their results based on server logs. This technique rates the amount of clicks a given link receives out of the total visitors on the page, based on the idea that the most useful links get the most clicks, and that all links should at least get some clicks. The biggest argument for this technique of finding a clickthrough rate is that every link can be objectively measured and compared. However, a drawback with the technique is that links become competitors, and will be optimized towards obtaining the most traffic for themselves, instead of being optimized as a part of a collective.

\subsection{Machine Learning}\label{related_machine_learning}

A theme that occurs repeatedly in related work with a semantic approach is machine learning. In~\cite{mihalcea2007wikify} and~\cite{milne2008learning} they both work with a naive bayes classifer, with the latter also testing different C4.5s and support vector machines, for the purpose of classifying matters of ambiguity and relatedness. Both~\cite{hyperlink-structure-using-logs} and~\cite{west2015mining} avoids this by employing the clickthrough rate as their primary measure, giving them fewer problems with ambiguity.

An approach we have seen less of, is the use of machine learning to improve Wikipedia links in a structural setting. But using machine learning for predicting links in graphs is not an unexplored concept, as~\cite{tang2015line} and~\cite{al2006link} are just two examples of. Due to this, we wish to explore the idea of using machine learning techniques to improve linking on Wikipedia through a graph structure approach.


%Wikify! Linking Documents to Encyclopedic Knowledge 		mihalcea2007wikify
%Learning to Link with Wikipedia 							milne2008learning
%Hyperlink Structure Using Server logs						hyperlink-structure-using-logs	
%Human Navigation Traces									west2015mining 				
%Prediction in Complex Networks 							lu2011link
%Link Prediction using Supervised Learning 					al2006link
%LINE algo													tang2015line