\section{Summary}

After examining different aspects of graph modeling and machine learning, and our choices within these areas, we consider the implications of those choices in terms the development of a system.

The direction that our choices have moved our solution towards, is one with multiple interesting and complex areas.%If we where to fully explore each of these areas we would cover material enough for entire projects on data analysis, feature learning, classification, and web application development.
If we where to fully explore each of these areas, we would have to do extensive analysis and development within the subjects of data analysis, feature learning, classification, and web application development.

An approach to this problem of subject areas, could be to focus equally on all these aspects and thereby developing a [adjective] [Type-description] solution. While this might be the most suitable development method in another setting, the interest in exploring a less common approach to the linking problem, that is expressed in \cref{sec:problem_statement}, gives this project an experimental direction. Because of this, we choose to focus the development on attempting to break new ground in what we believe to be the area, where the stated problem diverges from those seen in related work. We consider this area to be feature learning. 

While it can be argued that the core of the problem is classification, and perhaps rightly so, we believe that once the area of feature learning, in the context of Wikipedia, has been covered, the problem of classification can be approached with much inspiration from related work on machine learning. It should also be considered that an attempt to focus on classification based on a run-of-the-mill feature learning approach, could result in an inheirent inability to identify discriminating information within the features. We believe that the preferable approach is to focus on feature generation and then accepting a less specialized classification process.

\todo{fix square brackets}
\todo{consequences for the project, and report. Træk lidt i land, mht. det store focus på feature learning, og nævn at vi altså stadig dækker hele problemstillingen.}