\section{Development Focus}
After examining aspects of graph modeling and machine learning, we consider the implications of our choices in terms of the development of a system.

Given the choices we have made, we are presented with multiple interesting topics to consider. If we were to fully explore each of these topics, we would have to perform extensive analysis and development within the areas of data analysis, feature learning, classification, and system development.

An approach to this problem of many and large subject areas, is to focus equally on all aspects, with the risk of developing a shallow solution. While a broad development would touch on many topics, the problem lends itself to an experimental direction, where exploration of less commonly seen approaches is in focus. We believe that exploring feature learning on Wikipedia structures is such an approach.

It can be argued that the core of the problem is classification, and that it should therefore be the focus of our project. However, it should be considered that a focus on classification, based on a generic feature learning approach, could result in an inherent inability to identify discriminating information within the features. 

However, choosing to focus on the subject of feature learning, does not imply that we will completely disregard the other areas. In order to understand the requirements for the feature learning process, we must consider the surrounding elements. Therefore, we choose to design and implement a complete system, where every required part is included, but where the feature learning elements of the system is in focus. This choice is also influenced by the consideration that further development is often easier on a system where every part has received development, since it tends to uncover unnoticed requirements and a better understanding of holistic requirements.