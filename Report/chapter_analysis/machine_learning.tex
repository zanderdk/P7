\section{Machine Learning Task}\label{sec:machine_learning_task}
%With a graph defined,
In this section we introduce a machine learning task based on the problem we are trying to solve. To solve the problem of suggesting links between Wikipedia articles. To solve this using machine learning we formulate it as a binary classification problem of determining whether two Wikipedia articles should be linked.

As described in \cref{sec:choice_of_graph} we assume featured articles to have appropriate linking. This makes links from featured articles suitable as positive training pairs when learning if a link should be present. Similarly articles that are not linked from a featured article can be used as negative training pairs.

We define the set of positive training pairs $P \subset V \times V$, as well as a set of negative training pairs $N \subset V \times V$ in \cref{eq:training_pairs}.

\begin{equation}
\label{eq:training_pairs}
  \begin{split}
    P &= \Set{(a,b) \mid a\text{ is featured}\ \wedge\ a \Rightarrow b}, \\
    N &= \Set{(a,b) \mid a\text{ is featured}\ \wedge\ a \not\Rightarrow b}
  \end{split}
\end{equation}

Under this assumption labeled training data can be generated, and so we limit the scope of the problem to only consider supervised learning approaches. This delimits the problem to binary classification problem in a supervised machine learning context.

\todo{Kasper: Udvid denne sætning. ``Vi vil gerne bruge det her labelled data som eksempler på...''}

\subsection{Defining the Learning Task}
\label{sec:ml_def}
In order to give a precise understanding of the problem we will define it formally, before explaining how we intend to solve it.

%We first formulate the classification task as an optimization problem where we maximize the objective function $\Pr(y \mid f(a,b))$, where $a,b \in V$:

% \begin{equation}
% \label{objective}
% \argmax_y\ \Pr(y \mid f(a,b))
% \end{equation}

\todo{der mangler nok en overgang her. ALEXANDER HAR DET HER AFSNIT}

\todo{Overordnet set, vil vi gerne lære en function g gående fra V x V til L. Først skal vi dog beskrive feature representation så comptueren fatter V x V. Bagefter laver vi en ``classification''-ting.}
Let $f: V\times V \to \mathbb{R}^d$ be a function mapping from article pairs to a feature representation as vector, where $d$ is the dimension of the feature vector. The set of all possible labels is given by $L=\{0,1\}$, where $1$ represents the class of article pairs that should be linked, and $0$ the ones that should not.

We want to learn a function $g: \mathbb{R}^d \to L$, that assigns a label $y_{ab} \in L$ to a feature representation $x_{ab} \in \mathbb{R}^d$, for a given article pair $(a,b) \in V \times V$, such that \cref{eq:labels} holds.
\todo{Skal måske omstruktureres. Vi vil gerne bare lære $f_{c}: V \times V \to L$ direkte, men vi har brug for feature repræsentation først.}

\begin{equation}
\label{eq:labels}
    y_{ab}=
    \begin{cases}
        1 & \text{if } a \Rightarrow b\\
        0 & \text{otherwise}
    \end{cases}
\end{equation}

Using $f$ to find the feature representation, we can assign a label $y_{ab}$ to an article pair $(a,b)$ using \cref{eq:classification_function}.

\begin{equation}
\label{eq:classification_function}
  y_{ab} = g\big(f(a,b)\big)
\end{equation}

There are two parts to solving this problem. We must train a classifier to approximate the function $g$, identifying which label $y$ should be given to a feature representation $x$. Before we can do that, we must define the function $f$ that generates the feature representation vector. The choice of classifier will be examined in \cref{choosing_classifier}, while $f$ will be described in the following section.

%We wish to find which value for $y$ that gives the highest probability for $\Pr$ given a feature representation $f(a,b)$.

%To learn predicting the label $y \in \Set{1,0}$ of a pair of articles $(a,b) \in V \times V$, we seek to find the value of $y$ that maximizes the function $\Pr(y \mid f(a,b))$, which represents the probability that articles should be linked, conditioned on the feature representation $f(x)$:

%There are two parts to solving this problem. We must train a classifier to identify which label for $y$ that gives the highest probability. But before we can do that, we must define the function $f$ that generates the feature vector. The choice of classifier will be examined in \cref{choosing_classifier}, while $f$ will be described in the following section.

%\subsubsection{Article Features}
%\todo{this needs review, and should maybe be moved to another section}
%
%The function $f: V\times V \to \mathbb{R}^d$ maps article pairs to a feature representation in $d$ dimensions, by combining the features of each article pair.
%
%$f(a,b) = g(h(a), h(b))$ where $h: V \to \mathbb{R}^d$ is a function mapping an article to a feature vector and $g: \mathbb{R}^d \times \mathbb{R}^d \to \mathbb{R}^d$ is a binary operation combining a pair of feature vectors.
%
%The feature vector $h(x)$ represents the learned features of article $x$.

%\paragraph{Input:}
%A set of $m$ training examples $(x^j,y^j)$ for $j=1,2..m$, sampled from a distribution $f(E_j)$ of article pair features, with $x^j \in R^n$ being a feature vector and $y^j \in \set{-1,+1}$ a training label. A training label $y^j=+1$ will denote a \textit{positive sample} and $y^j=-1$ a \textit{negative sample}. The i-th feature of a sample $x^j$ defined $x^j_i$ represents a pair of articles and is a combination of article features generated by node2vec~\cite{node2vec} from the training pairs $P$ and $N$.

%\todo{explain/generalize article combination without mentioning node2vec? (binary operation)}
%\todo{beskriv hvilken objective function vi vil minimere/maksimere}
%\todo{input: $x^j$ skal være par af artikler}

%node2vec:
%Given two nodes u and v, and a binary operator o' the feature vectors f(u) and f(v) are combined into the representation g(u,v).

%\paragraph{Output:}
%A function $f: R^n \to \set{-1,+1}$ that classifies additional samples ${x^k}$ sampled from $F$.
