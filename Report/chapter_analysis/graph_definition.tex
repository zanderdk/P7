\section{Data Model}\label{sec:choice_of_graph}
We organize the data as a graph, where nodes represent Wikipedia articles, since we wish to examine connections between these. The choice of edges is more complex. We motivate our choice through a consideration of what constitutes an \emph{appropriate link}. In \cref{sec:related_work} clickthrough rate and Wikipedia's own guidelines were mentioned as possible metrics. While clickthrough rates are easier to work with, we believe that they are inferior to the guidelines, due to the authoritative nature of the official Wikipedia guidelines. Therefore our choice of \emph{appropriate links} will be based on the guidelines.

Wikipedia maintains a collection of articles, that they deem outstanding examples of the highest article quality that the website can offer. There are almost 5000 of these articles, called \emph{featured articles}~\cite{wiki-featured-articles}. The featured articles follow the \emph{Wikipedia Policies and Guidelines}~\cite{wiki-editor-guidelines}, which means that they follow the guidelines for linking. Since featured articles are classified by experts, we gain some notion of expert domain knowledge by learning from these examples.

While we cannot assume that every example of appropriate linking is covered, the featured articles should provide a sizable set of good examples,
since they cover the most common cases of good linking style and exhibits the structure of well-linked articles.

We let our choice of \emph{appropriate links} be links that share characteristics with the links from featured articles. These links are considered appropriate, while all other links are of undetermined quality. 
%Determining this quality, is the problem which we will attempt to solve using machine learning. \todo{fix sætning}

Knowing this, we can now conclude that our choice of edges should preserve the link structure of the featured articles. Therefore edges in the graph will represent article hyperlinks. 
%In order to define this formally, we first define a binary relation for links between articles, shown in \cref{eq:binary_relation}. \todo{Definition af $G=(V,\Rightarrow)$ skal ind før} \todo{Equations skal måske ind som definitions?}

We formally define and unweighted directed graph $G = (V,\Rightarrow)$ where $V$ is the set of all Wikipedia articles, and $\Rightarrow$ is a binary relation of links between articles defined in \cref{eq:binary_relation}.

%$$\Rightarrow = \Set{(a,b) \mid \text{article } a \text{ has a link to article } b}$$
\begin{equation}
\label{eq:binary_relation}
\Rightarrow\ =\ \{\ (a,b)\ \in V \times V\ |\ \text{article } a \text{ links to article } b\ \}
\end{equation}

We will also use the infix notation shown in \cref{eq:infix_link}.
\begin{equation}
\label{eq:infix_link}
  \begin{split}
    a\ & \Rightarrow\ b \text{ meaning } (a,b) \in\ \Rightarrow,\\
    a\ & \not\Rightarrow\ b \text{ meaning } (a,b) \not\in\ \Rightarrow
  \end{split}
\end{equation}

In the following section we will describe how machine learning can be used to suggesting links between Wikipedia articles based on the graph $G$.

%We can now formally define our graph as an unweighted directed graph $G = (V,E)$ where $V$ is the set of all Wikipedia articles, and $E\subset V \times V = \ \, \Rightarrow$.
