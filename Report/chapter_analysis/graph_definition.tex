\section{Choice of Graph}\label{sec:choice_of_graph}
Before we explore the problem of using machine learning on graph structures, we will define the graph structure we will be working on.

For our vertices we choose to let them represent Wikipedia articles, since we wish to examine connections between them. The choice of edges is a bit more complex. We motivate our choice through a consideration of what constitutes a \emph{good link}. In the related work section the given examples of metrics are clickthrough rate and Wikipedia's own guidelines. While clickthrough rates are easier to work with, we believe that they are inferior to the guidelines, simply due to the authoritative nature of the guidelines, as they are written by Wikipedia. Therefore our choice of \emph{good links} is based on the guidelines.

However, the guidelines does not translate easily into deterministic evaluation of connections. It is perhaps possible to determine some set of semantic rules for links, based on the guidelines, but it is not a task that we wish to undertake.

Instead we choose to let our \emph{good links} be defined by example. Wikipedia maintains a collection of articles, that they deem outstanding examples of the highest article quality that the website can offer. These are called \emph{featured articles}~\cite{wiki-featured-articles}, and there are almost 5000 of these. The featured articles follow the \emph{Wikipedia Policies and Guidelines}~\cite{wiki-editor-guidelines}, which means that they follow the guidelines for linking. We therefore assume that they link appropriately. While we can not assume that every example of good linking is covered, the featured articles provide a sizable set of good examples.

Because of this we let our choice of \emph{good links} be links that share characteristics with the links in featured articles. These links are considered good, while all other links are of undetermined quality. Determining this quality, is the problem which we will attempt to solve using machine learning.

Knowing this, we can now conclude that our choice of edges should preserve the link structure of the featured articles. Therefore we choose article hyperlinks to be our edges.

\subsection{Formal Definition}
In order to allow for a more precise discussion of this graph, we will define it formally. First we define a binary relation for links between articles:

%$$\Rightarrow = \Set{(a,b) \mid \text{article } a \text{ has a link to article } b}$$
$$\Rightarrow\ =\ \{\ (a,b)\ |\ \text{article } a \text{ has a link to article } b\ \}$$

We will also use the infix notation:
$$a\ \Rightarrow\ b \text{ meaning } (a,b) \in\ \Rightarrow,$$ 
$$a\ \not\Rightarrow\ b \text{ meaning } (a,b) \not\in\ \Rightarrow$$

We can now formally define our graph as a directed graph $G = (V,E)$ where $V$ is the set of all Wikipedia articles, and $E\subset V \times V = \ \, \Rightarrow$.