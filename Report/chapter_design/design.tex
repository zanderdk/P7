\chapter{Design \& Implementation}\label{chap:design}
\todo{Design of the entire system. The nice picture we have with the system flow}

\begin{chapterorganization}
  \item In \sectionref{sec:architecture} we provide an overview of the full system architecture;
  \item In Section...
\end{chapterorganization}

\section{Overall View}
The main system consists of a pipeline utilizing two trained models in order to make predictions. These models are
\begin{enumerate*}[label=(\roman*)]
  \item a feature learning model, and
  \item a machine learning model.
\end{enumerate*}
Both models are trained by separate workflows. A diagram of the main pipeline and the separate workflows for training the models can be seen in \figureref{fig:overall-flow} with the workflows for training marked with dashed lines.

\tikzsetnextfilename{diagram}
\begin{figure}[tbp]%
  \begin{tikzpicture}[node distance = 3cm, auto]
    \tikzstyle{bigblock} = [thick, rectangle, draw, fill=clnode, text width=4.5em, text centered, minimum height=4em]
    \tikzstyle{smallblock} = [thick, rectangle, draw, fill=clnode, text width=4.5em, text centered, minimum height=3em]
    \tikzstyle{database} = [thick, cylinder, draw, shape border rotate=90, aspect=0.15, fill=clnode, text width=3em, text centered, minimum height=4em]
    \tikzstyle{line} = [draw, -stealth, thick]
    %\tikzstyle{model} = [draw, ellipse,fill=red!20, minimum height=2em]
    \tikzstyle{container} = [rectangle, draw, inner sep=0.6 cm, dashed, fill=clshade]
    
    % Place nodes
    \node [database, yshift=1em] (db) {DB};
    \node [bigblock, right of=db] (ap) {Article Picker};
    \node [bigblock, right of=ap] (tfl) {Feature Extraction};
    \node [bigblock, right of=tfl] (ai) {AI};
    \node [database, right of=ai] (db2) {DB2};
    \node [bigblock, below=1cm of db2] (web) {Web};
    
    \node [smallblock, below=1.5cm of ap, xshift=-2em] (n2v) {Node2Vec};
    \node [smallblock, below=.5cm of n2v] (paropt) {Parameter Optimizer};
    
    \node [smallblock, below=4.5cm of tfl] (prepper) {Training Data Generator};
    \node [smallblock, right of=prepper] (aitrainer) {AI Trainer};
    
    \begin{scope}[on background layer]
    \node [container, fit=(n2v)(paropt)] (container1) {};
    \node[below right] at (container1.north west) {Feature Learning};
    \node [container, fit=(prepper)(aitrainer)] (container2) {};
    \node[above right] at (container2.south west) {MI Learning};
    
    \node [container, fit=(db)(db2)] (container3) {};
    \end{scope}
    
    %\draw [->] (db) -- (n2v);
    
    \path [line] (db) -- (ap);
    \path [line] (ap) -- (tfl);
    \path [line] (tfl) -- (ai);
    \path [line] (ai) -- (db2);
    \path [line] (db2) -- (web);
    
    \path [line] (db.300) |- (n2v);
    %\path [line] (db.300) |- (paropt);
    \path [line] (db.240) |- (prepper);
    
    \path [line, swap] (paropt) -- node{Parameters} (n2v);
    \path [line] (n2v.east) -| node [near end] {Model} ([xshift=-1.2em]tfl.south);
    
    \path [line] (prepper) -- (aitrainer);
    %\path [line] (prepper) -- (tfl);
    %\path [line] ([xshift=1.2em]tfl.south) -- ([xshift=1.2em]prepper.north);
    \draw [line] (prepper.north) |- ([xshift=1.2em, yshift=0.5em]tfl.south) -- ([xshift=1.2em]prepper.north);
    
    \path [line] (aitrainer) -- node{Model} (ai);
    
  \end{tikzpicture}
\caption[short desc]{Text}%
\label{fig:overall-flow}%
\end{figure}

%Throughout this chapter the different components will be explained in detail.
In the following sections, these areas will be described.

\subsection{Main Pipeline}
The main part of the system architecture consists of a pipeline concerning the components: database, article picker, feature extractor, AI, and Web.

The main pipeline is triggered when a user uses a web interface to query the system for articles with missing links.
The article picker component is given a source article and starts querying the database component for a list of candidate target articles that should be tested for missing links. The articles are arranged into pairs and forwarded to the feature extraction component. Previously learned features \todo{what is this?} for the article pairs are extracted as vectors which are fed into the AI component. The AI uses the feature vectors to predict whether or not a link should be present from the source article to the targets. Finally, this classification is forwarded to the web component which presents relevant information to the user.

\subsection{Training Workflows}
The main pipeline needs two models. These are trained in separate workflows:

\begin{description}
  \item[Feature Learning]
  
  \item[MI]
\end{description}

Meanwhile in the \textit{N2V Gym\texttrademark}. Feature learning is done by using node2vec for performing random walks between articles within the link graph gathered from the database. The walks are used to train the node2vec model which ultimately is used by the feature extractor.
A parameter optimization component is used to find node2vec parameters that increases accuracy for the Wikipedia link data.

The machine learning pipeline aka \textit{AI Gym\texttrademark} uses a list of labeled article pairs to train the AI model. The articles are gathered from the database and features are extracted using the feature extractor. The features and classification labels are passed to the AI Trainer component which trains the model that will be used by the AI component.


\section{Overview of Architecture}\label{sec:architecture}
\todo{Nice picture of entire architecture}

\section{Features}

\section{Machine Intelligence}

\section{Graphical User Interface}
??? Profit!