\chapter{Design \& Implementation}\label{chap:design}
In this chapter we will discuss the development of the solution that was outlined in the previous chapter. The system will be explained in a top-down fashion, and as such we start off with an overview of the entire system through the architecture. Afterward (in \cref{sec:db} and onwards in this chapter), the specifics of each component will be explained in their own separate sections.

\section{Architecture}\label{sec:design_overview}
This system is built on an pipeline architecture, where the output of each primary component is fed into the next. The main benefit of this architecture is the modularity and scalability. Each component has minimal coupling with each other, by only interacting through data. As long as two components both adhere to the rules for data exchange, they can completely disregard any logic or functionality of each other. This gives the system the two advantageous properties of effortless exchange of entire components and straightforward horizontal scaling.

If system usage grows sufficiently large the data exchange could be handled by intermediate databases using, for example, a master-slave architecture where components supplying data could write to the database through one of multiple masters, and components retrieving data could read from one of multiple slaves. It is also possible to have every component run on different machines in different places. While this might not be a computational advantage, it minimizes logistics problems during scaling operations, since the system can be partially moved, component by component. Shutdown of a component can be delayed until its replacement is running. Regardless of the specific implementations, our usage of the pipeline architecture makes the system inherently modular and scalable.

\todo{Nattiya: Scalability needs to be explained more in depth}

\subsection{Limiting Input Size}
The potential input size of testing all possible combinations of articles for missing links is roughly 5 million articles squared, which totals to $2.5 \times 10^{13}$. Suppose we can process 1 million of these pairs per second, it will still take almost 290 days. An input of this size will therefore be impossible to handle during this project period. However, we may be able to heuristically limit the input size, if we assume many article pairs can be discarded immediately --- but we need to generate these pairs dynamically instead of filtering. By filtering we encounter the same problem as before, since we have the same input size of $\num{5000000}^2$. By dynamically generating the pairs, we can generate a pair of articles as needed. This way, the computationally intensive evaluation is only performed on the data that is generated as needed.

\tikzsetnextfilename{system-overview}
\begin{figure}[tb]%
  \centering
  \begin{tikzpicture}[node distance = 2.84cm, auto]
    \node [database, yshift=1em] (db) {Wiki DB};
    \node [bigblock, right of=db] (ap) {Article Picker};
    \node [bigblock, right of=ap] (tfl) {Feature Extractor};
    \node [bigblock, right of=tfl] (classifier) {Classifier};
    \node [database, right of=classifier] (db2) {Results Pool};
    \node [bigblock, below=1cm of db2] (web) {Web Interface};
    
    %\node [above=1cm of ap] (inputarticle) {};
    
    \node [smallblock, below=1.5cm of ap, xshift=-2em] (n2v) {node2vec};
    \node [smallblock, below=.5cm of n2v] (paropt) {Parameter Optimizer};
    
    \node [smallblock, below=4.5cm of tfl] (prepper) {Training Data Generator};
    \node [smallblock, right of=prepper] (classifierTrainer) {Classifier Trainer};
    
    \begin{scope}[on background layer]
    \node [container, fit=(n2v)(paropt)] (container1) {};
    \node[below] at (container1.north) {Feature Learning};
    \node [container, fit=(prepper)(classifierTrainer)] (container2) {};
    \node[above] at (container2.south) {Classifier Model Learning};
    
    \node [container, fit=(db)(db2)] (container3) {};
    \node[below] at (container3.north) {Main Pipeline};
    \end{scope}
    
    %\draw [->] (db) -- (n2v);
    
    \path [line] (db) -- (ap);
    %\path [line] (inputarticle) -- (ap);
    \path [line] (ap) -- (tfl);
    \path [line] (tfl) -- (classifier);
    \path [line] (classifier) -- (db2);
    \path [line] (db2) -- (web);
    
    \path [line] (db.300) |- (n2v);
    %\path [line] (db.300) |- (paropt);
    \path [line] (db.240) |- (prepper);
    
    \path [line, swap, transform canvas={xshift=-1.3em}] (paropt) -- node{Parameters} (n2v);
    \path [line] (n2v.east) -| node [near end] {Model} ([xshift=-1.2em]tfl.south);
    
    \path [line] (prepper) -- (classifierTrainer);
    %\path [line] (prepper) -- (tfl);
    %\path [line] ([xshift=1.2em]tfl.south) -- ([xshift=1.2em]prepper.north);
    \draw [line] (prepper.north) |- ([xshift=1.2em, yshift=0.5em]tfl.south) -- ([xshift=1.2em]prepper.north);
    
    \path [line] (classifierTrainer) -- node{Model} (classifier);
    
  \end{tikzpicture}
\caption[Architecture diagram showing the major components of the system]{Architecture diagram showing the major components of the system. Each box is a component, and a dashed line box is a grouping of components. Arrows describe the way data flows.}%
\label{fig:system-overview}%
\end{figure}

\subsection{Main Pipeline}
The main pipeline consists of five components. The first one is a storage component, which holds all the information required to identify the link suggestions. The second component is \emph{Candidate Filtering}, which is responsible for the filtering process. As such, it extracts candidate article pairs from the database. The candidates are of the form $(A,B)$ where $A$ is a source article and $B$ is a suggested target article. The third component is the \emph{Feature Extractor}. This intermediate component generates a feature vector from a given candidate pair. As the fourth component we have the \emph{Classifier}, that is responsible for the refinement process. It does this by suggesting link that should be added to the input pair, based on the feature vector it received. Finally the \emph{Results Pool} is another storage component, which holds the suggestions until they are presented to a user through the user interface.

\subsection{Model Training}
Construction of the main pipeline includes learning a model for extracting features from article, as well as training a classifier. These tasks are performed by two support modules.

The support module for the feature learning consists of a \emph{Model Trainer} and a \emph{Parameter Optimizer}. The \emph{Parameter Optimizer} works on a smaller subset of the dataset in order to speed up the process of tuning the parameters. When an acceptable set of parameters has been found, the \emph{Model Trainer} uses these parameters to train a model on the entire dataset. The result of this is the model, which is used as the \emph{Feature Extractor}.

The other support module is for the classifier learning process. It consists of a \emph{Training Data Generator} and a \emph{Classifier Trainer}. The \emph{Training Data Generator} retrieves training pairs and labels from the dataset, and uses the \emph{Feature Extractor} to prepare the data. Then the \emph{Classifier Trainer} uses this prepared training data to create a model, which is used as our classifier.

\subsection{User Interface}

Finally, we cover the User Interface. The purpose of this component is to publish the results of the main pipeline to users who can then use the results to edit Wikipedia. Since the classifier can not be expected to reach a perfect precision, the final decision of whether to include a suggested link, should be done by a human editor. Therefore we create a user interface that allows people to retrieve the results of the main pipeline.

We consider this component separate from the main pipeline. The reason is different requirements of the two parts. While both the user interface and the main pipeline share the goals of then entire system, they have different subgoals. The main pipeline is purely focused on producing results, and the user interface only considers the delivery of those results. This difference in subgoals requires a different evaluation approach, and as such it is valuable to separate the user interface from the main pipeline.

\section{Database}\label{sec:db}
The first component in our main pipeline is the database, as shown in \cref{fig:system-overview}. Our approach to identify missing links requires readily available data, which we facilitate by storing required data in a local database.
\todo{node2vertex det her afsnit? neo4j bruger selv node.}

\subsection{Database Design}\label{sec:db_design}
We use the native graph database system Neo4j~\cite{neo4j} as it performs well for graph based queries, and supports extensions to the query language with Java plugins. The data model in Neo4j consists of nodes, relationships between nodes, and properties on these. Each node and relationship can be annotated with a number of labels used to distinguish between different node and relationship types when querying the database.

We store Wikipedia pages as nodes with their title as a property, and links between pages as relationships. Labels are used to distinguish between different types of pages and links, as shown in \cref{tab:db_labels_nodes,tab:db_labels_edges}.

\begin{table}[tbp]
  \centering
  
    \begin{tabular}{@{}p{.20\textwidth}p{.60\textwidth}@{}}
      \toprule
      \textbf{Label}         & \textbf{Description}                            \\ \midrule
      \mono{Page}                   & A Wikipedia article                             \\
      \mono{FeaturedPage}           & A Wikipedia article marked as \emph{Featured}   \\
      \mono{GoodPage}               & A Wikipedia article marked as \emph{Good}       \\
      \mono{RedirectPage}           & A redirecting Wikipedia page                    \\
      \bottomrule
    \end{tabular}
    \caption[Node labels in the database]{Node labels in the database}%
    \label{tab:db_labels_nodes}
\end{table}
\begin{table}[tbp]
    \centering
    \begin{tabular}{@{}p{.20\textwidth}p{.60\textwidth}@{}}
      \toprule
      \textbf{Label}         & \textbf{Description}                            \\ \midrule
      \mono{LINKS\_TO}              & A link between two articles                     \\
      \mono{TRAINING\_DATA}         & A link that can be used during training         \\
      \mono{TEST\_DATA}             & A link that should only be using during testing \\
      \mono{REDIRECTS\_TO}          & An edge describing a redirect                   \\ \bottomrule
    \end{tabular}
    \caption[Edge labels in the database]{Edge labels in the database}%
    \label{tab:db_labels_edges}
\end{table}

\subsection{Populating the Database}\label{sec:db_populate}
We use two datasets provided by DBpedia~\cite{dbpedia} to populate the database; one contains links and the other redirects. Both datasets contain tuples in the following format:

%page_links_unredirected_en.ttl.bz2
%redirects.en.ttl.bz2

\begin{center}
\mono{\emph{<source\_page> <type> <target\_page>}}
\end{center}

We prune non-article pages based on the namespace prefixes used on Wikipedia. This removes meta pages such as \enquote{Talk:} and \enquote{User:} pages. We also remove redirects from the link dataset as it both contains links and redirects. Links are split into three groups randomly, according to a partitioning given in \cref{sec:training_data}. \todo{lige nu kommer forklaring på splittet senere, skal sektioner flyttes?}

From the pruned data we generate separate files containing pages, redirect pages, partitioned links, and redirect links. The files are imported with the appropriate labels, and additional labels are added to \emph{featured} and \emph{good} articles.
\todo{we removed some redirect nodes as well.} The counts for the different labels in the populated database is shown in \cref{tab:db_counts_nodes,tab:db_counts_edges}.

\begin{table}[tbp]
  \centering
  \begin{minipage}[t]{0.45\textwidth}
    \centering
    \begin{tabular}{@{}lr@{}}
      \toprule
      \textbf{Label}         & \textbf{Count}     \\
      \midrule
      \textit{Nodes (total)} & \textit{\num{18172630}}  \\
      \mono{Page}                   & \num{11159213}           \\
      \mono{FeaturedPage}           & \num{4820}               \\
      \mono{GoodPage}               & \num{23741}              \\
      \mono{RedirectPage}           & \num{7013417}            \\ \bottomrule
    \end{tabular}
    \caption[Counts for node labels]{Counts for node labels. Note that some nodes have multiple labels.}%
    \label{tab:db_counts_nodes}
  \end{minipage}
  \hfill
  \begin{minipage}[t]{0.45\textwidth}
    \centering
    \begin{tabular}{@{}lr@{}}
      \toprule
      \textbf{Label}         & \textbf{Count}     \\ \midrule
      \textit{Relationships (total)} & \textit{\num{145878042}} \\
      \mono{LINKS\_TO}              & \num{138422339}          \\
      \mono{TRAINING\_DATA}         & \num{294857}             \\
      \mono{TEST\_DATA}             & \num{147429}             \\
      \mono{REDIRECTS\_TO}          & \num{7013417}            \\ \bottomrule
    \end{tabular}
    \caption[Counts for edge labels]{Counts for edge labels.}%
    \label{tab:db_counts_edges}
  \end{minipage}

\end{table}


%\section{Detailed View of Components}\label{sec:detailed_view}
%\todo{Maybe each of the following subsections should be their own section. As it is now, we go pretty deep into subsubsection! This probably depends on how brief/long each component will be.}

%Here, we will describe each component individually. \todo{Refer to \cref{fig:system-overview}}

\section{Candidate Generator}
\todo{Similar to featured articles, Wikipedia also has \emph{good articles} that follow most of the guidelines, providing a satisfactory overall quality~\cite{wiki-good-articles}. However, there is no requirement for them to follow the guidelines for proper linking.}

The candidate generator component is used in production after the classification models have been trained. The reason for doing this generation step, is that the number of possible article pair combinations is roughly 5 million articles squared, which totals to $2.5 \times 10^{13}$. Suppose we can process 1 million of these pairs per second, it will still take almost 290 days all of them. An input of this size will therefore be impossible to handle during this project period. However, the majority of these article pairs will not need a link, and we may thus be able to heuristically limit the input size, if we assume many article pairs can be discarded immediately --- but we need to generate these pairs dynamically instead of filtering. By filtering we encounter the same problem as before, since we have the same input size of $\num{5000000}^2$. By dynamically generating the pairs, we can generate a pair of articles as needed. This way, the computationally intensive evaluation is only performed on the data that is generated as needed.

The article pairs are selected based on a heuristic policy. The policy can be a combination of multiple approaches, as long as they provide ordered candidate pairs. A perfect policy provides pairs, where all the articles with missing links occur before any article without missing links. While such a policy is helpful, it is unrealistic and only serves as an example. As the policy is a heuristic, it only needs to perform reasonably well, which is why we do not employ any systematic evaluation of possible policies, but our choice is based on our own intuition. As long as the classifier suggests links at a sufficient rate, this way of choosing a policy will be acceptable.

Our choice of policy is a combination of two approaches. One is based on a clickstream dataset and the other on a simple syntactic analysis.

\subsection{Clickstream Approach}

The clickstream approach is inspired by related work that uses server logs to predict missing links~\cite{hyperlink-structure-using-logs}. We use clickstream~\cite{wiki-clickstream} data built from Wikipedia server logs, and as such it is based on user behavior. A clickstream data source is a list of requests that Wikipedia received within a given timespan\footnote{We use a data source collected from March 1 -- March 31, 2016}. A request holds information on a referrer and a resource, as well as a count of the occurrence of this request and a request type. The request type can be \emph{link}, \emph{external}, or \emph{other}. For this approach, the type \emph{other} is the only relevant one. A request of type \emph{other} means that the referrer and resource were both articles, but that referrer does not link to the resource. One source of this type of requests could be searches~\cite{wiki-clickstream}.

The heuristic for this approach, is that some of these requests will indicate that a user, after reading one article, was prompted to read another article, and that they did not have the possibility of following a link. While there are plenty of cases where this does not constitute a missing link, chances are that some of the requests are examples of this. The approach works by going through the clickstream data, picking out requests of type \emph{other}, and ranking them by their number of occurrences.

Though preliminary results with this approach have been promising, it only produces a limited amount of candidate pairs (1.7 million). Therefore, we choose to complement it with another approach to get a bigger set of candidate pairs.

\subsection{n-gram Approach}

For the complementary approach we took inspiration from~\cite{milne2008learning}, where they employ a syntactic analysis to find article pairs. The inspiration lies in their use of n-grams to find article pairs. For each article we create a set of shingles from an n-gram search. We then search through the titles of Wikipedia articles, for ones that are contained in at least one of the shingles. This gives us a large set of article pairs $(A,B)$, where article $B$ mentions article $A$.

However, this is a crude approach and a significant amount of the candidate pairs can easily be discarded. Consider the Wikipedia article on the word \enquote{the}. With the n-gram approach, nearly all 5.2 million articles would be in a candidate pair with this article, where only a fraction would be worth considering.

A way to combat this would be to order the candidate pairs according to the inverse frequency of target articles. The intuition is that articles, which are rarely mentioned across all articles, will be more significant when they \emph{are} mentioned. It is likely that a frequency threshold will be required, since the advantages of the approach becomes negligible at some sufficiently high frequency.

Another considerable problem with the approach is that while it does provide many candidate pairs, it can not be guaranteed to order all possible pairs, and therefore it will not completely cover the complement of the clickstream approach. A preliminary test with a 5-gram search through featured articles gave us more than a million candidate pairs. When all articles are searched, this approach will generate significantly more than the clickstream approach. This will be sufficient for our usage, and further scaling will not be implemented in this project.


%from an article $a$ to an article $b$ in the time frame where the data was collected, as well as the method of navigation of these clicks. One of these methods of navigation is a \emph{teleportation}, which means the user ended up on article $b$ after having viewed article $a$, but not following a link. \todo{Give good explanations of how the clicks then might have happened}

%By selecting article pairs without an existing link and that users have navigated between by teleportation, the articles might be related and inserting a link between the articles might improve navigability.

%\todo{Show code and implementation}

%\todo{write if we end up doing this}
%Another policy to consider is selecting article pairs based on text content. Specifically using n-grams of words (shingles) to find text in the article that references titles of other articles without already linking to it. 

%\todo{teleport/clickstream like sources}
% https://cs.stanford.edu/people/jure/pubs/wiki-www15.pdf
% others

\section{Feature Extractor}\label{feature_extractor}
The next component in the main pipeline is the feature extractor. This component generates the feature representation of a candidate link, which can then be used for classification. \todo{ref til func i kapitel 3} The feature extractor returns a single feature vector, representing the link between a given source and target article pair. As we need to classify both existing and non-existing links, it must be possible to extract feature vectors in both cases. Therefore the feature vector for a link is constructed by combining the feature vectors of the source and target articles.

As mentioned in \cref{sec:feature_generation} we aim to use feature learning, specifically network embedding, to learn these feature representations. While there are several different network embedding models, we found node2vec~\cite{node2vec} to be a suitable model. \cite{node2vec} compares node2vec to other well-known models, and finds that node2vec delivers better or equal precision for link prediction in all cases. Additionally, node2vec performs well for large networks, and offers flexibility due to tunable hyperparameters, which provides a more fine-grained control of neighborhood exploration, allowing us to test different options. This section describes how node2vec is used to generate the feature representation, along with the process of optimizing the models hyperparameters. \todo{update argumentation for node2vec choice. Flexible parameters is nice with limited knowledge of dataset + competitive performance}

\subsection{Description of node2vec} % (fold)
\label{sec:node2vec}
\todo{Should include much more theory throughout section.}
\todo{Explain: We need to find node feature vectors that we can combine to edge feature vectors.}
\todo{Explain: Similar to word embedding, where we see nodes as words, and node sequences (walks) as sentences}
Node2vec is an algorithmic framework for semi-supervised feature learning in networks~\cite{node2vec}. The goal of the algorithm is to learn a model that can reconstruct the neighborhood of a given node.


\subsubsection{Description of word2vec}
\todo{describe word2vec, including neural network etc.}
Using nodes as words and the random walks as sentences allows using word embedding algorithms on the sentences, mapping words to vectors of real numbers. The node2vec reference implementation uses word2vec for word embedding. Intuitively, if two words are close two each other in the sentence, they are related. The tunable parameter $k$ specifies the size of a context window that defines the number of surrounding words that should be considered in the same context. In node2vec the context window is used for sampling neighborhoods of nodes, where it determines the size of the neighborhood.

\subsubsection{Finding Neighborhoods}
\todo{Describe the biased random walk using a figure}
The neighborhoods of each node is constructed by performing a number of biased random walks. The walks are biased using two parameters $p$ and $q$. A high $p$ value, relative to $max(1,q)$, approximates a depth-first search, while a high $q$ value, relative to $max(1,p)$, approximates the behavior of a breath-first search. The motivation of these tunable parameters is the observation that real-world networks can be structured in many ways. The randomness and the tunable $p$ and $q$ values accounts for different types of networks, by allowing control of the exploration method.

There are three additional parameters in node2vec, $d$, $r$, and $l$, which determines the dimensions of the resulting feature vector, the number of random walks performed per node, and the maximum length of each walk, respectively.

\subsubsection{Combining Feature Vectors}
\todo{Explain how we combine feature vectors. We needed non-commutative operation}
By training a node2vec model on the graph, we are able to learn feature representations for each node. However, we need to combine these features in order to construct a feature representation of a potential edge between the two nodes. We combine these two node feature vectors by using the Hadamard product, as this was found to give the best results in~\cite{node2vec}.

% \section{Overview of node2vec} % (fold)
% \label{sec:overview_of_node2vec}

% Node2vec is an algorithmic framework for semi-supervised feature learning in networks~\cite{node2vec}. The goal of the algorithm is to learn a model that can reconstruct the neighborhood of a given node.

% The neighborhoods of each node is constructed by performing a number of biased random walks. The walks are biased using two parameters $p$ and $q$. A high $p$ value, relative to $max(1,q)$, approximates a depth-first search, while a high $q$ value, relative to $max(1,p)$, approximates the behavior of a breath-first search. The motivation of these tunable parameters is the observation that real-world networks can be structured in many ways. The randomness and the tunable $p$ and $q$ values accounts for different types of networks, by allowing control of the exploration method.

% Using nodes as words and the random walks as sentences allows using word embedding algorithms on the sentences, mapping words to vectors of real numbers. The node2vec reference implementation uses word2vec for word embedding. Intuitively, if two words are close two each other in the sentence, they are related. The tunable parameter $k$ specifies the size of a context window that defines the number of surrounding words that should be considered in the same context. In node2vec the context window is used for sampling neighborhoods of nodes, where it determines the size of the neighborhood.

% There are three additional parameters in node2vec, $d$, $r$, and $l$, which determines the dimensions of the resulting feature vector, the number of random walks performed per node, and the maximum length of each walk, respectively.

% By training a node2vec model on the graph, we are able to learn feature representations for each node. However, we need to combine these features in order to construct a feature representation of a potential edge between the two nodes. We combine these two node feature vectors by using the Hadamard product, as this was found to give the best results in~\cite{node2vec}.
% section overview_of_node2vec (end)

\tikzsetnextfilename{n2vbowtie}
\begin{figure}[tbp]%
  \centering
  \input{chapter_design/n2v_bowtie_fig}

\caption[short desc]{Text. Adapted from \todo{insert source}}%
\label{fig:n2v-figure}%
\end{figure}

\subsection{Optimizing Performance / Scalability}
\todo{Find better title}
\todo{Performance concerns: Parallel walks, memory concerns (write walks to file), gensim is parallel}

\subsection{Hyperparameter Optimization}
\todo{tilpas til nye værdier + ny metode med forskellige kombinations funktioner}
As described above, node2vec has many tunable parameters. To find the parameters that give the best set of features, we do a hyperparameter optimization pass. We first specify the parameter space that should be searched in. As this is a large space, it is not feasible to exhaust all possibilities. We therefore use the tool Spearmint~\cite{snoek2012practical} which performs Bayesian optimization by maximizing the expected improvement to efficiently search for parameters that will minimize an objective function. The objective function is $1 - \text{f-score}$, where the f-score is found by running a classifier on the node2vec model learned from the parameters. By keeping the classifier and its parameters constant, we can find the best set of parameters by finding the lowest objective function value.

To speed each iteration of the optimization up, we limit the set of start nodes to featured articles. The hyperparameter optimization is entirely autonomous and the process should be able to find good $p$ and $q$ values on its own. This means that one can learn on graph networks without knowing the underlying graph structure.

We start with a coarse search to find a good range for each parameter. This is done with large grained values for each parameter that would show local optima. After this, we search a second time with finer grained values to further refine the local optima. After completing 83 experiments, the objective function seemed to have converged by minimizing it with the parameters seen in~\cref{tab:paramopt_goodvalues}.

\begin{figure}%
\centering
\begin{tabular}{cccccc}
\toprule
$p$  & $q$     & $d$ & $r$ & $k$ & $l$ \\
0.50 & 100,000 & 256 & 1   & 80  & 80 \\
\bottomrule
\end{tabular}
\caption[The found parameter values producing a minimization]{The found parameter values producing a minimization \todo{should we write k=120?}}%
\label{tab:paramopt_goodvalues}%
\end{figure}


Since the constant $\alpha = 1$ is higher that both $1/q$ and $1/p$ it is most likely that the walk will progress close within the neighborhood of a node. Additionally since $q$ is significantly higher than $p$ there is a higher probability that a walk will return to the previous node rather than exploring outwards, which further increases the likeliness of staying in the neighborhood.

$r=1$ means that only a single random walk is performed from each node. This might be sufficient because the other parameters already increases traversal of the neighboring nodes.

The size of the context window $k$ did not have much impact on the objective function as long as it was around a value of 80. The maximum length of the walk $l=80$ \todo{comment on these values}.



\section{Feature Extractor}\label{feature_extractor}
The next component in the main pipeline is the feature extractor. This component generates the feature representation of a candidate link, which can then be used for classification. \todo{ref til func i kapitel 3} The feature extractor returns a single feature vector, representing the link between a given source and target article pair. As we need to classify both existing and non-existing links, it must be possible to extract feature vectors in both cases. Therefore the feature vector for a link is constructed by combining the feature vectors of the source and target articles.

As mentioned in \cref{sec:feature_generation} we aim to use feature learning, specifically network embedding, to learn these feature representations. While there are several different network embedding models, we found node2vec~\cite{node2vec} to be a suitable model. \cite{node2vec} compares node2vec to other well-known models, and finds that node2vec delivers better or equal precision for link prediction in all cases. Additionally, node2vec performs well for large networks, and offers flexibility due to tunable hyperparameters, which provides a more fine-grained control of neighborhood exploration, allowing us to test different options. This section describes how node2vec is used to generate the feature representation, along with the process of optimizing the models hyperparameters. \todo{update argumentation for node2vec choice. Flexible parameters is nice with limited knowledge of dataset + competitive performance}

\subsection{Description of node2vec} % (fold)
\label{sec:node2vec}
\todo{Should include much more theory throughout section.}
\todo{Explain: We need to find node feature vectors that we can combine to edge feature vectors.}
\todo{Explain: Similar to word embedding, where we see nodes as words, and node sequences (walks) as sentences}
Node2vec is an algorithmic framework for semi-supervised feature learning in networks~\cite{node2vec}. The goal of the algorithm is to learn a model that can reconstruct the neighborhood of a given node.


\subsubsection{Description of word2vec}
\todo{describe word2vec, including neural network etc.}
Using nodes as words and the random walks as sentences allows using word embedding algorithms on the sentences, mapping words to vectors of real numbers. The node2vec reference implementation uses word2vec for word embedding. Intuitively, if two words are close two each other in the sentence, they are related. The tunable parameter $k$ specifies the size of a context window that defines the number of surrounding words that should be considered in the same context. In node2vec the context window is used for sampling neighborhoods of nodes, where it determines the size of the neighborhood.

\subsubsection{Finding Neighborhoods}
\todo{Describe the biased random walk using a figure}
The neighborhoods of each node is constructed by performing a number of biased random walks. The walks are biased using two parameters $p$ and $q$. A high $p$ value, relative to $max(1,q)$, approximates a depth-first search, while a high $q$ value, relative to $max(1,p)$, approximates the behavior of a breath-first search. The motivation of these tunable parameters is the observation that real-world networks can be structured in many ways. The randomness and the tunable $p$ and $q$ values accounts for different types of networks, by allowing control of the exploration method.

There are three additional parameters in node2vec, $d$, $r$, and $l$, which determines the dimensions of the resulting feature vector, the number of random walks performed per node, and the maximum length of each walk, respectively.

\subsubsection{Combining Feature Vectors}
\todo{Explain how we combine feature vectors. We needed non-commutative operation}
By training a node2vec model on the graph, we are able to learn feature representations for each node. However, we need to combine these features in order to construct a feature representation of a potential edge between the two nodes. We combine these two node feature vectors by using the Hadamard product, as this was found to give the best results in~\cite{node2vec}.

% \section{Overview of node2vec} % (fold)
% \label{sec:overview_of_node2vec}

% Node2vec is an algorithmic framework for semi-supervised feature learning in networks~\cite{node2vec}. The goal of the algorithm is to learn a model that can reconstruct the neighborhood of a given node.

% The neighborhoods of each node is constructed by performing a number of biased random walks. The walks are biased using two parameters $p$ and $q$. A high $p$ value, relative to $max(1,q)$, approximates a depth-first search, while a high $q$ value, relative to $max(1,p)$, approximates the behavior of a breath-first search. The motivation of these tunable parameters is the observation that real-world networks can be structured in many ways. The randomness and the tunable $p$ and $q$ values accounts for different types of networks, by allowing control of the exploration method.

% Using nodes as words and the random walks as sentences allows using word embedding algorithms on the sentences, mapping words to vectors of real numbers. The node2vec reference implementation uses word2vec for word embedding. Intuitively, if two words are close two each other in the sentence, they are related. The tunable parameter $k$ specifies the size of a context window that defines the number of surrounding words that should be considered in the same context. In node2vec the context window is used for sampling neighborhoods of nodes, where it determines the size of the neighborhood.

% There are three additional parameters in node2vec, $d$, $r$, and $l$, which determines the dimensions of the resulting feature vector, the number of random walks performed per node, and the maximum length of each walk, respectively.

% By training a node2vec model on the graph, we are able to learn feature representations for each node. However, we need to combine these features in order to construct a feature representation of a potential edge between the two nodes. We combine these two node feature vectors by using the Hadamard product, as this was found to give the best results in~\cite{node2vec}.
% section overview_of_node2vec (end)

\tikzsetnextfilename{n2vbowtie}
\begin{figure}[tbp]%
  \centering
  \input{chapter_design/n2v_bowtie_fig}

\caption[short desc]{Text. Adapted from \todo{insert source}}%
\label{fig:n2v-figure}%
\end{figure}

\subsection{Optimizing Performance / Scalability}
\todo{Find better title}
\todo{Performance concerns: Parallel walks, memory concerns (write walks to file), gensim is parallel}

\subsection{Hyperparameter Optimization}
\todo{tilpas til nye værdier + ny metode med forskellige kombinations funktioner}
As described above, node2vec has many tunable parameters. To find the parameters that give the best set of features, we do a hyperparameter optimization pass. We first specify the parameter space that should be searched in. As this is a large space, it is not feasible to exhaust all possibilities. We therefore use the tool Spearmint~\cite{snoek2012practical} which performs Bayesian optimization by maximizing the expected improvement to efficiently search for parameters that will minimize an objective function. The objective function is $1 - \text{f-score}$, where the f-score is found by running a classifier on the node2vec model learned from the parameters. By keeping the classifier and its parameters constant, we can find the best set of parameters by finding the lowest objective function value.

To speed each iteration of the optimization up, we limit the set of start nodes to featured articles. The hyperparameter optimization is entirely autonomous and the process should be able to find good $p$ and $q$ values on its own. This means that one can learn on graph networks without knowing the underlying graph structure.

We start with a coarse search to find a good range for each parameter. This is done with large grained values for each parameter that would show local optima. After this, we search a second time with finer grained values to further refine the local optima. After completing 83 experiments, the objective function seemed to have converged by minimizing it with the parameters seen in~\cref{tab:paramopt_goodvalues}.

\begin{figure}%
\centering
\begin{tabular}{cccccc}
\toprule
$p$  & $q$     & $d$ & $r$ & $k$ & $l$ \\
0.50 & 100,000 & 256 & 1   & 80  & 80 \\
\bottomrule
\end{tabular}
\caption[The found parameter values producing a minimization]{The found parameter values producing a minimization \todo{should we write k=120?}}%
\label{tab:paramopt_goodvalues}%
\end{figure}


Since the constant $\alpha = 1$ is higher that both $1/q$ and $1/p$ it is most likely that the walk will progress close within the neighborhood of a node. Additionally since $q$ is significantly higher than $p$ there is a higher probability that a walk will return to the previous node rather than exploring outwards, which further increases the likeliness of staying in the neighborhood.

$r=1$ means that only a single random walk is performed from each node. This might be sufficient because the other parameters already increases traversal of the neighboring nodes.

The size of the context window $k$ did not have much impact on the objective function as long as it was around a value of 80. The maximum length of the walk $l=80$ \todo{comment on these values}.



\section{Classifier}\label{sec:classifier}
As described in \cref{sec:machine_learning_task} we want to solve a supervised binary classification problem. In order to solve this problem, we want to construct a classifier. The previous section describes our feature extraction method. We will in this section describe how we use this method to extract features from training examples, and based on these choose a classification algorithm. We further describe how we optimize the classifier's hyperparameters for increased performance.

\subsection{Training Data}\label{sec:training_data}
Like all supervised learning, the classifier model is learned through training data.

As described in \ref{sec:choice_of_graph} we will be using featured articles and what they link to, to train the classifier. This data is split into 20\% test data and 80\% training data. This specific split is made to preserve the link structure in the graph while still having a reasonable amount of samples to test on. To train the classifier to find links that do not already exist, a random distribution of 50\% of the edges is removed from the training data. This ensures that we do not overfit the classifier on already existing links. \todo{Det er muligvis lidt dårligt forklaret mht. de 50\%}
%One sample of training data has the form \mono{<source article> <target article> <label>}. The source and targeet article are titles of pairs of articles. The label on the training samples can either be positive (link) or negative (do not link).

As described in \cref{sec:machine_learning_task} we use a set of positive labeled training pairs $P$ and a set of negative labeled training pairs $N$. The positive training samples are pairs of linked articles as defined by $P$. As $\left\vert{P}\right\vert$ is smaller than $\left\vert{N}\right\vert$, we randomly sample from $N$ until we have as many negative as positive training pairs.

%The positive samples are pairs of articles $(a,b)$ such that article $a$ is featured and there exists a link from article $a$ to $b$.



%We first experimented with negative samples being random article pairs $(a,b)$ fulfilling the condition of $N$: that article $a$ is featured and there does not exist a link from article $a$ to $b$. The problem with this approach is that it is too easy to overfit using these negative samples. As the article pairs are randomly sampled, most of the times, there is very little relatedness between the two articles. This is of course the point of a negative training sample, but because most of the negative samples were so unrelated, the classifiers used in our experiments could too easily differentiate positives from negatives. We needed another method that would give us negative samples where the articles have a higher likelihood of being related. Our second approach was to extract N-grams from all featured articles. Iterating through all N-grams for a featured article, a negative sample would be generated if another article had the exact same title as the N-gram, but the two article were not linked.

\subsection{Choosing a Classifier}\label{choosing_classifier}
To get a coarse estimate of which classifier seems the most appropriate for our purpose, we create a test harness that will run a range of classification algorithms on the same dataset. By comparing the algorithms against each other, we should be able to weed out the worst performing algorithms. Each algorithm will be run with its default parameters which is suboptimal, but finding good parameters for all algorithms is unrealistic given the time frame of this project. To still allow for a fair comparison, we choose the top performing algorithms, and derive 3 parameter sets for each algorithm. Based on this, we choose the algorithm that looks the most promising to optimize further.

The examined algorithms are chosen mostly based on classifiers available in the scikit-learn python library. There are many algorithms to choose from, and as all the algorithms are in the same library, the interface is the same, significantly simplifying the test harness.

\subsubsection{Evaluation Metrics}
To evaluate the performance of each algorithm, we perform a 3-fold cross-validation. As the core of our problem definition is to find suggestions to missing links, false negatives should be punished harder than false positives. In other words, suggesting a pair of articles to be linked that should not actually be linked, is worse than not suggesting a pair of articles that should be linked. 

Because of the before mentioned higher penalty rate of false negatives we want to favor precision over recall, as precision decreases as the number of false negatives increases. We therefore choose precision as our evaluation metric.

\begin{equation}\label{eq:precision}
\text{Precision} = \frac{\text{true positives}}{\text{true positives} + \text{false positives}}
\end{equation}

\begin{equation}\label{eq:recall}
\text{Recall} = \frac{\text{true positives}}{\text{true positives} + \text{false negatives}}
\end{equation}

\subsubsection{Results}
A barchart of results can be seen in \cref{fig:classifier_comparison}. \emph{QDA} gives the best precision of $0.985$. However the recall is low, which is concerning as this means the number of suggestions are reduced. The next best result, \emph{Nearest Centroid}, gives a precision of $0.979$, but has a far better recall. Because of the concern regarding suggestions, we choose \emph{Nearest Centroid} as our classifier.


\begin{figure}[tbp]
\label{fig:classifier_comparison}
\includegraphics{classifier-comparison.pdf}
\caption{Precision and recall scores for the chosen algorithms. The algorithms are sorted ascending by precision.}
\end{figure}

% \begin{table}[tbp]
% \centering
% \begin{tabular}{@{}lp{.75\textwidth}@{}}
% \toprule
% \textbf{Classifier} & \textbf{Result} \\
% \midrule
% Baseline           &  DummyClassifier()                \\
% Nearest Neighbors  &  KNeighborsClassifier(3)          \\
% Linear SVM         &  SVC(kernel="linear", C=0.025)    \\
% RBF SVM            &  SVC(gamma=2, C=1)                \\
% Gaussian Process   &  GaussianProcessClassifier        \\
% Decision Tree      &  DecisionTreeClassifier           \\
% Random Forest      &  RandomForestClassifier           \\
% Neural Net         &  MLPClassifier, KerasClassifier   \\
% AdaBoost           &  AdaBoostClassifier()             \\
% Naive Bayes        &  GaussianNB()                     \\
% QDA                &  QuadraticDiscriminantAnalysis()  \\
% \bottomrule
% \end{tabular}
% \caption[Classifiers]{Classifiers}%
% \label{tab:classifiers}
% \end{table}


\todo{mangler afsnit om hyper parameter optimization af den fundne classifier. Motiver hvorfor vi gør det, og hvordan vi gør, lignende node2vec optimization. Præsenter også resultater for, hvad der virkede og hvor godt det var}

\todo{hvordan laver vi segway over mod ui afsnit?}


\section[UI]{User Interface}
% Første parameter i [] er tekst i header. {} er i indholdsfortegnelsen.

% Slide med emneoverskrift.
\begin{frame}
  \frametitle{}
  \begin{center}
    {\Huge User Interface}
  \end{center}
\end{frame}
\note{
  \begin{itemize}
		\item Introduce
    \item Design choices
    \item Testing
    \item Demo
  \end{itemize}
}

\begin{frame}
    \frametitle{The Design Process}
    \framesubtitle{Starting the process}
    \begin{itemize}
    	\item Design from requirements
    	\item Incremental development
    \end{itemize}
\end{frame}
\note{
	\begin{itemize}
    \item Use Web Engneering Techniques.
    \item built on requirement engineering.
    \item Little by Little.
    \item Fit well with project.
	\end{itemize}
}

\begin{frame}
    \frametitle{The Design Process}
    \framesubtitle{Gathering requirements}
    \begin{itemize}
    	\item Information Flow Diagram
    \end{itemize}
    \includegraphics[width=\textwidth]{wikiAPI.pdf}
\end{frame}
\note{
	\begin{itemize}
    \item Organizational Concerns.
    \item Information Delivery.
    \item Understanding our model wishes.
    \item Diagram
    \item Incremental -> simple editors
    \item Created two requirements
	\end{itemize}
}

\begin{frame}
    \frametitle{The Design Process}
    \framesubtitle{Gathering Requirements}
    \begin{itemize}
    	\item A user must be able to query the UI for link suggestions.
    	\item A user should be able to submit reviews of link suggestions.
    \end{itemize}
\end{frame}
\note{
	\begin{itemize}
    \item First and foremost
    \item Secondly -> reaction
    \item learn from it.
	\end{itemize}
}

\begin{frame}
    \frametitle{The Design Process}
    \framesubtitle{Creating a Solution}
    \begin{itemize}
    	\item User Demographic    
    	\item API > GUI
    \end{itemize}
\end{frame}
\note{
	\begin{itemize}
    \item Create Solution Idea
    \item STEP1: Analyze TARGET
    \item Speculation NO CONTACT
    \item Bacbone -> influence
    \item ----------
    \item Reasearch -> API
    \item access as possible
    \item direct GUI
    \item API = Broad dev
    \item realistic
    \item no initiation?
	\end{itemize}
}

\begin{frame}
    \frametitle{Testing the API}
    \framesubtitle{}
    \begin{itemize}
    	\item Unit testing
    	\item Acceptance test
    \end{itemize}
\end{frame}
\note{
	\begin{itemize}
    \item integrity
    \item Django helps
    \item Presedence
	\end{itemize}
}

\begin{frame}
  \frametitle{}
  \begin{center}
    {\Huge DEMO}
  \end{center}
\end{frame}
\note{
  \begin{itemize}
		\item Notes...
  \end{itemize}
}

% \begin{frame}
%     \frametitle{Some Example Title}
%     \framesubtitle{Some example subtitle}
%     \centering
%     Some text, content, etc.
% \end{frame}
% \note{
% 	\begin{itemize}
%     \item Notes here...
% 	\end{itemize}
% }
