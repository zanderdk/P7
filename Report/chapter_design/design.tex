\chapter{Design \& Implementation}\label{chap:design}
\todo{Design of the entire system. The nice picture we have with the system flow}

\begin{chapterorganization}
  \item In \sectionref{sec:architecture} we provide an overview of the full system architecture;
  \item In Section...
\end{chapterorganization}

\section{Overall View}
The main system consists of a pipeline utilizing two trained models in order to make predictions. These models are
\begin{enumerate*}[label=(\roman*)]
  \item a feature learning model, and
  \item a machine learning model.
\end{enumerate*}
Both models are trained by separate workflows. A diagram of the main pipeline and the separate workflows for training the models can be seen in \cref{fig:overall-flow} with the workflows for training marked with dashed lines.

\tikzsetnextfilename{diagram}
\begin{figure}[tbp]%
  \begin{tikzpicture}[node distance = 2.84cm, auto]
    
    
    % Place nodes
    \node [database, yshift=1em] (db) {DB};
    \node [bigblock, right of=db] (ap) {Article Picker};
    \node [bigblock, right of=ap] (tfl) {Feature Extraction};
    \node [bigblock, right of=tfl] (ai) {AI};
    \node [database, right of=ai] (db2) {DB2};
    \node [bigblock, below=1cm of db2] (web) {Web};
    
    \node [smallblock, below=1.5cm of ap, xshift=-2em] (n2v) {Node2Vec};
    \node [smallblock, below=.5cm of n2v] (paropt) {Parameter Optimizer};
    
    \node [smallblock, below=4.5cm of tfl] (prepper) {Training Data Generator};
    \node [smallblock, right of=prepper] (aitrainer) {AI Trainer};
    
    \begin{scope}[on background layer]
    \node [container, fit=(n2v)(paropt)] (container1) {};
    \node[below right] at (container1.north west) {Feature Learning};
    \node [container, fit=(prepper)(aitrainer)] (container2) {};
    \node[above right] at (container2.south west) {MI Learning};
    
    \node [container, fit=(db)(db2)] (container3) {};
    \end{scope}
    
    %\draw [->] (db) -- (n2v);
    
    \path [line] (db) -- (ap);
    \path [line] (ap) -- (tfl);
    \path [line] (tfl) -- (ai);
    \path [line] (ai) -- (db2);
    \path [line] (db2) -- (web);
    
    \path [line] (db.300) |- (n2v);
    %\path [line] (db.300) |- (paropt);
    \path [line] (db.240) |- (prepper);
    
    \path [line, swap] (paropt) -- node{Parameters} (n2v);
    \path [line] (n2v.east) -| node [near end] {Model} ([xshift=-1.2em]tfl.south);
    
    \path [line] (prepper) -- (aitrainer);
    %\path [line] (prepper) -- (tfl);
    %\path [line] ([xshift=1.2em]tfl.south) -- ([xshift=1.2em]prepper.north);
    \draw [line] (prepper.north) |- ([xshift=1.2em, yshift=0.5em]tfl.south) -- ([xshift=1.2em]prepper.north);
    
    \path [line] (aitrainer) -- node{Model} (ai);
    
  \end{tikzpicture}
\caption[short desc]{Text}%
\label{fig:overall-flow}%
\end{figure}

%Throughout this chapter the different components will be explained in detail.
In the following sections, these areas will be described.

\subsection{Main Pipeline}
The main part of the system architecture consists of a pipeline concerning the components: database, article picker, feature extractor, AI, and Web.

When the system is started, the article picker component will extract pairs of articles from the database. A pair is a source article and a target candidate article that should be tested for missing links. A feature vector is then extracted from each of these pairs by the feature extractor component, based on a model that has been trained ahead of time. This vector of features is then supplied to the AI component, which will then make a prediction on whether or not the pair should be linked.

If a pair is predicted to have a missing link, it will be added to another database. This database holds all the pairs suggested for links, until a user extracts this link with the intention of deciding whether to add a link or not.

%The article picker component queries the database component for a source article and a list of candidate target articles that should be tested for missing links. The articles are arranged into pairs and forwarded to the feature extraction component. Previously learned features, for the article pairs, are extracted as vectors which are fed into the AI component. The AI uses the feature vectors to predict whether or not a link should be present from the source article to the targets. Finally this classification is forwarded to the Web component which presents relevant information to the user.

\subsection{Support pipelines}
\todo{check this after the new diagram has been made}
Prior to this, article features must be learned, and the AI model must be trained. The is performed by two support pipelines.

Meanwhile in the \textit{N2V Gym\texttrademark}. Feature learning is done by using node2vec for performing random walks between articles within the link graph gathered from the database. The walks are used to train the node2vec model which ultimately is used by the feature extractor.
A parameter optimization component is used to find node2vec parameters that increases accuracy for the Wikipedia link data.

The machine learning pipeline aka \textit{AI Gym\texttrademark} uses a list of labeled article pairs to train the AI model. The articles are gathered from the database and features are extracted using the feature extractor. The features and classification labels are passed to the AI Trainer component which trains the model that will be used by the AI component.


\section{Overview of Architecture}\label{sec:architecture}
\todo{Nice picture of entire architecture}

\section{Components}

\subsection{Article Picker}
The article picker component of the system is used to select candidate article pairs to test for linking.

The article pairs are selected in accordance with a policy, because picking articles at random would yield many irrelevant pairs ($\left\vert{\text{articles}}\right\vert ^{2}$ pairs). We base the policy on teleport links from the Wikipedia clickstream and thus human navigation paths. The reasoning is that by selecting article pairs without an existing link that users have navigated between by other means, the articles might be related and inserting a link between the articles could improve navigability.

\todo{write if we end up doing this}
Another policy to consider is selecting article pairs based on text content. Specifically using n-grams of words (shingles) to find text in the article that references titles of other articles without already linking to it. 

\todo{teleport/clickstream like sources}
% https://cs.stanford.edu/people/jure/pubs/wiki-www15.pdf
% others

\subsection{Feature Extractor}
The feature extractor receives a pair of articles and extracts features from a node2vec model that has been trained ahead of time. The feature pair are represented as two vectors which is combined using an operation such as multiplication \todo{maybe write that we always multiply them, maybe cite node2vec paper, that says that multiplication yielded best results in their experiments.} and used as input for the AI.

\section{Features}

\section{Machine Intelligence}

\section{Graphical User Interface}
??? Profit!