\section{Feature Extractor}\label{feature_extractor}
%Here we input the intro text.
%This paragraph covers the first point of the intro from the writing guide
The purpose of the feature extractor component, is to generate a feature representation of a candidate pair as described by the function $f$ in \cref{sec:ml_def}, such that the classifier component can decide whether to suggest a link or not. This is done through a feature learning approach, as chosen in \cref{sec:feature_generation}.

%The two paragraphs below covers the second point of the writing guide
In order to do generate these features we use an algortimic framework for semi-supervised feature learning called node2vec~\cite{node2vec}. Node2vec is a way of creating feature for nodes in a graph based on structure, which can then be used to identify characteristics of the relation between two nodes. When we have identified these characteristics, our classifier component can use them as the feature representation for our candidate pairs. We choose to use node2vec based on reports of its performance as well as its flexibility, seen in~\cite{node2vec}.

In this section we will cover a basic theoretical understanding of node2vec, in order to consider the implications that this has on our usage of node2vec. Afterwards we will cover how we adapt node2vec to our specific problem, and at last we consider then implementation of the feature learner support module and the feature extractor.


%The next component in the main pipeline is the feature extractor. This component generates the feature representation of a candidate pair as described by the function $f$ in \cref{sec:ml_def}, which can then be used for classification.

%The feature extractor returns a single feature vector, representing the link between a given source and target article pair. As we need to classify both existing and non-existing links, it must be possible to extract feature vectors in both cases. Therefore the feature vector for a link is constructed by combining the feature vectors of the source and target articles.

%As mentioned in \cref{sec:feature_generation} we aim to use feature learning, specifically network embedding, to learn these feature representations. While there are several different ways to approach network embedding, we found node2vec~\cite{node2vec} to be suitable. The main advantage of node2vec is increased flexibility compared to other well-known approaches due to tunable hyperparameters~\cite{node2vec}, which allows us to experiment with different neighborhood exploration methods Additionally, node2vec offers highly competitive performance, and performs well for large networks~\cite{node2vec}.

%This section describes how node2vec is used to generate the feature representation, along with the process of optimizing the models hyperparameters.

%node2vec is an algorithmic framework for semi-supervised feature learning in networks~\cite{node2vec}. The goal of the algorithm is to learn feature representations for nodes in a network. To find feature representations for edges, the feature representation for two nodes can be combined. The node2vec paper proposes a way of generating features representations for nodes based on their neighborhoods. The idea is to use word embedding, where nodes are used as words, and node sequences are used as sentences. The neighborhood of a node consists of the nodes that are close in the node sequence. In the following sections we briefly describe word embedding, and how node sequences are constructed using biased random walks in the graph. Furthermore we describe the approaches to combine node vectors presented in \cite{node2vec}, and propose a new combination method tailored for this project.
\subsection{node2vec Overview}\label{sec:node2vec}
The node2vec framework employs a network embedding approach to the task of learning features for nodes in a graph structure. Network embedding entails an embedding of information into each node, such that feature representations can be derived. In our case, and the case of node2vec, we embed structural information.

\subsubsection{Word Embedding}
The technique which node2vec uses for network embedding is built upon a similar framework called word2vec~\cite{word2vec}. word2vec performs word embedding in text, and similarly to network embedding, this is a way of embedding each word with information. We motivate the use of node2vec by introducing a typical word2vec example, to illustrate the concept behind embedding techniques.

The core idea behind word2vec is that every word can be characterized by its relation to other words. An example could be a text where the words \enquote{king} and \enquote{queen} are both equally related to the word \enquote{regent}, but \enquote{king} has a closer relation to the word \enquote{man}, and \enquote{queen} has a similar relation to the word \enquote{woman}.

These relations, or word context, are discovered by considering which words occur in the same sentences. word2vec examines every sentence in a body of text, and aggregates the analysis of the word relations, such that information on every word is gathered.
%For the case of node2vec, word2vec uses the skip-gram model for learning feature vectors~\cite{DBLP:journals/corr/Rong14}\todo{?}. In short, skip-gram uses a neural network with a single hidden layer, for predicting the word context of a single word. Given a sequence of words, the context window of a word $w_{i}$ is defined as $\{w_{i-k}, \ldots ,w_{i-1}, \ w_{i+1}, \ldots ,w_{i+k}\}$.\todo{?} The feature representation of a given word $w_{i}$ is then given by the $i^{th}$ row of the neural network's weight matrix from the input layer to the hidden layer. The amount of nodes in the hidden layer should match the number of desired features $d$.\todo{?}

node2vec employs the same approach to nodes, in that each node is characterized by its relation to other nodes, both the direct relations seen as edges but also indirect relations across multiple edges.
%But since there is no predifined replacement for sentences in the context of graph analysis, node2vec creates a graph equivalent.
In the same way that word2vec considers sequences of words, i.e.\ sentences, node2vec considers sequences of nodes. These sequences are generated by performing multiple biased random walks throughout the graph.

%\subsubsection{Word Embedding}
%Using nodes as words and node sequences generated by biased random walks as sentences allows the use of word embedding algorithms on the sentences, mapping words to vectors. The node2vec reference implementation uses a well-known model for word embedding called word2vec. Intuitively, two words are related if they are close to each other in a sentence. A parameter $k$ specifies the size of a context window that defines the number of surrounding words that should be considered in the same context. In node2vec the context window is used for sampling neighborhoods of nodes, where it determines the size of the neighborhood.

\subsubsection{Biased Random Walks}
In textual analysis the ordering of words in a sentence provides structure, whereas nodes in a graph does not have any equivalent ordering. node2vec derives the structure of a graph through biased random walks. By walking randomly through a graph multiple times,
node2vec samples sequences of nodes, that can be used in a similar way as sentences are used in word embedding.

Biasing the random walks in different ways can account for different types of graph structures, by influencing the walks to represent the most significant structures in the graph. The random walks are biased by adjusting the probability of following certain types of edges in the graph. This is done through two parameters, $p$ and $q$. An example of the biased random walk is shown in \cref{fig:randomwalk}. In this example, the walk is currently in $v$, after walking the edge $(t,v)$. Here a search bias $\alpha$ is given to each outgoing edge from $v$ according to \cref{eq:bias}, where $d_{tx}$ denotes the shortest path distance from $t$ to $x$. The next edge in the walk is selected through probability sampling using the alias method~\cite{alias-method}.

\begin{figure}%
  \centering
  \tikzsetnextfilename{nodewalk}
  \begin{tikzpicture}[node distance = 1.7cm, auto]
    \node [node] (v) {$v$};
    \node [node, fill=none, above left=of v,
            pin={[pin distance=1.2em,pin edge={very thick}]15:},
            pin={[pin distance=1.2em,pin edge={very thick}]150:},
            pin={[pin distance=1.2em,pin edge={very thick}]230:}
            ] (x1) {$x_1$};
    \node [node, fill=none, above right=of v,
            pin={[pin distance=1.2em,pin edge={very thick}]350:},
            pin={[pin distance=1.2em,pin edge={very thick}]295:},
            pin={[pin distance=1.2em,pin edge={very thick}]150:}
            ] (x2) {$x_2$};
    \node [node, fill=none, below left=of v,
            pin={[pin distance=1.2em,pin edge={very thick}]170:},
            pin={[pin distance=1.2em,pin edge={very thick}]210:}
            ] (t) {$t$};
    \node [node, fill=none, below right=of v,
            pin={[pin distance=1.2em,pin edge={very thick}]330:},
            pin={[pin distance=1.2em,pin edge={very thick}]20:}
            ] (x3) {$x_3$};
    \path [draw, very thick, inner sep=.5pt] (x1) -- node [] {$\alpha = 1$} (v);
    \path [draw, very thick, inner sep=.5pt] (x2) -- node [] {$\alpha = 1/q$} (v);
    \path [draw, very thick, swap, inner sep=.5pt] (x3) -- node [] {$\alpha = 1/q$} (v);
    \path [draw, very thick] (t) -- (x1);
    \path [draw, very thick, swap, inner sep=.5pt] (t) -- node [] {$\alpha = 1/p$} (v);
\end{tikzpicture}
\caption[Illustration of random walk in node2vec]{Illustration of the random walk procedure in node2vec. The walk is currently in $v$ and came from $t$. It is now evaluating its next step out of node $v$. Adapted from~\cite{node2vec}.}%
\label{fig:randomwalk}%
\end{figure}

\begin{equation}
\label{eq:bias}
\alpha_{pq}(t,x)=
\begin{cases}
  \frac{1}{p} & \text{if } d_{tx}=0 \\
  1           & \text{if } d_{tx}=1 \\
  \frac{1}{q} & \text{if } d_{tx}=2
\end{cases}
\end{equation}

%%%%%% YES, THERE IS A D in BREADTH. IT IS NOT SPELLED BREATH-FIRST SEARCH!  %%%%%%%%
By adjusting $p$ and $q$ it is possible to approximate different search strategies. A high $p$ value, relative to $\max(1,q)$, approximates a depth-first search, while a high $q$ value, relative to $\max(1,p)$, approximates the behavior of a breadth-first search. According to \cite{node2vec} an approximation of depth-first search will result in more structural information across the entire graph, where an approximation of a breadth-first search will yield information on communities in the graph.
%%%%%% YES, THERE IS A D in BREADTH. IT IS NOT SPELLED BREATH-FIRST SEARCH!  %%%%%%%%
Once an adequate number of walks have been completed, the word2vec framework is used to analyze them, and produces the feature representations of the nodes in the graph.
\subsection{Applying node2vec}
In this section we describe how we apply the node2vec framework in the implementation of the feature extractor.

\subsubsection{Combining Feature Vectors}\label{subsub:combining_feature_vectors}
By learning a node2vec model on the graph, we are able to learn a function $k:V \to \mathbb{R}^d$, mapping a node to a vector representation of features. However, since we wish to classify connections between article pairs, we need a way of combining the feature vectors of two nodes into a single feature vector for one pair. As described in \cref{sec:ml_def} the function $f$ maps article pairs to feature vectors. In \cref{squareDotProduct} we define it based on $k$ and a binary operation $\circledcirc$ used for combining feature vectors.

\begin{equation}\label{squareDotProduct}
f(a,b) = k(a) \circledcirc k(b)
\end{equation}

In~\cite{node2vec}, four different binary operators for combining two feature vectors are examined, with \emph{Hadamard product} yielding the best results in all tested cases.
%$\circ : \mathbb{R}^d \times \mathbb{R}^d \to \mathbb{R}^d$
All of the four operations are commutative, which is suitable for undirected graphs, where the combined result vector does not depend on the order of the nodes. For directed graphs however, the order is important. If we are given two article pairs $(a,b)$ and $(b,a)$ where $a \Rightarrow b$ and $b \not \Rightarrow a$, their feature representations should be different because they have different labels. 

We test this claim in \cref{sec:hyperopt} by comparing the performance of three different binary operations, seen in \cref{table:binary_operators}, as a part of the parameter optimization component. The tested operations are Hadamard product, Hadamard division, and concatenation. Hadamard product is tested because it is the preferred choice in~\cite{node2vec}. Hadamard division is examined because of its relation to Hadamard product, but with a non-commutative property. Concatenation of the feature vectors is chosen since it preserves all features in their original state.

\makeatletter
\newcommand*{\boxwedge}{%
  \mathbin{%
    \mathpalette\@boxwedge{}%
  }%
}
\newcommand*{\@boxwedge}[2]{%
  % #1: math style
  % #2: unused
  \sbox0{$#1\boxplus\m@th$}%
  \dimen2=.5\dimexpr\wd0-\ht0-\dp0\relax % side bearing
  \dimen@=\dimexpr\ht0+\dp0\relax
  \def\lw{.09}% linw width as factor for height of \boxplus
  \kern\dimen2 % side bearing
  \tikz[
    line width=\lw\dimen@,
    line join=square,
    x=\dimen@,
    y=\dimen@,
  ]
  \draw
    (\lw/2,0) rectangle (1-\lw,1-\lw)
    (0+\lw/2,1-\lw) -- (1-\lw ,0)
  ;%
  \kern\dimen2 % side bearing
}

\begin{table}[tbp]
\centering
\begin{tabular}{@{}lcl@{}}
\toprule
\textbf{Operator} & \textbf{Symbol} & \textbf{Definition} \\
\midrule
Hadamard product & $\boxdot$ & $\lbrack f(u) \boxdot f(v) \rbrack_{i} = f_i(u)*f_i(v) $ \\
Hadamard division & $\boxwedge$ & $\lbrack f(u) \boxwedge f(v) \rbrack_{i} = \frac{f_i(u) }{f_i(v) }$ \\
Concatenation & $^\frown$ & $f(u) {}^\frown f(v)  = \left[ \begin{smallmatrix}
           u \\ v
         \end{smallmatrix} \right]$ \\
\bottomrule
\end{tabular}
\caption[Binary operators]{Binary operators for combining features}\label{node2vec_operators}
\label{table:binary_operators}
\end{table}

%Because Hadamard product produced good results in \cite{node2vec} we will be evaluating it, and compare it to the corresponding non-commutative operation Hadamard division. Furthermore we choose to test our own method of concatenating the feature vectors $\mathbb{R}^d \times \mathbb{R}^d \to \mathbb{R}^{2d}$, leaving it up to the classifier to interpret the combination of article features.

\subsubsection{Implementation}
The reference implementation of node2vec~\cite{node2vec}, was not sufficient for our usage. Specifically, we had concerns regarding the amount of data we needed to analyze. node2vec as a framework is able to scale well, however, the reference implementation does not support parallelism. Therefore, we decided to implement our own version in order to gain better performance.

The primary speed up was gained by enabling random walks to be done in parallel. Since every walk is independent from each other, the problem of walking them is embarrassingly parallel~\cite{matloff2011art}. In our implementation each walk can be done in its own thread, allowing for linear speedup.

%In our case, our test machine had 16 CPU cores which we wanted to utilize. We therefore spawn 16 threads, each walking the graph, allowing full utilization of all the CPU cores.

Another major concern was regarding the memory footprint. Since all the biased random walks are completed before starting the analysis, it is likely that a big dataset would require a lot of memory. In order to avoid running out of memory, we continuously store the walks.

For the analysis we use an implementation of word2vec included in the library \emph{Gensim}~\cite{rehurek_lrec}, that covers our need for scalability in terms of CPU utilization.



\subsection{Feature Learning Support Module}
As seen in the architecture design in \cref{sec:design_overview}, the feature learning model is created in a support module. This support module consists of a parameter optimizer and a feature learner, where the parameter optimizer is responsible for identifying the hyperparameters which the feature learner will use. Here we will discuss the development of these two components.

\subsubsection{Hyperparameter Optimization}\label{sec:hyperopt}
\todo{hvis vi ikke skriver en sperat subsubsection om feature learner component, så skal denne overskrift nok heller ikke være der}
In order to produce a good feature learning model, we do an optimization process that attempts to find the optimal set of hyperparameters. An overiew of these parameters can be seen in \cref{tab:node2vecparams}. One of them is the choice of function for combining node features, as mentioned in \cref{subsub:combining_feature_vectors}. Since the number of parameters result in a rather large search space, we use a tool called Spearmint~\cite{snoek2012practical} for the task of approaching an optimal set of hyperparameters.

Spearmint performs bayesian optimization by maximizing the expected improvement, to efficiently search for parameters that will minimize an objective function. In our case we chose the objective function $1- F_{0.5}\text{-score}$.

We obtain the $F_{0.5}\text{-score}$ for a set of hyperparameters from the results of a stochastic gradient descent classification with constant parameters\todo{specific the details of this SGD}. The intuition is that a set of parameters that causes this classifier to perform well, will perhaps generalize to other classifiers.\todo{overvej hvordan denne pointe skal fremstå, så vi ikke kommer til at foreshadow'e for meget. Lad være med bare at slette den!}

%As described in \cref{sec:node2vec}, node2vec has many tunable hyperparameters. An overview of these can be seen in \cref{tab:node2vecparams}. In addition to the hyperparameters we will also examine the performance of different binary operations for combining node feature vectors. To find the parameters that give the best feature representation, we do a hyperparameter optimization pass. We first specify the parameter space that should be searched in. As this is a large space, it is not feasible to exhaust all possibilities. We therefore use the tool called Spearmint~\cite{snoek2012practical}, which performs Bayesian optimization by maximizing the expected improvement, to efficiently search for parameters that will minimize an objective function. The objective function is $1 - \text{F-score}$, where the F-score is found by running a classifier on the node2vec model learned from the parameters. By keeping the classifier and its parameters constant, we can find the best set of parameters by finding the lowest objective function value.

\begin{table}[tbp]
\centering
\begin{tabular}{@{}p{.15\textwidth}p{.81\textwidth}@{}}
\toprule
\textbf{Parameter} & \textbf{Description} \\
\midrule
$p$          &   Return parameter that controls the likelihood of immediately revisiting the previous node in the walk~\cite{node2vec}.   \\
$q$          &   Controls to which degree the walk prefers to stay in the neighborhood or to explore outwards.   \\
$d$          &   The dimensionality of the learned features.   \\
$r$          &   The number of random walks per node.   \\
$k$          &   The size of the context window.   \\
$l$          &   The max length of the random walk.   \\
function     &   The function to use for combining node features, it can be hadamard product, hadamard divide or concatenate.   \\
\bottomrule
\end{tabular}
\caption[Description of parameters in node2vec]{Description of the various parameters in node2vec}%
\label{tab:node2vecparams}
\end{table}
\todo{as of writing this todo, we are no longer mentioning the term \emph{context window}, since we do not really mention how the analysis is done. Consider the implications of this, if it is still the case}




%\todo{Beskriv at vi optimere på et sub sæt af graphen og at det er pga speed up.}
%\todo{del mængden er featured => good og det er fordi at så behøver vi ikke lave features for all noder men kun featured og good.}
In order to increase the amount of parameter sets we can examine within our available timespan, we speed up each iteration of the optimization. This is done by limiting node2vec to only consider a subset of the graph, as well as constraining the set of nodes, that walks can start from. The subset is the set of featured and good articles, and the set of start nodes is featured articles. This limitation reduces the amount of generated feature representations to the featured and good articles. This improves the runtime of the optimization process, but relies on the assumption that the parameters found here will generalize to the entire data set.
%By doing this, features are only generated for featured and good articles, which greatly improves performance and still makes the algorithm able to walk several nodes, because these articles are be expected to be reasonably linked. Due to this limitation it is only possible to find features for featured and good articles, which means that we can only run parameter experiments on the subset $\{(a,b) \in P \cup N \mid \text{b is a good article}\}$ of our training data.

The parameter optimization is entirely autonomous and the process should be able to optimize the the parameters automatically. This enables learning on graph networks without knowing the underlying graph structure.

We start with a coarse search to find a good range for some parameters. This is done with large grained values for each parameter that would show local optima. After this, we search a second time with finer grained values to further refine the local optima. After completing 83 experiments the objective function seemed to have converged, as shown on \cref{fig:spearmint}. The best parameters found are shown in~\cref{tab:paramopt_goodvalues}.

\begin{figure}%
  \centering
  \tikzsetnextfilename{spearmint}
  \begin{tikzpicture}
  \begin{axis}[
    scale only axis,
    height=5cm,
    width=0.9\textwidth,
    xmin=0, xmax=90,
    ymin=0, ymax=1,
    legend entries={Unsorted, Sorted},
    legend pos=north east,
    legend style={draw=none},
    xlabel={Iteration},
    ylabel={Score}
  ]
    \addplot[mark=square*, color=color4!30!white] table [x=x, y=y, col sep=semicolon] {chapter_design/unsorted.csv};

    \addplot[
      scatter,
      color=color2,
      scatter src=explicit symbolic,
      scatter/classes={
          normal={mark=*, color2},%
          special={mark=*, colorG, draw=colorGshade}%
      }%,
      %pins near some coords={15/south:Hadamard,29/north:Divide}
    ] table [x=x, y=y, col sep=semicolon, meta=color] {chapter_design/sorted.csv}
    ;
    \node[coordinate,pin={[pin distance=1.1cm]above:{Hadamard product}}]
      at (axis cs:16,0.55340945711058)	{};

    \node[coordinate,pin={[pin distance=.9cm]below:{Hadamard division}}]
      at (axis cs:30,0.4252592596720848)	{};

    \node[coordinate,pin=above:{Concatenate}]
      at (axis cs:84,0.19147061365923185)	{};

  \end{axis}
\end{tikzpicture}
\caption[Hyperparameter optimization]{Graph showing the progress of the hyperparameter optimization process}%
\label{fig:spearmint}%
\end{figure}

\begin{table}%
\centering
\begin{tabular}{ccccc}
\toprule
$p$  & $q$     & $d$ & $k$ & $l$ \\
\num{0.50} & \num{100000} & 256 & 80  & 80 \\
\bottomrule
\end{tabular}
\caption[The parameter values found to perform the best]{The parameter values that performed best}%
\label{tab:paramopt_goodvalues}%
\end{table}

Experiments in~\cite{node2vec} show that increasing the $r$ value (number of walks per node) improves the quality of the generated features. Knowing this, we speed up the parameter optimization by using a constant value of $r=1$, which is useful for comparing the sensitivity of the other parameters. We are aware that a higher $r$ value should be used in the final node2vec training.

Concatenating the feature vectors generally gave better results compared to using Hadamard product and division.

Since the constant $\alpha = 1$, and the value $1/p$ are much higher than $1/q$ it is most likely that the walk will remain close within the community of the node in which the walk started.

The size of the context window $k$ did not have much impact on the objective function as long as it was around a value of $80$.

\todo{Maybe comment on d and l?}
%The maximum length of the walk $l=80$

\todo{Add a closing bit that includes commentary on how we use the parameter optimization in our feature learner component. Spørg Kasper}
