\section{Feature Extractor}\label{feature_extractor}
%Here we input the intro text.
%This paragraph covers the first point of the intro from the writing guide
The purpose of the feature extractor component, is to generate a feature representation of a candidate pair as described by the function $f$ in \cref{sec:ml_def}, such that the classifier component can decide whether to suggest a link or not. This is done through a feature learning approach, as chosen in \cref{sec:feature_generation}.

%The two paragraphs below covers the second point of the writing guide
In order to do generate these features we use an algortimic framework for semi-supervised feature learning called node2vec~\cite{node2vec}. Node2vec is a way of creating feature for nodes in a graph based on structure, which can then be used to identify characteristics of the relation between two nodes. When we have identified these characteristics, our classifier component can use them as the feature representation for our candidate pairs. We choose to use node2vec based on reports of its performance as well as its flexibility, seen in~\cite{node2vec}.

In this section we will cover a basic theoretical understanding of node2vec, in order to consider the implications that this has on our usage of node2vec. Afterwards we will cover how we adapt node2vec to our specific problem, and at last we consider then implementation of the feature learner support module and the feature extractor.


%The next component in the main pipeline is the feature extractor. This component generates the feature representation of a candidate pair as described by the function $f$ in \cref{sec:ml_def}, which can then be used for classification.

%The feature extractor returns a single feature vector, representing the link between a given source and target article pair. As we need to classify both existing and non-existing links, it must be possible to extract feature vectors in both cases. Therefore the feature vector for a link is constructed by combining the feature vectors of the source and target articles.

%As mentioned in \cref{sec:feature_generation} we aim to use feature learning, specifically network embedding, to learn these feature representations. While there are several different ways to approach network embedding, we found node2vec~\cite{node2vec} to be suitable. The main advantage of node2vec is increased flexibility compared to other well-known approaches due to tunable hyperparameters~\cite{node2vec}, which allows us to experiment with different neighborhood exploration methods Additionally, node2vec offers highly competitive performance, and performs well for large networks~\cite{node2vec}.

%This section describes how node2vec is used to generate the feature representation, along with the process of optimizing the models hyperparameters.

%node2vec is an algorithmic framework for semi-supervised feature learning in networks~\cite{node2vec}. The goal of the algorithm is to learn feature representations for nodes in a network. To find feature representations for edges, the feature representation for two nodes can be combined. The node2vec paper proposes a way of generating features representations for nodes based on their neighborhoods. The idea is to use word embedding, where nodes are used as words, and node sequences are used as sentences. The neighborhood of a node consists of the nodes that are close in the node sequence. In the following sections we briefly describe word embedding, and how node sequences are constructed using biased random walks in the graph. Furthermore we describe the approaches to combine node vectors presented in \cite{node2vec}, and propose a new combination method tailored for this project.

\subsection{Description of node2vec}
\label{sec:node2vec}
node2vec is an algorithmic framework for semi-supervised feature learning in networks~\cite{node2vec}. The goal of the algorithm is to learn feature representations for nodes in a network. To find feature representations for edges, the feature representation for two nodes can be combined. The node2vec paper proposes a way of generating features representations for nodes based on their neighborhoods. The idea is to use word embedding, where nodes are used as words, and node sequences are used as sentences. The neighborhood of a node consists of the nodes that are close in the node sequence. In the following sections we briefly describe word embedding, and how node sequences are constructed using biased random walks in the graph. Furthermore we describe the approaches to combine node vectors presented in \cite{node2vec}, and propose a new combination method tailored for this project.

\subsubsection{Word Embedding}
Using nodes as words and node sequences generated by biased random walks as sentences allows the use of word embedding algorithms on the sentences, mapping words to vectors. The node2vec reference implementation uses a well-known model for word embedding called word2vec. Intuitively, two words are related if they are close to each other in a sentence. A parameter $k$ specifies the size of a context window that defines the number of surrounding words that should be considered in the same context. In node2vec the context window is used for sampling neighborhoods of nodes, where it determines the size of the neighborhood.

\subsubsection{Neighborhoods}
The notion of neighborhoods is important in network embedding, as the features of a node is derived from its neighborhood. Intuitively, this implies similarity between nodes that have similar neighborhoods. The neighborhood of a given node is found by traversing the graph.

In node2vec the neighborhoods of each node $n$ is found by performing a series of biased random walks, and then observing the $k$ nearest nodes on either side of $n$ in the walks. To ensure that all nodes are visited during the random walks a number of walks, specified by the $r$ parameter, are started in every node. The maximum length of each walk is specified by the $l$ parameter.

The random walks are biased using two parameters, $p$ and $q$. An example of the biased random walk is shown in \cref{fig:randomwalk}. The walk is currently in $v$, after walking the edge $(t,v)$. Here a search bias $\alpha$ is given to each outgoing edge from $v$ according to \cref{eq:bias}, where $d_{tx}$ denotes the shortest distance from $t$ to $x$. The next edge in the walk is selected by probability sampling using the alias method.

\begin{equation}
\label{eq:bias}
\alpha_{pq}(t,x)=
\begin{cases}
  \frac{1}{p} & \text{if } d_{tx}=0 \\
  1           & \text{if } d_{tx}=1 \\
  \frac{1}{q} & \text{if } d_{tx}=2
\end{cases}
\end{equation}

By adjusting $p$ and $q$ it is possible to approximate different search strategies. A high $p$ value, relative to $\max(1,q)$, approximates a depth-first search, while a high $q$ value, relative to $\max(1,p)$, approximates the behavior of a breath-first search. The motivation for tuning these parameters is the observation that real-world networks can be structured in many ways. The biased random walks with the tunable $p$ and $q$ values accounts for different types of networks, by allowing control of the exploration method.

\begin{figure}%
  \centering
  \tikzsetnextfilename{nodewalk}
  \begin{tikzpicture}[node distance = 1.7cm, auto]
      \node [node] (v) {$v$};
      \node [node, fill=none, above left=of v,
              pin={[pin distance=1.2em,pin edge={very thick}]15:},
              pin={[pin distance=1.2em,pin edge={very thick}]150:},
              pin={[pin distance=1.2em,pin edge={very thick}]230:}
              ] (x1) {$x_1$};
      \node [node, fill=none, above right=of v,
              pin={[pin distance=1.2em,pin edge={very thick}]350:},
              pin={[pin distance=1.2em,pin edge={very thick}]295:},
              pin={[pin distance=1.2em,pin edge={very thick}]150:}
              ] (x2) {$x_2$};
      \node [node, fill=none, below left=of v,
              pin={[pin distance=1.2em,pin edge={very thick}]170:},
              pin={[pin distance=1.2em,pin edge={very thick}]210:}
              ] (t) {$t$};
      \node [node, fill=none, below right=of v,
              pin={[pin distance=1.2em,pin edge={very thick}]330:},
              pin={[pin distance=1.2em,pin edge={very thick}]20:}
              ] (x3) {$x_3$};
      \path [draw, very thick, inner sep=.5pt] (x1) -- node [] {$\alpha = 1$} (v);
      \path [draw, very thick, inner sep=.5pt] (x2) -- node [] {$\alpha = 1/q$} (v);
      \path [draw, very thick, swap, inner sep=.5pt] (x3) -- node [] {$\alpha = 1/q$} (v);
      \path [draw, very thick] (t) -- (x1);
      \path [draw, very thick, swap, inner sep=.5pt] (t) -- node [] {$\alpha = 1/p$} (v);
  \end{tikzpicture}
\caption[Illustration of random walk in node2vec]{Illustration of the random walk procedure in node2vec. The walk is currently in $v$ and came from $t$. It is now evaluating its next step out of node $v$. Adapted from~\cite{node2vec}.}%
\label{fig:randomwalk}%
\end{figure}

\subsubsection{Combining Feature Vectors}
By training a node2vec model on the graph, we are able to learn a function $h:V \to \mathbb{R}^d$, mapping a node to a vector representation of features. In order to construct a feature representation of a potential edge between two nodes, we need to combine their feature vectors. In the node2vec paper~\cite{node2vec} four different binary operators $\circ : \mathbb{R}^d \times \mathbb{R}^d \to \mathbb{R}^d$ for combining two feature vectors are examined, with \emph{Hadamard product} yielding the best results in all tested cases.

%We discovered that all 4 operators where commutative meaning that $a \circ b = b \circ a \ \forall a,b \in \mathbb{R}^d$ which works well in undirected graphs.

All of the four operations are commutative, which is suitable for undirected graphs, where the combined result vector does not depend on the order of the nodes. For directed graphs however, the order is important.

%If we have 2 pairs of articles $(a,b) \in \Rightarrow$ and $(b,a) \in V \times V \setminus \Rightarrow$ they will have the same feature representation using any of these 4 operators, but different labels.

If we are given two article pairs $(a,b)$ and $(b,a)$ where $a \Rightarrow b$ and $b \not \Rightarrow a$, their feature representations should intuitively be different because they have different labels.

%For this reason we will test different non-commutative operations.

Because Hadamard product produced good results in the node2vec paper we will be testing it and compare it to the corresponding non-commutative operation Hadamard division. Furthermore we choose to test concatenating the feature vectors $\mathbb{R}^d \times \mathbb{R}^d \to \mathbb{R}^{2d}$, leaving it up to the classifier to interpret the combination of article features.

% \section{Overview of node2vec} % (fold)
% \label{sec:overview_of_node2vec}

% Node2vec is an algorithmic framework for semi-supervised feature learning in networks~\cite{node2vec}. The goal of the algorithm is to learn a model that can reconstruct the neighborhood of a given node.

% The neighborhoods of each node is constructed by performing a number of biased random walks. The walks are biased using two parameters $p$ and $q$. A high $p$ value, relative to $max(1,q)$, approximates a depth-first search, while a high $q$ value, relative to $max(1,p)$, approximates the behavior of a breath-first search. The motivation of these tunable parameters is the observation that real-world networks can be structured in many ways. The randomness and the tunable $p$ and $q$ values accounts for different types of networks, by allowing control of the exploration method.

% Using nodes as words and the random walks as sentences allows using word embedding algorithms on the sentences, mapping words to vectors of real numbers. The node2vec reference implementation uses word2vec for word embedding. Intuitively, if two words are close two each other in the sentence, they are related. The tunable parameter $k$ specifies the size of a context window that defines the number of surrounding words that should be considered in the same context. In node2vec the context window is used for sampling neighborhoods of nodes, where it determines the size of the neighborhood.

% There are three additional parameters in node2vec, $d$, $r$, and $l$, which determines the dimensions of the resulting feature vector, the number of random walks performed per node, and the maximum length of each walk, respectively.

% By training a node2vec model on the graph, we are able to learn feature representations for each node. However, we need to combine these features in order to construct a feature representation of a potential edge between the two nodes. We combine these two node feature vectors by using the Hadamard product, as this was found to give the best results in~\cite{node2vec}.
% section overview_of_node2vec (end)

% \tikzsetnextfilename{n2vbowtie}
% \begin{figure}[tbp]%
%   \centering
%     \begin{tikzpicture}[node distance = .9ex, auto, remember picture]
    \tikzstyle{n2vcontainer} = [rectangle, draw, inner sep=.5ex, fill=clshade]
    \tikzstyle{n2vc} = [thick, circle, draw, fill=clnode, text width=1.5ex, text centered, minimum height=1.5ex, inner sep=0pt]
    
    % LEFT SIDE (INPUT LAYER)
    \node [n2vc, label=left:{$x_1$}] (c1) {};
    \node [n2vc, below=of c1, label=left:{$x_2$}] (c2) {};
    \node [n2vc, below=of c2, label=left:{$x_3$}] (c3) {};
    \node [n2vc, below=of c3, draw=none, fill=none] (c3-h) {};
    \node [n2vc, below=of c3-h, label=left:{$x_k$}] (c4) {};
    \node at ($(c3)!.5!(c4)$) {$\vphantom{\int^0}\smash[t]{\vdots}$};
    \node [n2vc, below=of c4, draw=none, fill=none] (c4-h) {};
    \node [n2vc, below=of c4-h, draw=none, fill=none] (c4-hh) {};
    \node [n2vc, below=of c4-hh, label=left:{$x_V$}] (c5) {};
    \node at ($(c4)!.5!(c5)$) {$\vphantom{\int^0}\smash[t]{\vdots}$};
    % Gray background box:
    \begin{scope}[on background layer]
      \node [n2vcontainer, fit=(c1)(c5)] (inputcontainer) {};
      \node[above] at (inputcontainer.north) {Input layer};
    \end{scope}
    
    % RIGHT SIDE (OUTPUT LAYER)
    \node [n2vc, right=8.5cm of c1,label=right:{$y_1$}] (o1) {};
    \node [n2vc, below=of o1, label=right:{$y_2$}] (o2) {};
    \node [n2vc, below=of o2, label=right:{$y_3$}] (o3) {};
    \node [n2vc, below=of o3, draw=none, fill=none] (o3-h) {};
    \node [n2vc, below=of o3-h, label=right:{$y_j$}] (o4) {};
    \node at ($(o3)!.5!(o4)$) {$\vphantom{\int^0}\smash[t]{\vdots}$};
    \node [n2vc, below=of o4, draw=none, fill=none] (o4-h) {};
    \node [n2vc, below=of o4-h, draw=none, fill=none] (o4-hh) {};
    \node [n2vc, below=of o4-hh, label=right:{$y_V$}] (o5) {};
    \node at ($(o4)!.5!(o5)$) {$\vphantom{\int^0}\smash[t]{\vdots}$};
    % Gray background box:
    \begin{scope}[on background layer]
      \node [n2vcontainer, fit=(o1)(o5)] (outputcontainer) {};
      \node[above] at (outputcontainer.north) {Output layer};
    \end{scope}
    
    % MIDDLE (HIDDEN LAYER)
    \begin{scope}[xshift=4.4cm, node distance = .7ex]
    %\node [draw] at ($(inputcontainer)!.5!(outputcontainer)$) {
    %  \begin{tikzpicture}[remember picture]
            \node [n2vc, yshift=-1.5em, label=right:{$h_1$}] (h1) {};
            \node [n2vc, below=of h1, label=right:{$h_2$}] (h2) {};
            \node [n2vc, below=of h2, draw=none, fill=none] (h3-h) {};
            \node [n2vc, below=of h3-h, label=right:{$h_i$}] (h4) {};
            \node at ($(h2)!.5!(h4)$) {$\vphantom{\int^0}\smash[t]{\vdots}$};
            \node [n2vc, below=of h4, draw=none, fill=none] (h4-h) {};
            \node [n2vc, below=of h4-h, label=right:{$h_N$}] (h5) {};
            \node at ($(h4)!.5!(h5)$) {$\vphantom{\int^0}\smash[t]{\vdots}$};
            % Gray background box:
            \begin{scope}[on background layer]
              \node [n2vcontainer, fit=(h1)(h5)] (hiddencontainer) {};
              \node[above, yshift=1.5em] at (hiddencontainer.north) {Hidden layer};
            \end{scope}
    %    \end{tikzpicture}
    %};
    \end{scope}
    
    % LINES MAKING THE BOW TIE
    \draw (inputcontainer.north east) -- (hiddencontainer.north west);
    \draw (inputcontainer.south east) -- (hiddencontainer.south west);
    \draw (outputcontainer.north west) -- (hiddencontainer.north east);
    \draw (outputcontainer.south west) -- (hiddencontainer.south east);
    
    % The math stuff with the arrows 1
    \node at ($(inputcontainer)!.5!(hiddencontainer)$) {
      \begin{tikzpicture}[remember picture, node distance = 1ex]
        \draw [line] (0,0) -- (1,.5);
        \draw [line] (0,.5) -- (1,0);
        %\node [fit=(o1)(o5)] (outputcontainer) {};
        \node [below=1em of {(0.5,0)}] {$\mathbf{W}_{V\times N}=\{w_{ki}\}$};
      \end{tikzpicture}
    };
    
    % The math stuff with the arrows 2
    \node at ($(hiddencontainer)!.5!(outputcontainer)$) {
      \begin{tikzpicture}[remember picture, node distance = 1ex]
        \draw [line] (0,0) -- (1,.5);
        \draw [line] (0,.5) -- (1,0);
        %\node [fit=(o1)(o5)] (outputcontainer) {};
        \node [below=1em of {(0.5,0)}] {$\mathbf{W'}_{V\times N}=\{w'_{ij}\}$};
      \end{tikzpicture}
    };
    
  \end{tikzpicture}

% \caption[short desc]{Text. Adapted from \cite{word2vec-param-learning}}%
% \label{fig:n2v-figure}%
% \end{figure}

\subsection{Performance \& Scalability}
As also detailed in~\cite{node2vec}, node2vec is able to scale well. Every walk is independent from each other which means that many walks can take place in parallel; it is embarrassingly parallel. Our test machine has 16 CPU cores which we want to utilize. We therefore spawn 16 threads, each walking the graph. This allows full utilization of every CPU core.

As all walks have to be stored for the word2vec phase, we are concerned that the memory footprint of storing all walks in memory is too high. We therefore dump the walks to a file, and later stream this file into memory when training word2vec. \emph{Gensim}, the implementation of word2vec we use, is also able to fully utilize the CPU when training word2vec on the walks.

\subsection{Hyperparameter Optimization}\label{sec:hyperopt}
As described in \cref{sec:node2vec}, node2vec has many tunable hyperparameters. An overview of these can be seen in \cref{tab:node2vecparams}. In addition to the hyperparameters we will also examine the performance of different binary operations for combining node feature vectors. To find the parameters that give the best feature representation, we do a hyperparameter optimization pass. We first specify the parameter space that should be searched in. As this is a large space, it is not feasible to exhaust all possibilities. We therefore use the tool called Spearmint~\cite{snoek2012practical}, which performs Bayesian optimization by maximizing the expected improvement, to efficiently search for parameters that will minimize an objective function. The objective function is $1 - \text{F-score}$, where the F-score is found by running a classifier on the node2vec model learned from the parameters. By keeping the classifier and its parameters constant, we can find the best set of parameters by finding the lowest objective function value.

\begin{table}[tbp]
\centering
\begin{tabular}{@{}lp{.75\textwidth}@{}}
\toprule
\textbf{Parameter} & \textbf{Description} \\
\midrule
$p$          &   Return parameter that controls the likelihood of immediately revisiting the previous node in the walk~\cite{node2vec}.   \\
$q$          &   Controls to which degree the walk prefers to stay in the neighborhood or to explore outwards.   \\
$d$          &   The dimensionality of the learned features.   \\
$r$          &   The number of random walks per node.   \\
$k$          &   The size of the context window.   \\
$l$          &   The max length of the random walk.   \\
function     &   The function to use for combining node features, it can be hadamard product, hadamard divide or concatenate.   \\
\bottomrule
\end{tabular}
\caption[Description of parameters in node2vec]{Description of the various parameters in node2vec}%
\label{tab:node2vecparams}
\end{table}


%\todo{Beskriv at vi optimere på et sub sæt af graphen og at det er pga speed up.}
%\todo{del mængden er featured => good og det er fordi at så behøver vi ikke lave features for all noder men kun featured og good.}
In order to run a sufficient amount of parameter experiments within a reasonable timespan, we speed up each iteration of the optimization.
This is done by limiting node2vec to only consider a subset of the graph, as well as constraining the set of nodes a walk can start from. The subset consists of the set of featured and good articles, and the set of start nodes is featured articles. By doing this, features are only generated for featured and good articles, which greatly improves performance and still makes the algorithm able to walk several nodes because these types of articles can be expected to be reasonably linked. Due to this limitation it is only possible to find features for featured and good articles, this means that we can only run parameter experiments on the subset $\{(a,b) \in P \cup N \mid \text{b is good}\}$ of our training data.

%To speed up each iteration of the optimization, we limit the set of start nodes to featured articles.
The parameter optimization is entirely autonomous and the process should be able to optimize the the parameters automatically. This means that one can learn on graph networks without knowing the underlying graph structure.

We start with a coarse search to find a good range for some parameters. This is done with large grained values for each parameter that would show local optima. After this, we search a second time with finer grained values to further refine the local optima. After completing 83 experiments the objective function seemed to have converged, as shown on \cref{fig:spearmint}. The best parameters found are shown in~\cref{tab:paramopt_goodvalues}.

\begin{figure}%
  \centering
  \tikzsetnextfilename{spearmint}
  \begin{tikzpicture}
    \begin{axis}[
      scale only axis,
      height=5cm,
      width=0.9\textwidth,
      xmin=0, xmax=90,
      ymin=0, ymax=1,
      legend entries={Unsorted, Sorted},
      legend pos=north east,
      legend style={draw=none},
      xlabel={Iteration},
      ylabel={Score}
    ]
      \addplot[mark=square*, color=color4!30!white] table [x=x, y=y, col sep=semicolon] {chapter_design/unsorted.csv};

      \addplot[
        scatter,
        color=color2,
        scatter src=explicit symbolic,
        scatter/classes={
            normal={mark=*, color2},%
            special={mark=*, colorG, draw=colorGshade}%
        }%,
        %pins near some coords={15/south:Hadamard,29/north:Divide}
      ] table [x=x, y=y, col sep=semicolon, meta=color] {chapter_design/sorted.csv}
      ;
      \node[coordinate,pin={[pin distance=1cm]above:{Hadamard}}]
        at (axis cs:16,0.55340945711058)	{};

      \node[coordinate,pin={[pin distance=.6cm]below:{Divide}}]
        at (axis cs:30,0.4252592596720848)	{};

      \node[coordinate,pin=above:{Concatenate}]
        at (axis cs:84,0.19147061365923185)	{};

    \end{axis}
  \end{tikzpicture}
\caption[Hyperparameter optimization]{Graph showing the progress of the hyperparameter optimization process}%
\label{fig:spearmint}%
\end{figure}

\begin{table}%
\centering
\begin{tabular}{ccccc}
\toprule
$p$  & $q$     & $d$ & $k$ & $l$ \\
\num{0.50} & \num{100000} & 256 & 80  & 80 \\
\bottomrule
\end{tabular}
\caption[The parameter values found to perform the best]{The parameter values found to perform the best}%
\label{tab:paramopt_goodvalues}%
\end{table}

Experiments in~\cite{node2vec} show that increasing the $r$ value (number of walks per node) improves the quality of the generated features. Knowing this, we speed up the parameter optimization by using a constant value of $r=1$, which is useful for comparing the sensitivity of the other parameters. We are aware that a higher $r$ value should be used in the final node2vec training.

Concatenating the feature vectors generally gave better results compared to using Hadamard product and division.

Since the constant $\alpha = 1$, and the value $1/p$ are much higher than $1/q$ it is most likely that the walk will remain close within the community of the node in which the walk started.

The size of the context window $k$ did not have much impact on the objective function as long as it was around a value of $80$.

\todo{Maybe comment on d and l?}
%The maximum length of the walk $l=80$
