\section{Database}\label{sec:db}
The first component in our main pipeline is the database, as shown in \cref{fig:system-overview}. Our approach to identify missing links requires readily available data, which we facilitate by storing required data in a local database.
\todo{node2vertex det her afsnit? neo4j bruger selv node.}

\subsection{Database Design}\label{sec:db_design}
We use the native graph database system Neo4j~\cite{neo4j} as it performs well for graph based queries, and supports extensions to the query language with Java plugins. The data model in Neo4j consists of nodes, relationships between nodes, and properties on these. Each node and relationship can be annotated with a number of labels used to distinguish between different node and relationship types when querying the database.

We store Wikipedia pages as nodes with their title as a property, and links between pages as relationships. Labels are used to distinguish between different types of pages and links, as shown in \cref{tab:db_labels_nodes,tab:db_labels_edges}.

\begin{table}[tbp]
  \centering
  
    \begin{tabular}{@{}p{.20\textwidth}p{.60\textwidth}@{}}
      \toprule
      \textbf{Label}         & \textbf{Description}                            \\ \midrule
      \mono{Page}                   & A Wikipedia article                             \\
      \mono{FeaturedPage}           & A Wikipedia article marked as \emph{Featured}   \\
      \mono{GoodPage}               & A Wikipedia article marked as \emph{Good}       \\
      \mono{RedirectPage}           & A redirecting Wikipedia page                    \\
      \bottomrule
    \end{tabular}
    \caption[Node labels in the database]{Node labels in the database}%
    \label{tab:db_labels_nodes}
\end{table}
\begin{table}[tbp]
    \centering
    \begin{tabular}{@{}p{.20\textwidth}p{.60\textwidth}@{}}
      \toprule
      \textbf{Label}         & \textbf{Description}                            \\ \midrule
      \mono{LINKS\_TO}              & A link between two articles                     \\
      \mono{TRAINING\_DATA}         & A link that can be used during training         \\
      \mono{TEST\_DATA}             & A link that should only be using during testing \\
      \mono{REDIRECTS\_TO}          & An edge describing a redirect                   \\ \bottomrule
    \end{tabular}
    \caption[Edge labels in the database]{Edge labels in the database}%
    \label{tab:db_labels_edges}
\end{table}

\subsection{Populating the Database}\label{sec:db_populate}
We use two datasets provided by DBpedia~\cite{dbpedia} to populate the database; one contains links and the other redirects. Both datasets contain tuples in the following format: 

\begin{center}
\mono{\emph{<source\_page> <type> <target\_page>}}
\end{center}

We prune non-article pages based on the namespace prefixes used on Wikipedia. This removes meta pages such as \enquote{Talk:} pages. We also remove redirects from the link dataset as it both contains links and redirects. Links are split into three groups randomly, according to the partitioning given in \cref{sec:machine_learning_task}. \todo{fix ref, lige nu kommer forklaring på splittet senere}

From the pruned data we generate separate files containing pages, redirect pages, partitioned links, and redirect links. The files are imported with the appropriate labels, and additional labels are added to \emph{featured} and \emph{good} articles.
\todo{we removed some redirect nodes as well.} The counts for the different labels in the populated database is shown in \cref{tab:db_counts_nodes,tab:db_counts_edges}.

\begin{table}[tbp]
  \centering
  \begin{minipage}[t]{0.45\textwidth}
    \centering
    \begin{tabular}{@{}lr@{}}
      \toprule
      \textbf{Label}         & \textbf{Count}     \\
      \midrule
      \textit{Nodes (total)} & \textit{\num{18172630}}  \\
      \mono{Page}                   & \num{11159213}           \\
      \mono{FeaturedPage}           & \num{4820}               \\
      \mono{GoodPage}               & \num{23741}              \\
      \mono{RedirectPage}           & \num{7013417}            \\ \bottomrule
    \end{tabular}
    \caption[Counts for node labels]{Counts for node labels. Note that some nodes have multiple labels.}%
    \label{tab:db_counts_nodes}
  \end{minipage}
  \hfill
  \begin{minipage}[t]{0.45\textwidth}
    \centering
    \begin{tabular}{@{}lr@{}}
      \toprule
      \textbf{Label}         & \textbf{Count}     \\ \midrule
      \textit{Relationships (total)} & \textit{\num{145878042}} \\
      \mono{LINKS\_TO}              & \num{138422339}          \\
      \mono{TRAINING\_DATA}         & \num{294857}             \\
      \mono{TEST\_DATA}             & \num{147429}             \\
      \mono{REDIRECTS\_TO}          & \num{7013417}            \\ \bottomrule
    \end{tabular}
    \caption[Counts for edge labels]{Counts for edge labels.}%
    \label{tab:db_counts_edges}
  \end{minipage}

\end{table}
