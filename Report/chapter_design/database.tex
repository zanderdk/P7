\section{Database}\label{sec:db}
Our approach to identifying missing links requires readily available data, which we facilitate by storing the required data in a local database.

\subsection{Database Design}\label{sec:db_design}
Because of the need to represent Wikipedia as a graph, as mentioned in \cref{sec:choice_of_graph}, a natural choice for storage is to use a native graph database. We use Neo4j~\cite{neo4j} as it performs well for graph based queries, and supports extensions to its query language through Java plugins. Neo4j scales gracefully~\cite{neo4jscale} and servers can be upgraded or added until performance is satisfactory. The data model in Neo4j is based on nodes, relationships between nodes, and properties on these. Each node and relationship can be annotated with labels to distinguish between types when querying the database.

We store Wikipedia articles as nodes with their title as a property, and links between articles as relationships. Labels are used to distinguish between types of articles and links.

\newcolumntype{R}[1]{>{\raggedleft\arraybackslash}p{#1}}

\begin{table}[tbp]
  \centering
    \begin{tabular}{@{}p{.24\textwidth}p{.50\textwidth}R{.15\textwidth}@{}}
      \toprule
      \textbf{Label} & \textbf{Description} & \textbf{Count} \\
      %\midrule
      \mono{Article} & A Wikipedia article & \num{11159213} \\
      \mono{FeaturedArticle} & A Wikipedia article marked as \emph{Featured} & \num{4820} \\
      \mono{GoodArticle} & A Wikipedia article marked as \emph{Good} & \num{23741}\\
      \mono{RedirectPage} & A redirecting Wikipedia page &\num{7013417} \\
      \midrule
      & \multicolumn{1}{r}{\emph{Total number of nodes}} & \num{18172630} \\
      \bottomrule
    \end{tabular}
    \caption[Node labels in the database]{Node labels in the database. Note that some nodes have multiple labels.}%
    \label{tab:db_labels_nodes}
\end{table}

\begin{table}[tbp]
    \centering
    \begin{tabular}{@{}p{.24\textwidth}p{.50\textwidth}R{.15\textwidth}@{}}
      \toprule
      \textbf{Label} & \textbf{Description} & \textbf{Count} \\
      %\midrule
      \mono{LinksTo} & A link between two articles & \num{138422339} \\
      \mono{TrainingData} & A link that can be used during training & \num{294857} \\
      \mono{TestData} & A link that is used only during testing & \num{147429} \\
      \mono{RedirectsTo} & An edge describing a redirect & \num{7013417} \\
      \midrule
      & \multicolumn{1}{r}{\emph{Total number of relationships}} & \num{145878042} \\
      \bottomrule
    \end{tabular}
    \caption[Relationship labels in the database]{Relationship labels in the database}%
    \label{tab:db_labels_edges}
\end{table}

\subsection{Populating the Database}\label{sec:db_populate}
Wikipedia requests that bots are not used to crawl the articles~\cite{wiki-bots}, and so we use readily available datasets\footnote{Data extracted from Wikipedia dumps in October 2015} from DBpedia~\cite{dbpedia}. The datasets used are a list of all page links and a list of page redirects. The pages consists of articles as well as other Wikipedia pages, such as user-, talk-, and file-pages. Since we are only interested in articles, we prune non-article pages based on the namespace prefixes used on Wikipedia.

Redirects are mainly responsible for redirecting synonyms and common misspellings to the main article. The redirects are necessary to include since they are referred to by many links. We handle them by following them as needed when we later traverse the graph.

We currently store about 11 million articles of which only about 5 million are encyclopedia articles. The remaining articles are pages such as lists, disambiguation pages, and automatically generated pages with little or no content. While these pages are not encyclopedia articles, they still provide some structural information, therefore we chose to store these as well.

A node is created in the database for each Wikipedia article and redirect page in the datasets. The nodes are connected by adding relationships for the links and redirects.

While importing the articles and links into the database, additional labels \mono{FeaturedArticle} and \mono{GoodArticle} are added to \emph{featured} and \emph{good} articles respectively. A \emph{good} article~\cite{wiki-good-articles} meets the core set of editorial standards on Wikipedia, providing a satisfactory overall quality. They do not meet all of the criteria of a featured article such as the guidelines for appropriate linking. We will be using these \emph{good} articles for hyperparameter optimization in \cref{sec:hyperopt}. The labels used for nodes, along with the count for each label, is shown in \cref{tab:db_labels_nodes}.

The links are split into three groups, \mono{LinksTo}, \mono{TrainingData}, and \mono{TestData}. \mono{TrainingData} and \mono{TestData} are links from featured articles that will later be used to train and evaluate the classifier. \mono{LinksTo} are all of the remaining links. The splits are made randomly, according to a partitioning given in \cref{sec:classifier}. \Cref{tab:db_labels_edges} shows the different labels used for relationships, along with their counts.

%page_links_unredirected_en.ttl.bz2
%redirects.en.ttl.bz2

%Out of all articles, 0.04\% are featured, and 0.32\% of all links are links from featured articles. These are low proportions and might not generalize the entire graph. As described in \cref{sec:choice_of_graph}, links from featured articles is our ground truth for appropriate linking, and for that reason we want to learn the structure of these links even though they comprise a smaller proportion of the data.
%\todo{Vi kalder det ikke ground truth i graph def afsnit.}
%\todo{Vi har fjernet noget (proportions problematikken). Er det okay?}
