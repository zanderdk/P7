\section{Candidate Filtering}
The candidate filtering component performs the filtering step of the \emph{filtering and refinement} process. It selects candidate article pairs which later are refined in the classifier component.

The article pairs are selected based on a heuristic policy. We use clickstream~\cite{wiki-clickstream} data built from Wikipedia server logs, and as such it is based on user behavior. The data contains the number of clicks from an article $a$ to an article $b$ in the time frame where the data was collected, as well as the method of navigation of these clicks. One of these methods of navigation is a \emph{teleportation}, which means the user ended up on article $b$ after having viewed article $a$, but not following a link. \todo{Give good explanations of how the clicks then might have happened}

By selecting article pairs without an existing link and that users have navigated between by teleportation, the articles might be related and inserting a link between the articles might improve navigability.

\todo{Show code and implementation}

%\todo{write if we end up doing this}
%Another policy to consider is selecting article pairs based on text content. Specifically using n-grams of words (shingles) to find text in the article that references titles of other articles without already linking to it. 

%\todo{teleport/clickstream like sources}
% https://cs.stanford.edu/people/jure/pubs/wiki-www15.pdf
% others