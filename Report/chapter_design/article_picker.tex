\section{Candidate Generator}
\todo{Similar to featured articles, Wikipedia also has \emph{good articles} that follow most of the guidelines, providing a satisfactory overall quality~\cite{wiki-good-articles}. However, there is no requirement for them to follow the guidelines for proper linking.}

The candidate generator component is used in production after the classification models have been trained. The reason for doing this generation step, is that the number of possible article pair combinations is roughly 5 million articles squared, which totals to $2.5 \times 10^{13}$. Suppose we can process 1 million of these pairs per second, it will still take almost 290 days all of them. An input of this size will therefore be impossible to handle during this project period. However, the majority of these article pairs will not need a link, and we may thus be able to heuristically limit the input size, if we assume many article pairs can be discarded immediately --- but we need to generate these pairs dynamically instead of filtering. By filtering we encounter the same problem as before, since we have the same input size of $\num{5000000}^2$. By dynamically generating the pairs, we can generate a pair of articles as needed. This way, the computationally intensive evaluation is only performed on the data that is generated as needed.

The article pairs are selected based on a heuristic policy. The policy can be a combination of multiple approaches, as long as they provide ordered candidate pairs. A perfect policy provides pairs, where all the articles with missing links occur before any article without missing links. While such a policy is helpful, it is unrealistic and only serves as an example. As the policy is a heuristic, it only needs to perform reasonably well, which is why we do not employ any systematic evaluation of possible policies, but our choice is based on our own intuition. As long as the classifier suggests links at a sufficient rate, this way of choosing a policy will be acceptable.

Our choice of policy is a combination of two approaches. One is based on a clickstream dataset and the other on a simple syntactic analysis.

\subsection{Clickstream Approach}

The clickstream approach is inspired by related work that uses server logs to predict missing links~\cite{hyperlink-structure-using-logs}. We use clickstream~\cite{wiki-clickstream} data built from Wikipedia server logs, and as such it is based on user behavior. A clickstream data source is a list of requests that Wikipedia received within a given timespan\footnote{We use a data source collected from March 1 -- March 31, 2016}. A request holds information on a referrer and a resource, as well as a count of the occurrence of this request and a request type. The request type can be \emph{link}, \emph{external}, or \emph{other}. For this approach, the type \emph{other} is the only relevant one. A request of type \emph{other} means that the referrer and resource were both articles, but that referrer does not link to the resource. One source of this type of requests could be searches~\cite{wiki-clickstream}.

The heuristic for this approach, is that some of these requests will indicate that a user, after reading one article, was prompted to read another article, and that they did not have the possibility of following a link. While there are plenty of cases where this does not constitute a missing link, chances are that some of the requests are examples of this. The approach works by going through the clickstream data, picking out requests of type \emph{other}, and ranking them by their number of occurrences.

Though preliminary results with this approach have been promising, it only produces a limited amount of candidate pairs (1.7 million). Therefore, we choose to complement it with another approach to get a bigger set of candidate pairs.

\subsection{n-gram Approach}

For the complementary approach we took inspiration from~\cite{milne2008learning}, where they employ a syntactic analysis to find article pairs. The inspiration lies in their use of n-grams to find article pairs. For each article we create a set of shingles from an n-gram search. We then search through the titles of Wikipedia articles, for ones that are contained in at least one of the shingles. This gives us a large set of article pairs $(A,B)$, where article $B$ mentions article $A$.

However, this is a crude approach and a significant amount of the candidate pairs can easily be discarded. Consider the Wikipedia article on the word \enquote{the}. With the n-gram approach, nearly all 5.2 million articles would be in a candidate pair with this article, where only a fraction would be worth considering.

A way to combat this would be to order the candidate pairs according to the inverse frequency of target articles. The intuition is that articles, which are rarely mentioned across all articles, will be more significant when they \emph{are} mentioned. It is likely that a frequency threshold will be required, since the advantages of the approach becomes negligible at some sufficiently high frequency.

Another considerable problem with the approach is that while it does provide many candidate pairs, it can not be guaranteed to order all possible pairs, and therefore it will not completely cover the complement of the clickstream approach. A preliminary test with a 5-gram search through featured articles gave us more than a million candidate pairs. When all articles are searched, this approach will generate significantly more than the clickstream approach. This will be sufficient for our usage, and further scaling will not be implemented in this project.


%from an article $a$ to an article $b$ in the time frame where the data was collected, as well as the method of navigation of these clicks. One of these methods of navigation is a \emph{teleportation}, which means the user ended up on article $b$ after having viewed article $a$, but not following a link. \todo{Give good explanations of how the clicks then might have happened}

%By selecting article pairs without an existing link and that users have navigated between by teleportation, the articles might be related and inserting a link between the articles might improve navigability.

%\todo{Show code and implementation}

%\todo{write if we end up doing this}
%Another policy to consider is selecting article pairs based on text content. Specifically using n-grams of words (shingles) to find text in the article that references titles of other articles without already linking to it. 

%\todo{teleport/clickstream like sources}
% https://cs.stanford.edu/people/jure/pubs/wiki-www15.pdf
% others