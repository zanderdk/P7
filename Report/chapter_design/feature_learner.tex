\subsection{Feature Learner}
As discussed in \cref{sec:feature_generation}, there are multiple ways to find features. We will in this project focus on feature learning. \todo{This is not the place to take that decision. It is already taken in chapter 2?}

The feature learner component should be able to identify features suitable for classification problems. In our research on feature learning, node2vec~\cite{node2vec} appeared as a good candidate. It is designed to learn features based on the structure of graph networks. As we have been storing our Wikipedia data as a graph, node2vec seems fitting. node2vec is not the only solution to the problem domain of feature learning, but it is the approach we will use in this project.
\todo{more about other possible feature learning techniques we could have used?}

We want the feature learner to find feature vectors that can discriminate articles that should have a link, and articles that should not have a link. In the graph database, we already have knowledge about existing links. How is the feature learner supposed to reason about articles that should have a link, but currently does not?

\subsubsection{node2vec Overview}
As described in the node2vec paper~\cite{node2vec}, the graph structure is explored using biased random walks. The biasness comes from two parameters $p$ and $q$, that guides the random walk. A high $p$ \todo{ikke en høj p værdi men hvis $p\ >\ max(1,q)$ } value approximates a depth-first search. A high $q$ \todo{samme fejl som sidste todo} value approximates the behavior of a breath-first search. The motivation of these tunable parameters is the observation that real-world networks can be structured in many ways. The randomness and the tunable $p$ and $q$ values accounts for different types of networks, by allowing control of the exploration method.

The main idea of node2vec is to start a random walk at every node in the graph. This gives us a list of walks. We now want to find feature representations that maximizes the likelihood of preserving the network neighborhoods of the graph. These feature representations are essentially learned by observing each random walk as a sentence. Each node in the walk is a word in the sentence. This observation allows using word embedding algorithms on the sentences to map words to vectors of real numbers. The node2vec reference implementation uses word2vec for word embedding. Intuitively, if two nodes are close two each other in the sentence, they are related. A tunable parameter $k$ specifies this context window. \todo{expand on this context window}

\todo{Vi kan måske lade det overview være sådan, og så forklare de enkelte dele for sig i en section. Jeg ved dog ikke hvor dybt vi skal gå. Vi skal ikke opfinde node2vec igen}

\todo{afsnit om 2 noder -> 1 edge feature vector}

\tikzsetnextfilename{n2vbowtie}
\begin{figure}[tbp]%
  \centering
    \begin{tikzpicture}[node distance = .9ex, auto, remember picture]
    \tikzstyle{n2vcontainer} = [rectangle, draw, inner sep=.5ex, fill=clshade]
    \tikzstyle{n2vc} = [thick, circle, draw, fill=clnode, text width=1.5ex, text centered, minimum height=1.5ex, inner sep=0pt]
    
    % LEFT SIDE (INPUT LAYER)
    \node [n2vc, label=left:{$x_1$}] (c1) {};
    \node [n2vc, below=of c1, label=left:{$x_2$}] (c2) {};
    \node [n2vc, below=of c2, label=left:{$x_3$}] (c3) {};
    \node [n2vc, below=of c3, draw=none, fill=none] (c3-h) {};
    \node [n2vc, below=of c3-h, label=left:{$x_k$}] (c4) {};
    \node at ($(c3)!.5!(c4)$) {$\vphantom{\int^0}\smash[t]{\vdots}$};
    \node [n2vc, below=of c4, draw=none, fill=none] (c4-h) {};
    \node [n2vc, below=of c4-h, draw=none, fill=none] (c4-hh) {};
    \node [n2vc, below=of c4-hh, label=left:{$x_V$}] (c5) {};
    \node at ($(c4)!.5!(c5)$) {$\vphantom{\int^0}\smash[t]{\vdots}$};
    % Gray background box:
    \begin{scope}[on background layer]
      \node [n2vcontainer, fit=(c1)(c5)] (inputcontainer) {};
      \node[above] at (inputcontainer.north) {Input layer};
    \end{scope}
    
    % RIGHT SIDE (OUTPUT LAYER)
    \node [n2vc, right=8.5cm of c1,label=right:{$y_1$}] (o1) {};
    \node [n2vc, below=of o1, label=right:{$y_2$}] (o2) {};
    \node [n2vc, below=of o2, label=right:{$y_3$}] (o3) {};
    \node [n2vc, below=of o3, draw=none, fill=none] (o3-h) {};
    \node [n2vc, below=of o3-h, label=right:{$y_j$}] (o4) {};
    \node at ($(o3)!.5!(o4)$) {$\vphantom{\int^0}\smash[t]{\vdots}$};
    \node [n2vc, below=of o4, draw=none, fill=none] (o4-h) {};
    \node [n2vc, below=of o4-h, draw=none, fill=none] (o4-hh) {};
    \node [n2vc, below=of o4-hh, label=right:{$y_V$}] (o5) {};
    \node at ($(o4)!.5!(o5)$) {$\vphantom{\int^0}\smash[t]{\vdots}$};
    % Gray background box:
    \begin{scope}[on background layer]
      \node [n2vcontainer, fit=(o1)(o5)] (outputcontainer) {};
      \node[above] at (outputcontainer.north) {Output layer};
    \end{scope}
    
    % MIDDLE (HIDDEN LAYER)
    \begin{scope}[xshift=4.4cm, node distance = .7ex]
    %\node [draw] at ($(inputcontainer)!.5!(outputcontainer)$) {
    %  \begin{tikzpicture}[remember picture]
            \node [n2vc, yshift=-1.5em, label=right:{$h_1$}] (h1) {};
            \node [n2vc, below=of h1, label=right:{$h_2$}] (h2) {};
            \node [n2vc, below=of h2, draw=none, fill=none] (h3-h) {};
            \node [n2vc, below=of h3-h, label=right:{$h_i$}] (h4) {};
            \node at ($(h2)!.5!(h4)$) {$\vphantom{\int^0}\smash[t]{\vdots}$};
            \node [n2vc, below=of h4, draw=none, fill=none] (h4-h) {};
            \node [n2vc, below=of h4-h, label=right:{$h_N$}] (h5) {};
            \node at ($(h4)!.5!(h5)$) {$\vphantom{\int^0}\smash[t]{\vdots}$};
            % Gray background box:
            \begin{scope}[on background layer]
              \node [n2vcontainer, fit=(h1)(h5)] (hiddencontainer) {};
              \node[above, yshift=1.5em] at (hiddencontainer.north) {Hidden layer};
            \end{scope}
    %    \end{tikzpicture}
    %};
    \end{scope}
    
    % LINES MAKING THE BOW TIE
    \draw (inputcontainer.north east) -- (hiddencontainer.north west);
    \draw (inputcontainer.south east) -- (hiddencontainer.south west);
    \draw (outputcontainer.north west) -- (hiddencontainer.north east);
    \draw (outputcontainer.south west) -- (hiddencontainer.south east);
    
    % The math stuff with the arrows 1
    \node at ($(inputcontainer)!.5!(hiddencontainer)$) {
      \begin{tikzpicture}[remember picture, node distance = 1ex]
        \draw [line] (0,0) -- (1,.5);
        \draw [line] (0,.5) -- (1,0);
        %\node [fit=(o1)(o5)] (outputcontainer) {};
        \node [below=1em of {(0.5,0)}] {$\mathbf{W}_{V\times N}=\{w_{ki}\}$};
      \end{tikzpicture}
    };
    
    % The math stuff with the arrows 2
    \node at ($(hiddencontainer)!.5!(outputcontainer)$) {
      \begin{tikzpicture}[remember picture, node distance = 1ex]
        \draw [line] (0,0) -- (1,.5);
        \draw [line] (0,.5) -- (1,0);
        %\node [fit=(o1)(o5)] (outputcontainer) {};
        \node [below=1em of {(0.5,0)}] {$\mathbf{W'}_{V\times N}=\{w'_{ij}\}$};
      \end{tikzpicture}
    };
    
  \end{tikzpicture}
  
\caption[short desc]{Text}%
\label{fig:n2v-figure}%
\end{figure}


\subsubsection{Hyperparameter Optimization}
As described above, node2vec has many tuneable parameters. To find the parameters that give the best set of features, we do a hyperparameter optimization pass. We first specify the parameter space that should be searched in. As this is a large space, it is not feasible to exhaust all possibilities. We therefore use the tool Spearmint~\cite{snoek2012practical} which performs bayesian optimization by maximizing the expected improvement to efficiently search for parameters that will minimize an objective function. The objective function is $1 - \text{f-score}$, where the f-score is found by running a classifier on the node2vec model learned from the parameters. By keeping the classifier and its parameters constant, we can find the best set of parameters by finding the lowest objective function value.

To speed each iteration of the optimization up, we limit the set of start vertices to featured articles. The hyperparameter optimization is entirely autonomous and the process should be able to find good p and q values on its own. This means that one can learn on graph networks without knowing the underlying graph structure.

\todo{Vis bedste parametre, og kommenter på dem}