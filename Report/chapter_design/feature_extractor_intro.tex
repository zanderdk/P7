The purpose of the feature extractor component, is to generate a feature representation of a candidate pair, such that the classifier component can decide whether to suggest





The next component in the main pipeline is the feature extractor. This component generates the feature representation of a candidate pair as described by the function $f$ in \cref{sec:ml_def}, which can then be used for classification.

The feature extractor returns a single feature vector, representing the link between a given source and target article pair. As we need to classify both existing and non-existing links, it must be possible to extract feature vectors in both cases. Therefore the feature vector for a link is constructed by combining the feature vectors of the source and target articles.

As mentioned in \cref{sec:feature_generation} we aim to use feature learning, specifically network embedding, to learn these feature representations. While there are several different ways to approach network embedding, we found node2vec~\cite{node2vec} to be suitable. The main advantage of node2vec is increased flexibility compared to other well-known approaches due to tunable hyperparameters~\cite{node2vec}, which allows us to experiment with different neighborhood exploration methods Additionally, node2vec offers highly competitive performance, and performs well for large networks~\cite{node2vec}.

This section describes how node2vec is used to generate the feature representation, along with the process of optimizing the models hyperparameters.