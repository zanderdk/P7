\chapter{Introduction}\label{ch:introduction}
One of the ideas upon which The World Wide Web is built is \emph{creating a space in which anything could be linked to anything}~\cite[ch.~1, p.~4]{Weaving-the-web}. Hyperlinking can thus be seen as a vital component of the World Wide Web and even a critical component to its success. Hyperlinking permits websites to be easily navigationable, enabling users to quickly find relevant or in-depth information. From social media messages to publicly available databases of technical data, the World Wide Web provides access to an unprecedented amount of information, which is made readily available through easy navigationable hyperlinks.

A prominent example of a website combining both hyperlinking and access to information is Wikipedia. Wikipedia is a free, online encyclopedia built on the principle that anyone can edit its articles. As of October 2016, the English version of Wikipedia consists of more than 5.2 million articles~\cite{wiki-about} and has about 150.000 edits per day~\cite{wiki-num-edits}, as well as about 800--900 new articles per day~\cite{wmcharts}. Considering the community based approach and the large body of work on Wikipedia, quality control is a rather daunting task for any single human editor; no editor can keep up with the vast number of changes and number of new articles and their relationship with other old or new articles. Therefore, bots run through the articles carrying out repetitive and mundane tasks, ranging from notifying editors of mismatched parentheses to vandalism detection or copyright violation on new pages~\cite{wiki-bots}. However since the bots will not fix all possible errors, the primary solution is to have some trust in the individual author~\cite{wiki-editor-guidelines}.

An important issue that editors and authors have to deal with, is the matter of linking articles to each other. In an attempt to ensure the good quality of this interlinking of articles, Wikipedia has a set of guidelines regarding linking between articles~\cite{wiki-manual-of-style-overlinking}. One of the concerns that these guidelines mention is the problem of \emph{overlinking}. In most contexts overlinking is, among other things, linking to everyday words understood by most readers, major geographic locations, common units of measurement, and dates. In addition, a link should generally appear only once in an article. Since links tend to compete for attention~\cite{hyperlink-structure-using-logs}, the guidelines suggest that you should only link to articles that will help a reader to better understand the article being linked from. Because of this, it is not enough to check whether a given article is being mentioned, but semantic and contextual analysis is also required. While many cases are easily decidable for a human author, creating a deterministic approach for computers is a non-trivial task.

\section{Problem Statement}\label{sec:problem_statement}

We believe that Wikipedia provides humanity with a great source of free knowledge, which is a cause we find admirable. We also believe that the existence of a system that can help improve the quality of Wikipedias articles, will benefit Wikipedia and ultimately its users. Therefore, for this project, we wish to explore the problems described above by trying to give an answer to the following question:

\begin{formal}
How can a system be developed that can supply reasonable suggestions of links that should be added to Wikipedia articles?
\end{formal}

The linking problem, whether two articles should be linked, is not an easy problem to solve deterministically, as it is inherently subjective; there is no definitive answer to this problem. The focus of our approach will therefore be on article link suggestions, instead of definitive answers. %This gives us the property that lower confident results can also be included in the results presented by our approach.

We hope that an answer to this question will not only be applicable in this project, but might provide us with an insight into an underlying general problem.

\section{Report Organization}
In this report we propose a solution for the above problem. The proposal is based on an analysis of different solution ideas and techniques. We also cover the design and implementation process of the realization of our solution. We then cover the evaluation of the developed system, and finally conclude on the project as a whole. \todo{Expand this}

% The only good is knowledge and the only evil is ignorance. - Socrates (469 BC - 399 BC)
% All men by nature desires knowledge. - Aristotle (384 BC - 322 BC), Metaphysic






