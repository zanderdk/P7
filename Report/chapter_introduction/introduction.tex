\chapter{Introduction}\label{ch:introduction}
The World Wide Web is built on the idea of \emph{``creating a space in which anything could be linked to anything''}~\cite[p.~4]{Weaving-the-web}. Hyperlinking can thus be seen as a vital component of the World Wide Web and even a critical component to its success. Hyperlinking permits easy navigation on websites, enabling users to quickly find relevant or in-depth information. From social media messages to publicly available databases of technical data, the World Wide Web provides access to an unprecedented amount of information, which is made readily available through easily navigationable hyperlinks.

Wikipedia is a prominent example of a website where hyperlinking is an important aspect. The free, online encyclopedia is built on the principle that anyone can edit its articles. As of October 2016, the English version of Wikipedia consists of more than 5.2 million articles~\cite{wiki-about} and has about \num{150000} edits per day~\cite{wiki-num-edits}, as well as about 800 new articles per day~\cite{wmcharts}. With this scale of information, hyperlinks are essential in connecting knowledge scattered over the many articles. Considering the community based approach and the large body of work on Wikipedia, quality control is a rather daunting task for any single human editor; no editor can keep up with the vast number of changes and number of new articles and their relationship with other old or new articles. Therefore, many mundane and repetitive tasks are performed by bots, such as notifying editors of mismatched parentheses, vandalism detection, or copyright violation on new pages~\cite{wiki-bots}. Bots are, however, not able to autonomously correct all possible errors, and as such manual verification performed by the individual author is still needed~\cite{wiki-bot-policy}.

Linking articles to each other is an important issue editors and authors have to handle. In an attempt to ensure high quality of this article interlinking, Wikipedia has a set of guidelines regarding linking between articles~\cite{wiki-manual-of-style-overlinking}. One of the concerns that these guidelines mention is the problem of \emph{overlinking}. In most contexts overlinking is, for example, linking to everyday words understood by most readers, major geographic locations, common units of measurement, and dates. In addition, a link should generally appear only once in an article. Since links tend to compete for attention~\cite{hyperlink-structure-using-logs}, the guidelines suggest that you should only link to articles that will improve understandability of the article being linked from. Because of this, it is not enough to check whether a given article is being mentioned, but semantic and contextual analysis is also required. While many cases are easily decidable for a human author, it might be more difficult for a computer.

\chapter{Related Work}\label{sec:related_work}

Wikipedia is the subject of multiple papers and research projects~\cite{wiki-research-newsletter}, and so there is much material to gain inspiration from. But before we can draw inspiration from solutions to related problems, we must consider what our problem is. As with almost all problems, this problem can be perceived from multiple angles, and modeled by just as many. Here we will consider some general approaches.

\section{Content-Based Analysis}\label{related_semantic_contextual}

The linking problem can be seen as a problem of deducing content-based relations. Solutions to this problem usually involve a degree of textual analysis in order to determine whether a piece of text refers to another subject. There are usually two parts to this problem, which are each given a different level of importance, depending on the solution:

\begin{description}
  \item[Syntactic Recognition] This part finds syntactic references to some subject. One approach is to find keywords or shingles using an n-gram technique and matching those to subjects, as seen in~\cite{mihalcea2007wikify}.
  
  \item[Semantic Recognition] Compared to syntactic recognition, this part combats possible syntactic ambiguity. An example here is to determine whether the syntactic reference to a tree semantically refers to a data structure or a type of plant. One of many ways of approaching semantic recognition is to train a classifier, as seen in~\cite{milne2008learning}. Here they attempt to determine the semantic reference by training on metrics for commonness, relatedness, and context quality. In short, commonness is the probability distribution of references, relatedness is a measure of similarity between the referrer and the possible referenced subject, and context quality is a measure of whether a given term is usually linked. The example given in~\cite{milne2008learning} is the English grammatical article \enquote{the}, which is used often, but rarely links to the subject article.
\end{description}



\section{Structural Analysis}\label{related_structural_analysis}

Another way of viewing the linking problem is as missing structural connections in a dataset. With this approach, solutions attempt to analyze a structure in order to gain insight into possible patterns that influences linking. For Wikipedia the most relevant structure to consider is a graph structure with articles as vertices. There are different ways to model edges in the graph. Examples of this include~\cite{hyperlink-structure-using-logs} and~\cite{west2015mining} where the authors create edges from the navigation of Wikipedia's visitors, based on the belief that an optimal linking structure can be deduced from user behavior. Of course anything that can be considered a relation between any two articles, can be the basis for a set of edges, as explored in~\cite{lu2011link}.

Regardless of the approach, a metric for a good link must be considered. We have already introduced Wikipedia guidelines on linking~\cite{wiki-editor-guidelines}, which is Wikipedia's own view of a good set of metrics. However, even though they clearly hold authority on the matter, there exists alternatives worth considering.

In~\cite{hyperlink-structure-using-logs} and~\cite{west2015mining} they consider good links to be ones that are in use by visitors, and as such they rank their results based on server logs. This technique rates the amount of clicks a given link receives out of the total visitors on the page, based on the idea that the most useful links get the most clicks, and that all links should at least get some clicks. An argument in favor of this technique of finding a clickthrough rate is that every link can be objectively measured and compared. However, a drawback with the technique is that links become competitors, and will be optimized towards obtaining the most traffic individually, instead of being optimized as a part of a collective. This seems to change the focus of interlinking. Rather than providing links to interesting articles to read in relation to a source article, the links are created to draw the user's attention to articles.

\section{Machine Learning}\label{related_machine_learning}

An underlying method that occurs frequently in related work with a content-based approach is machine learning. In~\cite{mihalcea2007wikify} and~\cite{milne2008learning} they both employ a naive Bayes classifier, with the latter also testing different C4.5s and support vector machines, for the purpose of classifying matters of ambiguity and relatedness. Both~\cite{hyperlink-structure-using-logs} and~\cite{west2015mining} reduces the problem of ambiguity by employing the clickthrough rate as their primary measure, since user navigation paths usually contains related articles. \todo{Also explain their methods}

However, an approach we have not seen as frequent, is the use of machine learning to improve Wikipedia links in a structural setting. But using machine learning for predicting links in graphs is not an unexplored concept, as~\cite{tang2015line} and~\cite{al2006link} are two examples of. Due to this, we wish to explore the idea of using machine learning techniques to improve linking on Wikipedia through a graph structure approach.


%Wikify! Linking Documents to Encyclopedic Knowledge 		mihalcea2007wikify
%Learning to Link with Wikipedia 							milne2008learning
%Hyperlink Structure Using Server logs						hyperlink-structure-using-logs	
%Human Navigation Traces									west2015mining 				
%Prediction in Complex Networks 							lu2011link
%Link Prediction using Supervised Learning 					al2006link
%LINE algo													tang2015line

\section{Problem Statement}\label{sec:problem_statement}
We believe that the existence of a system that helps editors improve the quality of the links in Wikipedia articles, will benefit Wikipedia and thereby ultimately its readers.

The use of machine learning for predicting links in a structural setting is not an unexplored concept, as~\cite{tang2015line} and~\cite{al2006link} are two examples of. It is however, an approach we have not seen used to a large extent in related work concerning Wikipedia. Due to this, we wish to explore the idea of using machine learning techniques to improve linking on Wikipedia through a graph-based approach.

Specifically, the problem we will attempt to solve in this project can be formulated as such:

\begin{formal}
How can a system be developed that supports Wikipedia editors by suggesting potential article links using machine learning on a graph structure?
\end{formal}

In order to better approach this problem, we identify a list of subproblems that enables us to focus on the different aspects involved:

\begin{itemize}
  \item How can Wikipedia articles be modelled as a graph, such that relevant information on linking is included?
  \item How can machine learning be used to suggest potential articles links, that would improve article quality?
  \item How can Wikipedia editors be presented with suggestions of article links that could be included?
\end{itemize}

\todo{afslut problem statement med en enkelt paragrah, se udkommentering for eksempel på hvad der ikke skal stå}

%The answer to the problem of whether two articles should be linked is inherently subjective --- there is no definitive answer to this problem. Therefore, the focus of our project will be on article link suggestions, instead of definitive answers.

\section{Report Organization}
In this report we propose a solution for the above problem, based on an analysis of different solution ideas and techniques. We also cover the design and implementation process of our solution. We then cover the evaluation of the developed system, and finally conclude on the project as a whole. \todo{Expand this}

% The only good is knowledge and the only evil is ignorance. - Socrates (469 BC - 399 BC)
% All men by nature desires knowledge. - Aristotle (384 BC - 322 BC), Metaphysic

