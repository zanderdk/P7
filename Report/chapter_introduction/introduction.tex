\chapter{Introduction}
The World Wide Web is built upon the idea of \emph{creating a space in which anything could be linked to anything} \cite[ch.~1, p.~4]{Weaving-the-web}. Thus, hyperlinking can be seen as a vital component of the World Wide Web, and even a critical component to its success. Hyperlinking allows websites to be easily navigationable, enabling users to quickly find relevant or in-depth information.

Considering a large website, finding pairs of pages that should be linked may be a difficult task, especially for humans. Hyperlinks need to be relevant and add value for a user browsing the website.

A wide array of research that tries to find links.

\todo{motivation and users}

\section{Problem Statement}

In this project, we will try to answer the following question:

\begin{formal}
How can an agent be developed which given a Wikipedia article can suggest link candidates to other Wikipedia articles?
\end{formal}

%How can we develop an agent that, given a Wikipedia article, can suggest link candidates to other relevant Wikipedia articles?

% Goals and study regulation
The main goal of this semester project is, of course, to conform to the requirements of the study regulation. The study regulation clearly states, that we should be able to demonstrate knowledge about and have the ability to develop an internet application, agent or service.

Expanding on the problem statement, this project will consist of two parts:
\begin{enumerate*}[label=(\roman*)]
  \item Identify relevant missing links; and
  \item present these to a user
\end{enumerate*}.

\todo{Describe overall idea in broad terms}

\section{Project Goals}
\todo{Success criteria (to be evaluated in the conclusion)}

\section{Report Organization}
This report describes and documents the design, implementation, evaluation and development method of the system described in this chapter. Immediately following this introduction, \chapterref{chap:analysis} ... \dummy ... \chapterref{chap:devmethodreflection} contains a discussion and reflection of the development process used throughout this project. \chapterref{chap:conclusion} contains the conclusion.