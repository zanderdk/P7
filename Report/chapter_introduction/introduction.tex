\chapter{Introduction}
The World Wide Web is built upon the idea of \emph{creating a space in which anything could be linked to anything} \todo{insert cite: Weaving the Web, chapter 1 by Tim Berners-Lee}. Thus, hyperlinking can be seen as a vital component of the World Wide Web, and even a critical component to its success. Hyperlinking allows websites to be easily navigationable, enabling users to quickly find relevant or in-depth information.

Considering a large website, finding pairs of pages that should be linked may be a difficult task, especially for humans. Hyperlinks need to be relevant and add value for a user browsing the website. Some websites allow many different users to edit the content of pages, including the hyperlinks. This further complicates the task of maintaining a good link structure, as the users might differ in opinon on the relevance of certain hyperlinks\todo{add ref to p51-ellis...}.

Wikipedia is a good example of a website where maintaining the link structure is both important and difficult. Properly linking between articles is important, as it facilitates easy acces to other relevant articles, that might contain necessary background knowledge. Articles on Wikipedia are authored and edited by many different users, who have to decide what articles should be linked. In an attempt to ensure a good link structure, Wikipedia has a set of guidelines regarding linking between articles\todo{ref wiki guidelines}. It is however still difficult for editors to decide when an article is relevant enough to warrant linking.


\todo{motivation and users}

\section{Problem Statement}
How can we develop an agent that, given a Wikipedia article, can suggest link candidates to other relevant Wikipedia articles?\todo{Refine initial idea}

\section{Project Goals}
\todo{Success criteria (to be evaluated in the conclusion)}

\section{Compliance with Study Regulation}
\todo{Maybe a description of how the project fits within the study regulation}

\section{Report Organization}
This report describes and documents the design, implementation, evaluation and development method of the system described in this chapter. Immediately following this introduction, \chapterref{chap:analysis} ... \dummy ... \chapterref{chap:devmethodreflection} contains a discussion and reflection of the development process used throughout this project.