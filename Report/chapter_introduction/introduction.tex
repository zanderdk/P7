\chapter{Introduction}
%\todo{Kasper:Wikipedia bliver lidt introduceret som 'et tilfældigt eksempel', så derfra er det mindre åbentlyst hvorfor vi vælger at gå videre med wiki. Det er som om vi kunne have taget en tilfældig 'large' side. Jeg vil have sagt: Information er nice > Wikipedia har information (og er ikke onde) > Hvis vi kan hjælpe wiki er det nice > vi kan lave links}
One of the ideas upon which The World Wide Web is built is \emph{creating a space in which anything could be linked to anything}~\cite[ch.~1, p.~4]{Weaving-the-web}. Hyperlinking can thus be seen as a vital component of the World Wide Web and even a critical component to its success. Hyperlinking permits websites to be easily navigationable, enabling users to quickly find relevant or in-depth information. From social media messages to publicly available databases of technical data, the World Wide Web provides access to an unprecedented amount of information, which is made readily available through easy navigationable hyperlinks.

A prominent example of a website combining both hyperlinking and access to information is Wikipedia. Wikipedia is a free, online encyclopedia built on the principle that anyone can edit its articles. As of October 2016 the English version of Wikipedia consists of more than 5.2 million articles~\cite{wiki-about} and has about 150.000 edits per day~\cite{wiki-num-edits}, as well as about 800--900 new articles per day~\cite{wmcharts}. Considering the community based approach and the large body of work on Wikipedia, quality control is a rather daunting task for any single human editor; no editor can keep up with the vast number of changes and number of new articles and their relationship with other old or new articles. Therefore, bots run through the site carrying out repetitive and mundane tasks, ranging from vandalism detection to notifying editors of mismatched parentheses~\cite{wiki-bots}, along with a trust in the individual author.

An important issue that editors and authors have to deal with, is the matter of linking articles to eachother. In an attempt to ensure the good quality of this interlinking of articles, Wikipedia has a set of guidelines regarding linking between articles~\cite{wiki-manual-of-style-overlinking}. 
%However, these guidelines are somewhat vague and merely provide examples of do's and don'ts. % Yes, this is a widely accepted way of spelling do's and don'ts
%This makes it particularly hard to automatize with a bot. On the other hand, this area would likely benefit from automation, since the size of Wikipedia makes it difficult to do manually.
One of the concerns that these guidelines mention is the problem of overlinking. A simple example of overlinking, would be to include a link to the article about the word \enquote{the}, everytime it is used in a sentence. Since links tend to compete for attention~\cite{hyperlink-structure-using-logs}, the guidelines suggest that you should only link to articles that will help a reader to better understand the article being linked from. Because of this, is it not enough to check whether a given article is being mentioned, but some semantic and contextual analysis is also required. While many cases are easily decidable for a human author, creating a deterministic approach for computers is a non-trivial task.

However, Wikipedia maintains a collection of articles that are deemed outstanding examples and are considered to be the best Wikipedia has to offer. These are called \emph{featured articles}~\cite{wiki-featured-articles} and there are almost 5000 of these. These articles follow the Wikipedia Policies and Guidelines~\cite{wiki-editor-guidelines}, which also means they follow the guidelines for linking. We can therefore assume that they link appropriately.


\section{Problem Statement}

We wish to explore the problems described above by trying to give an answer to the following question:

\begin{formal}
How can an agent be developed which given a Wikipedia article can suggest link candidates to other Wikipedia articles?
\end{formal}
%How can a system be developed that can supply reasonable suggestions of links that should be added to Wikipedia articles?
%How can we develop an agent that, given a Wikipedia article, can suggest link candidates to other relevant Wikipedia articles?

Expanding on the problem statement, this project will consist of two parts:
\begin{enumerate*}[label=(\roman*)]
  \item Identify relevant missing links; and
  \item present these to a user
\end{enumerate*}.
The technically interesting part of this project mainly lies in the first part. Identifying relevant missing links is not a trivial task, and we will spend most of our time tackling this problem. Not entirely dismissing a presentation to a user, but it will not be our main focus.

% Goals and study regulation
An important goal of this semester project is, of course, to conform to the requirements of the study regulation. The study regulation clearly states, that we should be able to demonstrate knowledge about and have the ability to develop an internet application, agent or service. And GUI! \todo{Tør vi overhovedet svare på om vi opfylder studieordningen?}

\todo{users}

\section{Project Goals}
\todo{Success criteria (to be evaluated in the conclusion)}

\section{Report Organization}
This report describes and documents the design, implementation, evaluation and development method of the system described in this chapter. Immediately following this introduction, \chapterref{chap:analysis} ... \dummy ... \chapterref{chap:devmethodreflection} contains a discussion and reflection of the development process used throughout this project. \chapterref{chap:conclusion} contains the conclusion.