\section{Related Work}\label{sec:related_work}
The problem of linking articles on Wikipedia, i.e. the linking problem, has been the subject of previous research, and so there is much material to gain inspiration from. However, before we draw inspiration from solutions to related problems, we must first consider how to approach our problem. As with almost all problems, this problem can be perceived from multiple angles, and modeled in many different ways. Here we will consider some general approaches and examine how they previously have been used for analyzing relationships between hypertext documents such as Wikipedia articles.
%\todo{Omskriv sidste 3 sætninger. Vi ved godt hvad problemet er her, vi kigger på relateret arbejde for at få ideer til løsningsmuligheder.}

\subsection{Content-Based Analysis}\label{related_semantic_contextual}
The linking problem can be seen as a problem of deducing content-based relations. Solutions to this problem usually involve a degree of textual analysis in order to determine whether a piece of text refers to another subject. There are usually two parts to this problem, which are each given a different level of importance, depending on the solution:

\begin{description}
  \item[Syntactic Recognition] This part finds syntactic references to some subject. One approach~\cite{mihalcea2007wikify} is to find keywords or shingles using an n-gram technique and matching those to subjects.

  \item[Semantic Recognition] Compared to syntactic recognition, this part combats possible syntactic ambiguity. An example here is to determine whether the word \emph{tree} semantically refers to the data structure or the plant. One of many ways of approaching semantic recognition is to train a classifier, as seen in~\cite{milne2008learning}. Here they attempt to determine the semantic reference by training on metrics for commonness, relatedness, and context quality. In short, commonness is the probability distribution of references, relatedness is a measure of similarity between the referrer and the possible referenced subject, and context quality is a measure of whether a given term is usually linked. The example given in~\cite{milne2008learning} is the English grammatical article \enquote{the}, which is used often, but rarely links to the subject article.
\end{description}


\subsection{Structure-Based Analysis}\label{related_structural_analysis}
Another way of viewing the linking problem is as missing structural connections in a dataset. With this approach, solutions attempt to analyze a structure in order to gain insight into possible patterns that influences linking. For Wikipedia the most relevant structure to consider is a graph structure with articles as nodes. There are different ways to model edges in the graph. Examples of this include~\cite{hyperlink-structure-using-logs} where the authors create edges from the navigation of Wikipedia's visitors, based on the belief that an optimal linking structure can be deduced from user behavior. Essentially, anything that can be considered a relation between any two articles, can be the basis for a set of edges, as explored in~\cite{lu2011link}.

Regardless of the approach, a metric for an appropriate link must be considered. We have already introduced Wikipedia's guidelines on linking~\cite{wiki-editor-guidelines}, which is Wikipedia's official view of a good set of metrics. However, even though they clearly hold authority on the matter, there exists alternatives worth considering.

In~\cite{hyperlink-structure-using-logs} they consider appropriate links to be ones that are in use by visitors, and as such they rank their results based on server logs. \todo{Skal forklares bedre} This technique rates the amount of clicks a given link receives out of the total visitors on the article, called the clickthrough rate, based on the idea that the most useful links get the most clicks, and that links are only useful if they are clicked. An argument in favor of using clickthrough rate is that every link can be objectively measured and compared.

One has to be aware of the competing links problem, where too many links are inserted on a popular article. This is solved in \cite{hyperlink-structure-using-logs} by optimizing an objective function with a diminishing returns property such that new links will naturally be spread on a larger number of articles.

\subsection{Machine Learning}\label{related_machine_learning}

Machine learning is often used as an underlying method in related work with a content-based approach. In~\cite{mihalcea2007wikify,milne2008learning} they employ a naive Bayes classifier, with the latter also testing support vector machines and different decision trees built using the C4.5 algorithm, for the purpose of classifying matters of ambiguity and relatedness. \cite{hyperlink-structure-using-logs} models user behavior as objective functions. The best performing objective functions are designed to have diminishing returns such that new links are spread evenly across Wikipedia. To return a list of link suggestions, they propose a greedy algorithm that seeks to maximize the designed objective functions, ultimately optimizing for the expected number of clicks a new link will receive.

Using machine learning for predicting links in a structural setting is not an unexplored concept, as~\cite{tang2015line} and~\cite{al2006link} are two examples of. It is however, an approach we have not seen as frequently. Due to this, we wish to explore the idea of using machine learning techniques to improve linking on Wikipedia through a structure-based approach.


%Wikify! Linking Documents to Encyclopedic Knowledge 		mihalcea2007wikify
%Learning to Link with Wikipedia 							milne2008learning
%Hyperlink Structure Using Server logs						hyperlink-structure-using-logs
%Human Navigation Traces									west2015mining
%Prediction in Complex Networks 							lu2011link
%Link Prediction using Supervised Learning 					al2006link
%LINE algo													tang2015line