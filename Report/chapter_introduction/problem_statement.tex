\section{Problem Statement}\label{sec:problem_statement}
We believe that a system which helps editors improve the quality of the links in Wikipedia articles, will benefit Wikipedia and ultimately its readers.

The use of machine learning for predicting links in a structural setting is not an unexplored concept~\cite{tang2015line,al2006link}. It is however an approach we have not seen used extensively in related work concerning Wikipedia. Due to this, we wish to explore the idea of using machine learning techniques to improve linking on Wikipedia through a graph-based approach.

The problem we attempt to solve in this project can be formulated as such:
\newcommand{\problemstatement}{
\begin{formal}
How can a software solution be developed that supports Wikipedia editors by suggesting potential article links using machine learning on a graph structure?
\end{formal}}

\problemstatement

In order to better approach this problem, we define a number of more concrete subproblems that enable us to focus on the different aspects involved:

\newcommand{\subproblemone}{How can Wikipedia articles be modeled as a graph, such that relevant information for the linking problem is captured?}
\newcommand{\subproblemtwo}{How can machine learning be used to suggest potential links, that would improve article quality?}
\newcommand{\subproblemthree}{How can editors be presented with suggestions of links that could be added to Wikipedia articles?}

\newcommand{\subproblems}{
\begin{itemize}
  \item \subproblemone
  \item \subproblemtwo
  \item \subproblemthree
\end{itemize}}

\subproblems

These questions will facilitate development of a solution, and will also be used as a basis to conclude the project.

%The answer to the problem of whether two articles should be linked is inherently subjective --- there is no definitive answer to this problem. Therefore, the focus of our project will be on article link suggestions, instead of definitive answers.
