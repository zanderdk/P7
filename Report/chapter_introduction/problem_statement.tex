\section{Problem Statement}\label{sec:problem_statement}
We believe that the existence of a system that can help improve the quality of Wikipedias articles, will benefit Wikipedia and ultimately its users. In this project, we wish to explore the problems described above by trying to give an answer to the following question:

\begin{formal}
How can a system be developed that can supply reasonable suggestions of links, that should be added to Wikipedia articles?
\end{formal}

This question can further be split into subproblems, that we wish to answer:
%\todo{Some VERY work-in-progress subquestions for inspiration only:}
\begin{itemize}
  \item What distinguishes article pairs that should be linked and those that should not?
  \item How can this distinction between two classes be implemented?
  \item What is a reasonable method of presenting these to a user?
  \item How can scalability of the system be achieved?\todo{Kommer meget pludseligt}
\end{itemize}
%\todo{Skriv nogle mere specifikke krav, ala: der skal laves en brugergrænseflade til vores fantasikunder}

The answer to the problem of whether two articles should be linked is inherently subjective --- there is no definitive answer to this problem. Therefore, the focus of our project will be on article link suggestions, instead of definitive answers.

%In order to gain a more detailed understanding of the stated problem, we have constructed a list of subproblems, that together can be seen as equivalent as the original problem:

%\begin{itemize}
%   \item How can a system do derterministic suggestions to a subjective problem?
%   \item How can a metric for appropriate linking be constructed?
%\end{itemize}

%\section{Title different from requirements} % (fold)
%\label{sec:title_different_from_requirements}




% section title_different_from_requirements (end)