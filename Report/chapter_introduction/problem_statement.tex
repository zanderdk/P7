\section{Problem Statement}\label{sec:problem_statement}
We believe that the existence of a system that helps editors improve the quality of the links in Wikipedia articles, will benefit Wikipedia and thereby ultimately its readers.

The use of machine learning for predicting links in a structural setting is not an unexplored concept, as~\cite{tang2015line} and~\cite{al2006link} are two examples of. It is however an approach we have not seen used to a large extent in related work concerning Wikipedia. Due to this, we wish to explore the idea of using machine learning techniques to improve linking on Wikipedia through a graph-based approach.

The problem we will attempt to solve in this project can be formulated as such:
\vspace{1ex} % Please do not remove this.
\begin{formal}
How can a system\todo{software solution} be developed that supports Wikipedia editors by suggesting potential article links using machine learning on a graph structure?
\end{formal}

In order to approach this problem, we identify a list of subproblems that enables us to focus on the different aspects involved:

\newcommand{\subproblems}{
\begin{itemize}
  \item How can Wikipedia articles be modeled as a graph, such that information relevant for the linking problem is captured?
  \item How can machine learning be used to suggest potential articles links, that would improve article quality?
  \item How can editors be presented with suggestions of links that could be added to Wikipedia articles?
\end{itemize}}

\subproblems

These questions will simplify the finding of a solution, and towards the end of the project we will conclude on the found solution based on these subproblems.

%The answer to the problem of whether two articles should be linked is inherently subjective --- there is no definitive answer to this problem. Therefore, the focus of our project will be on article link suggestions, instead of definitive answers.
