\chapter{Analysis}\label{chap:idea}
In this chapter we devise an idea to solve the problem of finding missing links on Wikipedia. This involves an analysis of the problem domain and of the available methods we can utilize and apply.

\begin{chapterorganization}
  \item In \sectionref{sec:analconsiderations} we summarize why this problem is non-trivial.
 
\end{chapterorganization}

\section{Considerations}\label{sec:analconsiderations}
The main problem making this project non-trivial is the fact that Wikipedias guidelines for linking~\cite{wiki-manual-of-style-overlinking} are a matter of judgment. 

\todo{Dette kan bruges som motivation for at bruge MI / learn-by-examples:} 
There is no clear definition of what a ``good'' or ``bad'' link is. However, we have plenty of examples of what a ``good'' link is.
Wikipedia maintains a collection of articles that are deemed outstanding examples and are considered to be the best Wikipedia has to offer. These are called \emph{featured articles}~\cite{wiki-featured-articles} and there are almost 5000 of these. These articles follow the Wikipedia Policies and Guidelines~\cite{wiki-editor-guidelines}, which also means they follow the guidelines for linking. We can therefore assume that they link appropriately.

\todo{needs to be reviewed so that it does not repeat too much from the previous introduction section. There are also some parts that maybe can be moved up to previous sections}
The content of Wikipedia is produced and maintained via crowdsourcing. That means that a large number of users collaboratively add and review content. As described in the previous section, it is non-trivial to find the correct articles to link when editing a Wikipedia article.

We will in this report investigate whether we can use computers to automate finding missing links on Wikipedia articles, in order to allow the user to focus on other parts of article authoring. The linking problem, namely whether two articles should be linked, is not easy to define in an objective manner, as it is inherently subjective; there is no definitive answer to this problem. The focus on our approach will therefore be on article link suggestions, instead of definitive answers. This gives us the nice property that lower confident results can also be included in the results presented by our approach.

To find missing links, our idea is to mimic how the contributors on Wikipedia link articles together. In other words, we want to learn how to link articles together by looking at how links are currently inserted/not inserted. This is is essentially the idea of machine learning, which we will be the main part of our solution idea. The following sections will expand on this idea in-depth.

\section{Machine Learning Task}\label{sec:machine_learning_task}
%With a graph defined,
In this section we introduce a machine learning task based on the problem we are trying to solve. In \cref{sec:problem_statement} we describe the problem of suggesting links between Wikipedia articles. To solve this using machine learning we formulate it as a binary classification problem of determining whether two Wikipedia articles should be linked or not.
%Specifically deciding between the classes \textit{linked} and \text{not linked}.

For the reasons described in \cref{sec:choice_of_graph} we assume featured articles to have appropriate linking. This makes links from featured articles suitable as positive training examples when learning where a link should be present. Similarly articles that are not linked from a featured article can be used as negative training examples.
Under this assumption a large amount of labeled training data can be generated and we therefore limit the scope of the problem to only consider supervised learning approaches. This delimits the problem to binary classification problem in a supervised machine learning context. \todo{should we explain why labeled training data makes supervised learning a good fit?}


%By constructing the training data only from $P$ and $N$, all training pairs can be labeled as either linked or not linked. 

\subsection{Definitions}
%We define a set of positive labeled training pairs $P \subset V \times V$ as $\Set{(a,b)\ |\ a\text{ is featured}\ \wedge\ a \Rightarrow b}$. Likewise we define a set of negative labeled training pairs $N \subset V \times V$ as $\set{(a,b)\ |\ a\text{ is featured}\ \wedge\ a \not\Rightarrow b}$.
We define a set of positive labeled training pairs $P \subset V \times V$, as well as a set of negative labeled training pairs $N \subset V \times V$:

$$P = \Set{(a,b)\ |\ a\text{ is featured}\ \wedge\ a \Rightarrow b},$$
$$N = \Set{(a,b)\ |\ a\text{ is featured}\ \wedge\ a \not\Rightarrow b}$$

%\subsubsection{Binary Classification}

%The problem can be defined as a binary classification:
We formulate the classification task as an optimization problem, maximizing the probability of $(a,b) \in \ \Rightarrow$. Let $f: V\times V \to \mathbb{R}^d$ be the function mapping from article pairs to a vector representation of features where $d$ is the dimension of the vector.

In order to learn predicting the linking class label $y$ of a pair of articles $x$,
%we seek to optimize the following objective function, by maximizing the probability of $x \in \ \Rightarrow$ conditioned on the feature representation $f(x)$, where $x \in V \times V$:
we seek to find values of $y$ that maximizes the objective function $\Pr(y \mid f(x))$, which represents the probability that articles should be linked, where $x \in \ \Rightarrow$ conditioned on the feature representation $f(x)$, and $x \in V \times V$:

\begin{equation}
\label{objective}
\argmax_y\ \Pr(y \mid f(x))
\end{equation}

With $y$ defined as:

\[
    y=
\begin{cases}
    1 & \text{if } x \in \  \Rightarrow\\
    0 & \text{otherwise}
\end{cases}
\]

Here, 1 represents the class of article pairs that should be linked, and 0 the ones that should not.
%$\Pr$ in \cref{objective} represents the probability that a pair of articles should be linked, and is the function we aim to learn.
In order to solve this optimization problem we first need to define $f$ which will be described in the following section.

\subsubsection{Article Features}
\todo{this needs review, and should maybe be moved to another section}

The function $f: V\times V \to \mathbb{R}^d$ maps article pairs to a feature representation in $d$ dimensions, by combining the features of each article pair.

f(a,b) = g(h(a), h(b)) where $h: V \to \mathbb{R}^d$ is a function mapping an article to a feature vector and $g: \mathbb{R}^d \times \mathbb{R}^d \to \mathbb{R}^d$ a binary operation combining a pair of feature vectors.

The feature vector $h(x)$ represents the learned features of article $x$.

%\paragraph{Input:}
%A set of $m$ training examples $(x^j,y^j)$ for $j=1,2..m$, sampled from a distribution $f(E_j)$ of article pair features, with $x^j \in R^n$ being a feature vector and $y^j \in \set{-1,+1}$ a training label. A training label $y^j=+1$ will denote a \textit{positive sample} and $y^j=-1$ a \textit{negative sample}. The i-th feature of a sample $x^j$ defined $x^j_i$ represents a pair of articles and is a combination of article features generated by node2vec~\cite{node2vec} from the training pairs $P$ and $N$.

%\todo{explain/generalize article combination without mentioning node2vec? (binary operation)}
%\todo{beskriv hvilken objective function vi vil minimere/maksimere}
%\todo{input: $x^j$ skal være par af artikler}

%node2vec:
%Given two nodes u and v, and a binary operator o' the feature vectors f(u) and f(v) are combined into the representation g(u,v).

%\paragraph{Output:}
%A function $f: R^n \to \set{-1,+1}$ that classifies additional samples ${x^k}$ sampled from $F$.





\section{Feature Generation}\label{sec:feature_generation}
A main component of any machine learning technique is features. The features should be used to discriminate the data so that a wanted structure appears. Deriving good features from a data set is paramount to the performance of the machine learning algorithm.

Traditionally, features are engineered from a large data set. This means that particular parts of the data set that are deemed characteristic for a given problem are hand-picked by humans. In recent years, feature learning has gained popularity. In short, features are automatically generated by looking at the structure of a data set.

In the following sections we will discuss strengths and weaknesses of each approach to feature generation and how they are used in our project.

\subsection{Feature Engineering}
The premise of feature engineering is that humans are able to identify information in a data set that has a high degree of discrimination power. For example, in a binary classification problem, this means features that are good at separating positive and negative samples. Identifying these features may require domain knowledge as the best features may not be the obvious ones. In general it is a time consuming process to identify suitable features. Typically, many trial and error iterations are needed to derive good features.

We do not have great domain knowledge of Wikipedia, but we have some ideas for features that intuitively should be good.

\begin{description}
    \item[Clickstream] \todo{describe what this is before? link to wikipedia clickstream and maybe alexanders regression paper}.
    \item[Terms similarity] This is a metric that describes the similarity of the most important terms on two given Wikipedia articles. For example, if article A has the terms w1, w2, w3 and article B has w3, w4, a metric could be the number of similar terms, in this case 1. The idea behind the feature is that articles with similar terms are more likely to be linked together.
    \item[Shortest path] Consider Wikipedia as a graph, where the vertices are articles and edges are links between articles. The idea of this feature is that there is a correlation between the shortest path of two articles A, B, and whether there is a link between A and B. 
    \item[Common children/parents] See \cref{fig:children-rel,fig:parent-rel}. The motivation of this feature is that if two articles have common direct predecessors or successors they may be related to such a degree that they should be linked. \todo{Explain the figure.}
\end{description}


\begin{figure}[tbp]%
  \centering
  \begin{minipage}{0.45\textwidth}
    \centering
    \tikzsetnextfilename{parents}
    \begin{tikzpicture}[node distance = 1.7cm, auto]
      \node [node,
              pin={[pin distance=1.2em,pin edge={-stealth, thin}]10:},
              pin={[pin distance=1.2em,pin edge={stealth-, thin}]150:}
              ] (p) {P};           
      \node [node, below left=of p,
              pin={[pin distance=1.2em,pin edge={stealth-, thin}]330:}, 
              pin={[pin distance=1.2em,pin edge={-stealth, thin}]100:},
              pin={[pin distance=1.2em,pin edge={stealth-, thin}]200:}
              ] (a) {A};           
      \node [node, below right=of p,
              pin={[pin distance=1.2em,pin edge={stealth-, thin}]330:}, 
              pin={[pin distance=1.2em,pin edge={-stealth, thin}]40:},
              pin={[pin distance=1.2em,pin edge={stealth-, thin}]85:},
              pin={[pin distance=1.2em,pin edge={-stealth, thin}]200:}
              ] (b) {B};
      \path [line, very thick] (p) -- (a);
      \path [line, very thick] (p) -- (b);
    \end{tikzpicture}
    \caption[short desc]{A nice figure showing a common direct predecessor of A and B}%
    \label{fig:parent-rel}%
  \end{minipage}
  \hfill
  \begin{minipage}{0.45\textwidth}
    \centering
    \tikzsetnextfilename{children}
    \begin{tikzpicture}[node distance = 1.7cm, auto]
      \node [node,
              pin={[pin distance=1.2em,pin edge={-stealth, thin}]350:},
              pin={[pin distance=1.2em,pin edge={stealth-, thin}]210:}
              ] (p) {S};           
      \node [node, above left=of p,
              pin={[pin distance=1.2em,pin edge={stealth-, thin}]15:}, 
              pin={[pin distance=1.2em,pin edge={-stealth, thin}]150:},
              pin={[pin distance=1.2em,pin edge={-stealth, thin}]230:},
              pin={[pin distance=1.2em,pin edge={stealth-, thin}]280:}
              ] (a) {A};           
      \node [node, above right=of p,
              pin={[pin distance=1.2em,pin edge={stealth-, thin}]330:}, 
              pin={[pin distance=1.2em,pin edge={-stealth, thin}]280:},
              pin={[pin distance=1.2em,pin edge={-stealth, thin}]150:}
              ] (b) {B};
      \path [line, very thick] (a) -- (p);
      \path [line, very thick] (b) -- (p);
    \end{tikzpicture}
    \caption[short desc]{A very nice figure showing a common direct successor of A and B}%
    \label{fig:children-rel}%
  \end{minipage}
  

\end{figure}

\subsection{Feature Learning}
The obvious advantage of feature learning is that it obviates the need for feature engineering. Instead of manually deriving discriminating information, features are generated by the structure of the data set. This has the advantage that the need for domain knowledge is reduced, and the feature learning algorithm is able to uncover structural patterns that are hard for humans to discover. As the features are generated, they are difficult to reason about. In other words, there is limited intuitive understanding available that can explain what each individual feature means. As the feature learning process is mostly autonomous, it is possible to mechanically try many combinations of parameters for the algorithm, in order to find the best set of features.

As described in \todo{some section} it was chosen to structure Wikipedia articles and links between them as a graph. Due to this structure in our information we chose to examine feature learning techniques working on graph structures also called network embedding.
We are aware that many more feature learning techniques exists, but we found this type of feature learning the most relevant and chose to delimit to only focusing on network embedding.

\section{Our choice/idea/solution/guess}
Wikipedia contains a big amount of data that represents different types of information. Articles have many different attributes and embeddings of information as well as editor and revision history. Attributes such as categories and article links provides structural information that can be used to describe the connection and relation of things.

Through feature engineering we believe that we can identify specific pieces of information that can be used as machine learning features. Because of our limited domain knowledge these features might not be adequate, but we believe that by utilizing feature learning on this information, useful features can be learned.

It is our intuition that the link-structure of Wikipedia can provide some of the most promising features for predicting missing links. Furthermore we believe that the Wikipedia clickstream will be beneficial because it captures the user behaviour of how links are used.

\paragraph{Plan:} We will be building a clickstream graph of Wikipedia that represents how articles are linked and navigated by users. Using this graph we will use feature learning to learn features from the structure, which will be used to train a classifier for suggesting missing article links.


\todo{other features, methods (text content...)}

\todo{insert clickstream source / motivation}



