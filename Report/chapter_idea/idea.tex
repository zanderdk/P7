\chapter{Analysis}\label{chap:idea}
In this chapter we will devise an idea to solve the problem of finding missing links on Wikipedia. This involves an analysis of the problem domain and of what methods are possible to use.

\begin{chapterorganization}
  \item In \sectionref{sec:analconsiderations} we summarize why this problem is non-trivial.
 
\end{chapterorganization}

\section{Considerations}\label{sec:analconsiderations}
The main problem making this project non-trivial is the fact that Wikipedias guidelines for linking~\cite{wiki-manual-of-style-overlinking} are a matter of judgment. 

\todo{Dette kan bruges som motivation for at bruge MI / learn-by-examples:} 
There is no clear definition of what a ``good'' or ``bad'' link is. However, we have plenty of examples of what a ``good'' link is.
Wikipedia maintains a collection of articles that are deemed outstanding examples and are considered to be the best Wikipedia has to offer. These are called \emph{featured articles}~\cite{wiki-featured-articles} and there are almost 5000 of these. These articles follow the Wikipedia Policies and Guidelines~\cite{wiki-editor-guidelines}, which also means they follow the guidelines for linking. We can therefore assume that they link appropriately.

\todo{needs to be reviewed so that it does not repeat too much from the previous introduction section. There are also some parts that maybe can be moved up to previous sections}
The content of Wikipedia is produced and maintained via crowdsourcing. That means that a large number of users collaboratively add and review content. As described in the previous section, it is non-trivial to find the correct articles to link when editing a Wikipedia article.

We will in this report investigate whether we can use computers to automate finding missing links on Wikipedia articles, in order to allow the user to focus on other parts of article authoring. The linking problem, namely whether two articles should be linked, is not easy to define in an objective manner, as it is inherently subjective; there is no definitive answer to this problem. The focus on our approach will therefore be on article link suggestions, instead of definitive answers. This gives us the nice property that lower confident results can also be included in the results presented by our approach.

To find missing links, our idea is to mimic how the contributors on Wikipedia link articles together. In other words, we want to learn how to link articles together by looking at how links are currently inserted/not inserted. This is is essentially the idea of machine learning, which we will be the main part of our solution idea. The following sections will expand on this idea in-depth.


%\section{Machine Learning}
%In this chapter we will describe the diffrent machine learning approches that has been taken into consideration and the reasoning behind.


\section{Machine Learning Task}\label{sec:machine_learning_task}
Before choosing which specific machine learning algorithms to use, we first consider which machine learning task would fit this project based on the problem we are trying to solve. In \cref{sec:problem_statement} we describe the problem of suggesting links between Wikipedia articles, part of which is to determine whether two Wikipedia articles should be linked. This problem can be seen as a binary classification problem.

As described in \cref{ch:introduction} there are $5000$ featured Wikipedia articles, which we assume to have appropriate linking. Under this assumption we can define a set of positive labeled training pairs $P$ as $\set{(a,b)\ |\ a\text{ is featured}\ \wedge\ a \to b}$ where $a \to b$ denotes that article $a$ has a link to article $b$. Likewise we define a set of negative labeled training pairs $N$ as  $\set{(a,b)\ |\ a\text{ is featured}\ \wedge\ a \not\to b}$, where $a \not\to b$ denotes article $a$ not having a link to article $b$.

By constructing the training data only from $P$ and $N$, all training pairs can be labeled as either linked or not linked. This further delimits the problem to be a supervised binary classification problem.
 
\subsubsection{Binary Classification}

The problem can be defined as a binary classification:

\paragraph{Input:}
A set of $m$ training examples $(x^j,y^j)$ for $j=1,2..m$, sampled from a distribution $f(E_j)$ of article pair features, with $x^j \in R^n$ being a feature vector and $y^j \in \set{-1,+1}$ a training label. A training label $y^j=+1$ will denote a \textit{positive sample} and $y^j=-1$ a \textit{negative sample}. The i-th feature of a sample $x^j$ defined $x^j_i$ represents a pair of articles and is a combination of article features generated by node2vec~\cite{node2vec} from the training pairs $P$ and $N$.

\todo{explain/generalize article combination without mentioning node2vec? (binary operation)}
\todo{beskriv hvilken objective function vi vil minimere/maksimere}
\todo{input: $x^j$ skal være par af artikler}

%node2vec:
%Given two nodes u and v, and a binary operator o' the feature vectors f(u) and f(v) are combined into the representation g(u,v).

\paragraph{Output:}
A function $f: R^n \to \set{-1,+1}$ that classifies additional samples ${x^k}$ sampled from $F$.



\section{new stuff}


\begin{itemize}
\item $G = (V, E)$: graph
\item $V$: set of all Wikipedia articles
\item $F \subset V$: set of all article pairs
\item $E \subseteq F \times V$: set of links from featured articles
\item $g$: objective function...
%\item $h: V \times V \to R^d$: feature function*
\item $h: E \to R^d$: feature function*
%\item $f: R^d \times \set{-1,1} \to \mathbb{R}$: 
\end{itemize}

%$P \subset E$

The graph $G$ represents articles and link structure.

%$G = (V, E)$ where $V$ is the set of all Wikipedia articles, and E is the set of links from featured articles. $E \subseteq F \times V$.

\subsection{more new stuff}

$h: V \times V \in R^d$ % feature of

$g(x) = arg max_y f(h(x), y)$

$g(x) = arg max_y Pr(y | h(x))$

where $f: R^d \times \set{-1,1} \to \mathbb{R}$, and $h: V \times V \to R^d$

$y \in \set{-1,1}$
%$y = \set{1 if x \in E, -1 if x \in E}$

\begin{equation}
  y^j =
  \begin{cases}
    1, & \text{if}\ j \in E \\
    -1, & \text{otherwise}
  \end{cases}
\end{equation}



