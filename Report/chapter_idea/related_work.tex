\section{Some Title}
\todo{needs to be reviewed so that it does not repeat too much from the previous introduction section. There are also some parts that maybe can be moved up to previous sections}



The content of Wikipedia is produced and maintained via crowdsourcing. That means that a large number of users collaboratively add and review content. As described in the previous section, it is non-trivial to find the correct articles to link when editing a Wikipedia article.

We will in this report investigate whether we can use computers to automate finding missing links on Wikipedia articles, in order to allow the user to focus on other parts of article authoring. The linking problem, namely whether two articles should be linked, is not easy to define in an objective manner, as it is inherently subjective; there is no definitive answer to this problem. The focus on our approach will therefore be on article link suggestions, instead of definitive answers. This gives us the nice property that lower confident results can also be included in the results presented by our approach.

To find missing links, our idea is to mimic how the contributors on Wikipedia link articles together. In other words, we want to learn how to link articles together by looking at how links are currently inserted/not inserted. This is is essentially the idea of machine learning, which we will be the main part of our solution idea. The following sections will expand on this idea in-depth.