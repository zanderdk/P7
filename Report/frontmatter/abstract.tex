%Linking is vital for navigation on the World Wide Web and provides structure to individual documents. However,
Maintaining link structure is difficult on large and often changing websites. We present a solution to the problem of suggesting missing links on Wikipedia in order to structurally improve the website as a whole. The problem is considered a supervised binary classification problem, and the core of the solution analyzes Wikipedia structurally in a link graph.

We populate a graph database with Wiki\-pedia articles as nodes and article links as edges. A nearest centroid classifier is trained to decide whether a link will be beneficial between two articles. Several other classification algorithms were tested. The classifier uses features extracted from nodes based on a model trained by the feature learning method node2vec. We also develop a web API that provides access to the suggestions. The solution achieves a precision score of \num{0.981} on a test set consisting of featured articles. While this result is promising, further tests on non-featured articles show that the solution does not generalize well to the whole of Wikipedia. Due to this, we provide suggestions for further improvement.


%Kom ind på:\\
%Structural analysis of Wikipedia link graph

%Machine learning

%Binary classification

%Feature learning

%Link Suggestions

%Evaluation

%UI - API