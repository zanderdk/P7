\section[MI Test]{MI Preformance}
% Første parameter i [] er tekst i header. {} er i indholdsfortegnelsen.

% Slide med emneoverskrift.
\begin{frame}
  \frametitle{}
  \begin{center}
    {\Huge MI Preformance}
  \end{center}
\end{frame}
\note{
  \begin{itemize}
		\item Notes...
  \end{itemize}
}

% Normal slide:
\begin{frame}
    \frametitle{MI Preformance}
    %\framesubtitle{Development Method}
    \begin{itemize}
      \item Test Results
      \item Overfitting
      \item Improvements
      \item New Test Results
    \end{itemize}
\end{frame}
\note{
  \begin{itemize}
    \item nerest centroid 97.89
  \end{itemize}
}

% Normal slide:
\begin{frame}
    \frametitle{MI Preformance}
    %\framesubtitle{Development Method}
    \begin{figure}[tb]%
    \centering
    \tikzsetnextfilename{barchart}
    \begin{tikzpicture}
      \begin{axis}[
        table/col sep=semicolon,
        xbar=0pt, xmin=0, xmax=1,
        yticklabels from table={10f-results.csv}{name},
        %yticklabel style={text height=1.5ex},
        ytick=data,
        width=0.6\textwidth,
        y=0.4cm,
        enlarge y limits={abs=0.6},
        bar width=4pt,
        /pgf/number format/fixed,
        axis lines*=left,
        xmajorgrids=true,
        legend entries={Recall, Precision},
        legend style={draw=none},
        reverse legend, area legend,
        legend style={at={(1,1.01)},anchor=south east}
      ]
      \addplot [fill=color1!20!white] table [y expr=-\coordindex, x=recall] {10f-results.csv};
      \addplot [fill=color1!70!white] table [y expr=-\coordindex, x=precision] {10f-results.csv};
      \end{axis}
    \end{tikzpicture}
  \end{figure}
\end{frame}
\note{
  \begin{itemize}
    \item nerest centroid 97.89
  \end{itemize}
}

\begin{frame}
    \frametitle{MI Preformance}
    %\framesubtitle{Development Method}
    \begin{itemize}
      \item Overfitting
      \begin{itemize}
        \item Poor preformance on manually labled data set
        \item Causation
        \begin{itemize}
          \item Not enogth training pairs
          \item Data set contains many automatically pages generated pages
        \end{itemize}
      \end{itemize}
    \end{itemize}
\end{frame}
\note{
  \begin{itemize}
    \item Lasse asumption
  \end{itemize}
}


\begin{frame}
    \frametitle{MI Preformance}
    %\framesubtitle{Development Method}
    \begin{itemize}
      \item For positive pairs
      \begin{itemize}
        \item For all featured articles a
        \begin{itemize}
          \item select all b where $a \rightarrow b$
          \item add $(a,b)$ to set of traning pairs
        \end{itemize}
      \end{itemize}
      \item For negative pairs pairs
      \begin{itemize}
        \item For all featured articles a
        \begin{itemize}
          \item select a random page b where $a \not \rightarrow b$
          \item add $(a,b)$ to set of traning pairs
        \end{itemize}
      \end{itemize}
    \end{itemize}
\end{frame}
\note{
  \begin{itemize}
    \item ??
  \end{itemize}
}


\begin{frame}
    \frametitle{MI Preformance}
    %\framesubtitle{Development Method}
    \begin{figure}[tb]%
    \centering
    \tikzsetnextfilename{barchart}
    \begin{tikzpicture}
      \begin{axis}[
        table/col sep=semicolon,
        xbar=0pt, xmin=0, xmax=1,
        yticklabels from table={new10f-results.csv}{name},
        %yticklabel style={text height=1.5ex},
        ytick=data,
        width=0.6\textwidth,
        y=0.4cm,
        enlarge y limits={abs=0.6},
        bar width=4pt,
        /pgf/number format/fixed,
        axis lines*=left,
        xmajorgrids=true,
        legend entries={Recall, Precision},
        legend style={draw=none},
        reverse legend, area legend,
        legend style={at={(1,1.01)},anchor=south east}
      ]
      \addplot [fill=color1!20!white] table [y expr=-\coordindex, x=recall] {new10f-results.csv};
      \addplot [fill=color1!70!white] table [y expr=-\coordindex, x=precision] {new10f-results.csv};
      \end{axis}
    \end{tikzpicture}
  \end{figure}
\end{frame}
\note{
  \begin{itemize}
    \item nerest centroid 97.89
  \end{itemize}
}
